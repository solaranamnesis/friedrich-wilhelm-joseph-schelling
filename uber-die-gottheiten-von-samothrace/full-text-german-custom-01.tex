\documentclass[a4paper, 11pt, oneside]{article}
\usepackage[T1]{fontenc}
\usepackage{ebgaramond}
% Load encoding definitions (after font package)
\usepackage{textalpha}

\usepackage[dvipsnames]{xcolor}
\usepackage{eso-pic,graphicx}
\usepackage[top=44mm, bottom=44mm, outer=27mm, inner=27mm]{geometry}
\setlength{\columnsep}{90pt}

% Babel package:
\usepackage[main=german,polutonikogreek]{babel}
\babelprovide[import]{hebrew}
\usepackage{arabtex}
\usepackage{cjhebrew}
\usepackage{svg}
\usepackage{listings}
\lstset{basicstyle=\ttfamily}

% With XeTeX$\$LuaTeX, load fontspec after babel to use Unicode
% fonts for Latin script and LGR for Greek:
\ifdefined\luatexversion \usepackage{fontspec}\fi
\ifdefined\XeTeXrevision \usepackage{fontspec}\fi

% "`Lipsiakos"' italic font `cbleipzig`:
\newcommand*{\lishape}{\fontencoding{LGR}\fontfamily{cmr}%
		       \fontshape{li}\selectfont}
\DeclareTextFontCommand{\textli}{\lishape}

\usepackage{booktabs}
\usepackage{graphicx}
\setlength{\emergencystretch}{15pt}
\graphicspath{ {./ } }
\usepackage[figurename=]{caption}
\usepackage{float}
\usepackage{fancyhdr}
\usepackage{microtype}

%define custom symbols
\newcommand*\symbolAAA{\includesvg[height=1em]{svgs-red/01.svg}}
\newcommand*\symbolAAB{\includesvg[height=1em]{svgs-red/02.svg}}
\newcommand*\symbolAAC{\includesvg[width=1em]{svgs-red/03.svg}}
\newcommand*\symbolAAD{\includesvg[height=1em]{svgs-red/04.svg}}
\newcommand*\symbolAAE{\includesvg[height=1em]{svgs-red/05.svg}}
\newcommand*\symbolAAF{\includesvg[height=1em]{svgs-red/06.svg}}
\newcommand*\symbolAAG{\includesvg[height=1em]{svgs-red/07.svg}}
\newcommand*\symbolAAH{\includesvg[height=1em]{svgs-red/08.svg}}
\newcommand*\symbolAAI{\includesvg[height=1em]{svgs-red/09.svg}}
\newcommand*\symbolAAJ{\includesvg[height=1em]{svgs-red/10.svg}}
\newcommand*\symbolAAK{\includesvg[height=1em]{svgs-red/11.svg}}
\newcommand*\symbolAAL{\includesvg[height=1em]{svgs-red/12.svg}}
\newcommand*\symbolAAM{\includesvg[height=1em]{svgs-red/13.svg}}
\newcommand*\symbolAAN{\includesvg[height=1em]{svgs-red/14.svg}}
\newcommand*\symbolAAO{\includesvg[height=1em]{svgs-red/15.svg}}
\newcommand*\symbolAAP{\includesvg[height=1em]{svgs-red/16.svg}}
\newcommand*\symbolAAQ{\includesvg[height=1em]{svgs-red/17.svg}}
\newcommand*\symbolAAR{\includesvg[height=1em]{svgs-red/18.svg}}
\newcommand*\symbolAAS{\includesvg[height=1em]{svgs-red/19.svg}}
\newcommand*\symbolAAT{\includesvg[height=1em]{svgs-red/20.svg}}
\newcommand*\symbolAAU{\includesvg[height=1em]{svgs-red/21.svg}}
\newcommand*\symbolAAV{\includesvg[height=1em]{svgs-red/22.svg}}
\newcommand*\symbolAAW{\includesvg[height=1em]{svgs-red/23.svg}}

\usepackage{setspace}
\onehalfspacing

% change color of text, example replace all \color{Goldenrod} with \color{lightgray}

\definecolor{red}{RGB}{172,0,25}

\makeatletter % change only the display of \thepage, but not \thepage itself:
\patchcmd{\ps@plain}{\thepage}{\color{red}\bfseries{\thepage}}{}{}
\makeatother

\color{red}

\begin{document}
\bfseries
\pagestyle{plain} % after changing a pagestyle command, it's necessary to invoke it explicitly
\AddToShipoutPictureBG{\includegraphics[width=\paperwidth,height=\paperheight]{samothrace2.jpeg}}

\renewcommand\thefootnote{{\color{red}\bfseries{\arabic{footnote}}}}
\let\oldfootnote\footnote
    \renewcommand{\footnote}[1]{\oldfootnote{{\bfseries\color{red}#1}}}
\begin{titlepage} % Suppresses headers and footers on the title page
	\centering % Centre everything on the title page
	%\scshape % Use small caps for all text on the title page

	%------------------------------------------------
	%	Title
	%------------------------------------------------
	
	\rule{\textwidth}{1.6pt}\vspace*{-\baselineskip}\vspace*{2pt} % Thick horizontal rule
	\rule{\textwidth}{0.4pt} % Thin horizontal rule
	
	\vspace{1\baselineskip} % Whitespace above the title
	
	{\scshape\Huge Über die Gottheiten von Samothraki}
	
	\vspace{1\baselineskip} % Whitespace above the title

	\rule{\textwidth}{0.4pt}\vspace*{-\baselineskip}\vspace{3.2pt} % Thin horizontal rule
	\rule{\textwidth}{1.6pt} % Thick horizontal rule
	
	\vspace{1\baselineskip} % Whitespace after the title block
	
	%------------------------------------------------
	%	Subtitle
	%------------------------------------------------
	
	{\scshape \Large Beilage zu den Weltaltern \\Von \LARGE Friedrich Wilhelm Joseph von Schelling} % Subtitle or further description
	
	\vspace*{1\baselineskip} % Whitespace under the subtitle
	    
	%------------------------------------------------
	%	Editor(s)
	%------------------------------------------------
        \vspace*{\fill}

	\vspace{1\baselineskip}

	{\small\scshape Stuttgart und Tübingen 1815}
	
	{\small\scshape{In der J. G. Gotta'schen Buchhandlung}}
	
	\vspace{0.5\baselineskip} % Whitespace after the title block

        \scshape Internet Archive Online Edition  % Publication year
	
	{\scshape\small Namensnennung Nicht-kommerziell Weitergabe unter gleichen Bedingungen 4.0 International} % Publisher
\end{titlepage}
\setlength{\parskip}{1mm plus1mm minus1mm}
\clearpage
\large
\tableofcontents
\clearpage
\section{Über die Gottheiten von Samothraki.}
\paragraph{}
Im Norden des ägäischen Meers erhebt sich die Samothrakische Insel, von Anfang wie es scheint Samos genannt, darauf zum Unterschiede von der ionischen und wegen der Nähe Thrakiens die thrakische.\footnote{\emph{Samothraki --- attollitur. Plin. N. Hist. 4. 23. p. 24. ed. Hard.} heutzutag Samothraki: den Mündungen des Hebrus und Lyssus gegenüber, nahe am Eingang des Melaeischen Meerbusens. Als älterer Name soll nach \emph{Schol. Apoll. Arg. v. 917.} Leukosia von Aristoteles erwähnt werden ἐν τῆ Σαμοθράκης πολιτεία; als solchen geben nebst Pausanias \emph{Achaic. 50. 4. p. 530. ed. Kuhn} mehrere andere auch Dardania an. Über die Ursachen des angeblich späteren Namens Samos finden sich abweichende Erklärungen, die sich zum Teil schon durch die homerische Erwähnung widerlegen. Das Wahrscheinlichste hat Strabo gesehen, \emph{Geogr. 50. 10. p. 457. ed. Paris}, von der Höhe ihrer Berge haben diese sowohl als die beiden andern Samos den Namen erhalten, ἐπειδὴ Σάμους ἐκάλουν τὰ ὕψη. Er mochte dabei an σῆμα \emph{dor.} σᾶμα denken: Bochart, \emph{Geogr. S. 50. 1. c. 8.} leitet dieselbe Bedeutung wahrscheinlicher aus morgenländischen Wurzeln ab. Vgl. die Meinung einiger Ausleger zu \emph{Genes. 2, 4.} und Münters Erklärung einer Inschrift, die auf die Samothrac. Geh. Bezug hat, Copenh. 1810. S. 29. Demnach würde der Name Samos schon den Zeiten angehören, da Phönikier diese Meere durchschifften, und nur der Zusatz thrakische das Spätere sein, das zusammengezogene Samothraki, das Homer noch nicht kennt, (s. die Bemerkung in \emph{Hesych. 2. p. 1148. ed. Alb.} und Virgils Ausdruck: \emph{Threiiciamque Samon, quae nunc Samothracia fertur}) das Allerspäteste. Diess ist die Meinung der im Text ausgedrückten Stelle des Strabo in \emph{Exc. 50. 7. extr.} ἐκάλειτο δὲ ἡ Σαμοθρακη Σάμος πρίν. Der Grund des Namens Leukosia mag auf sich beruhen, wie des auffallenderen \emph{Melite} (Maltha) den Strabo nennt 10. p. 472; \emph{Dardania} (nämlich \emph{insula}) ist aber so wenig Name, als \emph{Electria} (\emph{tellus}) bei \emph{Val. Flacc. Arg. 2. 431.} und hiernach zu beurteilen Plinius \emph{l. c.} "`\emph{Callimachus} eam antiquo nomine Dardaniam vocat.''} Erdkundige des Altertums schon vermuteten große Naturerschütterungen, die diese Gegenden noch zu Menschen-Zeiten betroffen. Es sei, dass durch bloße Anschwellung gehoben die Wasser des Euxinischen Meers erst die thrakische Meerenge, dann den Hellespontes durchbrochen,\footnote{So meinte namentlich Strabo, bei \emph{Strab. Geogr. 1. p. 49} die Anschwellung sei durch die einströmenden Flüsse nach und nach bewirkt worden.} oder dass die Gewalt eines unterirdischen Feuers den Stand der Gewässer verändert\footnote{Der Beiname des Erderschüttrers, den Poseidaon in den homerischen Gedichten führt, und die fortdaurenden Wirkungen des unterirdischen Feuers in jenen Gewässern machen eine solche Verbindung glaublich. Seit Menschengedenken (ohngefähr 237 J. v. C. G.) wurde durch dasselbe Therasia von Thera getrennt, kurz darauf zwischen beiden die neue Insel Hiera (die heilige; nach \emph{Choiseul-Gouffier} jetzt die große Cammeri) unter einer bis nach Rhodus verspürten, vielen Städten Asiens verderblichen Erschütterung emporgehoben, und dieser 46 J. v. C. G. die neue Insel Thia (die göttliche) hinzugefügt, \emph{Plin. 4. 23. p. 213}; im Anfang des 16ten folgte die kleine Cammeri, im Anfang des vorigen Jahrhunderts wurde in derselben Gegend eine neue Insel von 5 Meilen im Umfang, mit Erdbeben, unterirdischem Donner und Feuer emporgetrieben. Davon \emph{Choiseul-Gouffier Voy. pitt. de la Gr. T. 1. Missions du Levant T. 2.}}: die ältesten samothrakischen Erzählungen, die durch aufgezeigte Denkmäler in Erinnerungen sich verwandelten, erhielten eine Kunde dieser Ereignisse und bis in jene Zeiten hinauf rückten sie die Verehrung und den Schutz der vaterländischen Götter.\footnote{Damals als große Strecken Asiens für immer, andre für eine Zeitlang bedeckt worden, sein, so erzählten die Einwohner, auch die Niederungen Samothrakis überschwemmt worden; auf den höchsten Berggipfeln haben sie unter steten Gelübden gegen die vaterländischen Götter Hülse gesucht. Noch stehen, setzt \emph{Diod. Sic. 5. 47. p. 357. ed. Wessel.-Bip.} hinzu, im Umfang der ganzen Insel Altäre, welche die Grenzen der damaligen Gefahr und der Rettung bezeichnen.} Die Schrecken dieser Erinnerungen wurden erhöht durch die stets gegenwärtigen Schauer einer großen und mächtigen Natur; von Wäldern bedeckt bildete das fast unzugängliche Eiland\footnote{\emph{Vel importuosissima omnium. Plin. p. 214.} Daraus macht \emph{Sainte-Croix, Recherches sur les mystères du paganisme, p. 32. "`absolument sans port}, (wahrscheinlich den Zusatz zu erhöhen: \emph{la superstition seule engageoit à y aborder}), gleich unverträglich mit dem Superlativ, und mit \emph{Liv. Hist. 45. 6. "`Demetrium est portus in promontorio quodam Samothracae} (im Norden der Insel meint \emph{ChoiseulGouffier T. 2. p. 123.} wo er noch Spuren des Cerestempels zu erkennen glaubte) und \emph{Plut. in Vit. Paul. Aem. c. 26.}} nur Ein zusammenhangendes Gebirge\footnote{Diess zeigt der homerische Ausdruck: ἐπ’ ἀκροτάτῆς κορυφῆς Σάμου ὑληέσσης \emph{Il. 13, 12}, der schon erwähnte Plinius S. \emph{attollitur monte Saoce decem millia passuum altitudinis}, auch dass der Name dieses Bergs (Σαωκὶς) als Name der ganzen Insel galt, \emph{Hesych. 50. v. p. 1161.} Zu wenig ist uns übrigens von der Naturgeschichte Samothrakis bekannt. Ob das den Kabiren und der Hekate besonders heilige \emph{Zerynthium antrum} auf der Insel selbst oder an der Küste von Thracien lag, ist ungewiss, denn auch einige Städte des festen Landes und ihre Gebiete gehörten zu Samothracien. \emph{Lucret 6. 1042. 44. Exultare etiam Samothracia ferrea vidi --- --- lapis hic Magnes cum subditus esset}. Ob bloß aus samothrakischen Bergwerken (der Insel oder des festen Landes) gewonnenes Eisen gemeint ist, oder Idole, wie \emph{Turneb. Adv. 50. 20. c. 2.} will, oder (wahrscheinlicher) eiserne Ringe, Amulete, Talismane (an Boussolen wird niemand denken), die von dort kamen, ist nicht ganz zu entscheiden.} von dessen höchstem Gipfel während des Kampfs um Troja Poseidon die ganze Bergkette des Ida, des Priamus Stadt und der Danaer Schiffe überschaut.\footnote{Aber nicht achtlos lauschte der Erderschüttrer Poseidon, Denn er saß anstaunend den Kampf und die Wassen-Entscheidung, Hoch auf dem obersten Gipfel der grün-umwaldeten Samos Thrakias, dort erschien mit allen Höh'n ihm der Ida, Auch erschien ihm Priamos Stadt und der Danaer Schiffe. Il. 13, 10. ff. nach Voß.} Dort ward in unbestimmbarer Vorzeit ein geheimnisvoller Götterdienst gestiftet und wenn die vielreiche ionische Samos des göttlich geachteten Mannes sich rühmt, der zuerst einen höhere Menschlichkeit beabsichtenden Bund ersonnen, so ist die unansehnliche Samos Thrakiens herrlicher in der Geschichte der Menschheit durch den Dienst der Kabiren, dem ältesten des ganzen Griechenlandes, der mit dem ersten Licht höheren und besseren Wissens in diesen Gegenden aufging, und der nicht eher als zugleich mit dem alten Glauben selbst untergegangen scheint. Aus den Wäldern Samothrakiens erhielt Griechenland mit der geheimeren Göttergeschichte zuerst den Glauben an ein zukünftiges Leben. Besser und für das Leben wie für den Tod fröhlicher wurden nach allgemeiner Überzeugung die dort Eingeweihten.\footnote{\emph{Diod. Sic. 1. 49. p. 262. 63.}} Eine Zuflucht des Unglücks, ja des Verbrechens, soweit es durch Bekenntnis und Entsündigungen versöhnt werden mochte,\footnote{Auf die sogenannte Beichte wird daraus geschlossen, dass keiner einen Götter-Ausspruch erhalten konnte, ohne die unrechtmäßigste Handlung seines Lebens bekannt zu haben, wie von Lysander gefodert wurde, (\emph{Plutarch}) \emph{Apopht. Lac. Opp. ed. Wytt. Oxon. 1. p. 639.} Aber eben diese Frage zeigt, dass es Verbrechen gab, die nicht erlaubten, sich den Göttern zu nähern. \emph{Hesych. 2. p. 293.} sagt unbestimmt: Κοίης ἱερεὺς Καβείρων, ὁ κα καθαίρων φονέα, aber alle Umstände, besonders die Eleusinische Strenge, lassen vermuten, dass nur unverschuldeter Mord der Versöhnung fähig war.} hielt, in Zeiten früherer und späterer Wildheit, samothrakischer Gebrauch menschliches Gefühl aufrecht. Kein Wunder, dass der Name der heiligen Insel\footnote{\emph{Sacram hanc insulam et august. totam atque inviolati soli esse}, sagt der Römer L. Atilius in der Anrede an die Samothracier, \emph{Liv. l. 1.} Sogar der Name bedeute ἱερὰν νῆσον, ist eine von Diodor 3. p. 324. erwähnte Meinung.} mit allem verwebt wurde, was die ältesten Geschichten Ehrwürdiges und Ruhmvolles aufbewahrten. Jasion und Dardamus, Orpheus und die Argonauten, Herkules auch und Ulysses, sollen teils den geheimen Dienst dort geordnet, teils die Weihen empfangen haben.\footnote{\emph{Diod. 50. 5. c. 49. extr. Ap. Arg. 1. 915. ss. Orph. Arg. 465.} Von Odysseus s. \emph{Schol. Ap. l. c.}} Den Pythagoras nennt eine weder unwahrscheinliche noch unwichtige Nachricht unter denen, die Weisheit dort suchten und fanden.\footnote{\emph{Jambl. in Vit. Pyth. c. 28.}} Bei den kabirischen Orgien sahen sich zuerst der makedonische Philipp und, noch Kind, die Mutter des großen Alexander Olympias, vielleicht nicht ohne Einfluss auf das künftige Schicksal ihres Sohns.\footnote{Plutarch, \emph{vit. Al. c. 2.} erwähnt ausdrücklich, dass diese Frau ihr ganzes Leben der orphischen und bacchischen Begeisterung ergeben gewesen, ja sogar zu den Klodonen und Mimallonen gehört habe, (s. von diesen Kreuzers Symbolik und Mythologie Th. 3. S. 208. ff.). Ich weiß nicht, ob die Vermutung schon geäußert worden, dass dieser von der Mutter auf den Sohn, ihrer unbewusst, übertragene dionysische Anhauch es war, der den trunkenen Jüngling über den Indus führte.} Selbst der Römer Herrschaft schonte der Freiheit und alten, wie es scheint, theokratischen Verfassung Samothrakis\footnote{\emph{Samothraki, quae libera. Plin. l. c.} Der höchste Priester scheint sich als Herr des Landes betragen zu haben, "`\emph{Obvius-terris adytisque Sacerdos Excipit"' Val. Fl. 2. 437. 38.} Auch der Ausdruck des Livius: \emph{Theonda, qui summus magistratus apud eos erat} (\emph{regem ipsi adpellant}) deutet dahin.}; dort suchte seines Reiches beraubt der letzte makedonische König eine Freistatt, aus der ihn nicht Gewalt der schon übermütig herrschenden Römer, sondern die Heiligkeit des Orts selbst und der am eigenen Feldherrn verübte Mord austrieb.\footnote{\emph{Liv. l. c. Plutarch, Paul. Aem. c. 26. in.}, sagt ausdrücklich, den Schutz des Asyls (ἀσυλίαν) habe Cn. Octavius ihm gelassen, nur vom Meer und der Entweichung ihn abgeschnitten.} Dort hätte, wenn nicht durch Nordstürme zurückgetrieben, kurz vor seinem Ende noch der edle Germanicus die Weihen empfangen.\footnote{\emph{Tacit. Ann. 2. 53. extr.} Die Verfolgung des Mithridates versäumte über der Einweihung in Samothraki der römische Befehlshaber Voconius, \emph{Plut. in Luc. c. 13.} Mehr in Forschungs-als Einweihungs-Absichten hatte sich dort auch der große Gelehrte M. Terentius Varro umgesehn, Anm. 112.} Schriftsteller aus späterer Kaiserzeit erwähnen der samothrakischen Heiligtümer im Genuss fortdauernder Verehrung\footnote{Plinius in der unten (Anm. 46.) angeführten Stelle.} und wenn auch nicht in noch bestehenden heiligen Gebräuchen der Altertumsforscher Züge aus dem Bild des alten samothrakischen Dienstes zu erblicken meinte,\footnote{Dahin gehört der Dienst der Knaben am Altar, die Einweihung auch der Kinder, \emph{Donat. ad Terent. Phorm. Act. 1. Sc. 1. Terentius Apollodorum sequitur, apud quem legitur, in insula Samothraki a certe tempore pueros inititatos.} Vgl. \emph{Meursii Eleus. Opp. 2. p. 502.}} so würde man andern Spuren zufolge seine Fortdauer bis zum Ende des zweiten ja wohl bis ins dritte Jahrhundert der christlichen Zeitrechnung verfolgen können.\footnote{Münter setzt die auf samothrakische Weihen sich beziehende Inschrift mit guten Gründen an's Ende des zweiten oder ins dritte Jahrhundert. Die unbedingte Aufhebung der Mysterien überhaupt erfolgte bekanntlich erst unter Theodosius, \emph{Sainte-Croix p. 501.}} Wenn einst, da mehr als je wieder die öffentliche Aufmerksamkeit dem alten Griechenlande sich zuwendet, dies fast vergessene Eiland gleich andern durchforscht würde, vielleicht dass nicht Schätze der Kunst wie jener unvergleichliche Fund von Ägina, aber Denkmäler des ältesten Glaubens, wichtiger noch als jener für die ganze Geschichte unseres Geschlechts, die Ausbeute solcher Nachsuchungen würden.\footnote{Dasselbe äußert \emph{Choiseul-Gouiffier 2. p. 123.}} Einer wiederholten Betrachtung schien dieser geheime Götterdienst zwar in jeder Hinsicht, aber zumal aus folgenden Gründen, nach so vielen Untersuchungen nicht unwürdig. In Dunkel gehüllt ist noch die Bedeutung der einzelnen Gottheiten. Zwar ihre griechische Namen nennt mehr als Ein Schriftsteller. Wir wissen, dass Demeter, Dionysos, Hermes, auch Zeus als Kabiren verehrt wurden. Aber dies sind für uns bloße Namen, die den Zweifel übriglassen, ob die samothrakischen Götter jenen bekannten Gottheiten etwa nur ähnlich und vergleichbar, oder mit ihnen wirklich und dem Grundbegriff nach Eins gewesen. Ebenso ungewiss ist, wodurch sich diese Götter als Gegenstände der Geheimlehre von denselben Göttern im öffentlichen Dienst und allgemeinen Glauben unterschieden. Und doch vermag nur diese vereinte Kenntnis gründlichen Aufschluss zu geben über den Sinn der samothrakischen Lehre, über das eigentliche, ihr zu Grunde liegende, System. Eine einzige durch besonderes Glück gerettete Nachricht\footnote{\emph{Schol. Apoll. Arg. v. 917.}} scheint mit den wahren vom ersten Ursprung sich herschreibenden Namen zugleich die urkundliche Zeit- und Geburtsfolge der samothrakischen Götter erhalten zu haben. Billig schien daher, diese allen Untersuchungen zum Grunde zu legen. So lautet die Stelle des griechischen Auslegers, dem wir die Erhaltung jener Nachricht verdanken. In Samothraki empfängt man die Weihen der Kabiren. Mnasras sagt, es sein deren drei der Zahl nach, Axieros, Axiokersa, Axiokersos. Axieros sei die Demeter, Axiokersa die Persephone, Axiokersas aber der Hades. Einige fügen auch einen vierten hinzu, Kaswilos genannt, welcher, wie Dionysodoros erzählt, Hermes ist.\footnote{Ob zufällig oder aus irgendeinem Grund setzt Kreuzer 2. 294. den Axiokersos vor der Axiokersa. Kaum erkennt man noch die Stelle bei \emph{Sainte-Croix p. 27.} In der Hauptsache ist sie oben nach den Pariser Scholien gegeben, welche zeigen, dass auch die Auslegung der Namen des alten Geschichtsschreibers ist nicht des Scholiasten. Über den Geschichtsschreiber Mnaseas sind ausführliche Nachweisungen in \emph{Gerh. Vofs. de Hist. gr. Opp. 4. p. 96. b.}, die auch in Bezug auf die gegenwärtige Untersuchung verfolgt zu werden verdienen.} Auf die Folge dieser Persönlichkeiten, auf die einer jeden zukommende Zahl legt diese Stelle ein deutliches Gewicht; zugleich da sie die ursprünglichen Namen enthält, gibt sie Anlass zur Vergleichung und zu Erforschung des jeder Gottheit zu Grunde liegenden Begriffs. Denn gewagt, ja fast frevelhaft scheint es, die Namen von dem alten Geschichtsschreiber zu nehmen, die Bedeutung aber aus anderen Quellen, ganz unabhängig, erforschen zu wollen. Aller Grund ist anzunehmen, dass der die verborgenen Namen gewusst, auch der Bedeutung im Allgemeinen nicht unkundig gewesen. Dass sie nicht hellenischem, dass sie, nach Griechen-Weise zu reden, barbarischen Ursprungs sind, ist anerkannt\footnote{Den Verfasser der \emph{Recherches sur les Cabires} in den \emph{Mém. de l'Ac. des Inscr. T. 27.}, obwohl dessen Erklärungen aus dem Griechischen auch \emph{Sainte-Croix p. 27.} wiederholt, wird niemand zählen; der würde selbst dem Herodot widersprechen, welcher versichert, alle griechische Götternamen, mit wenigen Ausnahmen, kommen von den Barbaren her.}; welcher Zunge, welchem Volk sie ursprünglich angehört, dies ist eine von jeder geschichtlichen Voraussetzung unabhängige, einer rein sprachlichen Entscheidung fähige Frage. Dem fleißigen sein ganzes Leben in ägyptischen Forschungen vergrabenen Zoëga war natürlich, den Wurzeln dieser Namen in den zweifelhaften Überbleibseln altägyptischer Sprache nachzuspüren; wenn er aber durch seine Forschungen statt bestimmter, entschiedene Persönlichkeiten bezeichnender, Bedeutungen die allgemeinsten und unbestimmtesten zu Tage fördert, wenn Axieros den Allmächtigem, Kaswilos den vollkommen Weisen bedeuten soll,\footnote{\emph{Zoëga de Or. et Usu Obelisc. p. 220. Not.}} so wird durch solche Erklärungen die Ableitung allein schon verdächtig. Ob ans indischen Sprachschätzen ein mehr genügender Aufschluss möglich ist, sei dahingestellt; wir glaubten einen andern von früheren Forschern betretenen Weg wieder einschlagen zu müssen. Welches Volks auch immer die Namen und die dadurch bezeichneten Götter ursprünglich sein mögen, eines vorzugsweise die Meere beschaffenden Volkes waren sie gewiss. Denn das ist der allgemeinste Glaube, dass jene Götter zumal den Seefahrenden hülfreich und heilbringend sein.\footnote{Die Argonauten bei beiden Dichtern suchen die samothrakischen Weihen, um glücklicher zu schaffen, lästig wär' über eine so bekannte Sache Stellen zu häufen. Personen, die Stürmen entkamen, hiengen in S. Votiv-Tafeln auf, wie das bei andern Gelegenheiten oft wiederholte Wort des Diagoras zeigt, dem man auch Schuld gibt, die kabirischen Geheimnisse veröffentlicht zu haben, \emph{Cic. d. n. D. 3. 37.}} Der Ursprung wie die Beständigkeit dieses Glaubens, lässt schwerlich eine andere Erklärung zu, als dass sie zuerst als die Götter eines zur See unternehmenden also begünstigt scheinenden Volkes bekannt wurden. Wie natürlich auch, dass fern herschaffende Fremdlinge in jenen noch jetzt nicht ungefährlichen Meeren, auf dessen Inseln außer den Geschäften des Handels Unwetter und Stürme sie oft lange Zeit zurückhalten mussten, die heimischen Götter wieder zu finden und zu verehren wünschten, dass also dieselben Schiffe, welche Rauchwerk, Purpur oder Elfenbein dahin führten, auch ihre Götter und Heiligtümer auf die griechischen Küsten und Eilande verpflanzten.\footnote{S. Münters ang. Schr. S. 14. vgl. mit Jacobs über die Memnonien, Denkschr. der Akad. 1809. S. 18.} Ein solches Volk in jenen Urzeiten kennen wir nur an den Phönikiern, deren lange dauerndes Wirken und Walten, ja deren Ansiedelungen in jenen Gegenden geschichtlich nicht zu läugnen sind.\footnote{Auf Thasus, der Samothraki nordwestlich zunächst liegenden Insel, sah Herodot einen Tempel des Herakles, erbaut von den nach der entführten Europa ausgegangnen Phönikiern, "`wohl fünf Menschenalter vor dem griechischen Herakles, des Amphitryons Sohn,"' 50. 2. c. 44. dort bewunderte er noch die Goldbergwerke, welche die Phönikier unter Thasus, von dem die Insel benannt wurde, zuerst eröffnet hatten, 50. 6. c. 47.} Dazu kommt die Versicherung des Herodot, den Schutzgöttern der Phönikier, deren Zeichen sie an den Vorderteilen der Schiffe führten, sei'n der Gestalt nach ähnlich gewesen die ägyptischen Kabiren.\footnote{50. 3. c. 37.} Hat demnach Samothraki seine Götter, mittelbar oder unmittelbar, von Phönikischen Schiffern erhalten, und waren des Volks, des die Götter, nach aller Wahrscheinlichkeit, auch die Namen, so ist der mächtigste Grund vorhanden, der Bedeutung jener Namen in phönikischen, oder was bei der unbestreitbaren Einerleiheit beider Sprachen völlig gleichgültig ist, in hebräischen Sprachwurzeln nachzuspüren. Denn dass Götternamen des Morgenlandes, wo selbst menschliche Eigennamen, bedeutend sind, bedarf des Beweises nicht und kaum der Erinnerung. Wir betreten also jenen gefährlichen Weg der Sprachforschung, der sich mit Untersuchung der Herkunft und Abstammung von Namen oder Wörtern abgibt, nicht unwissend, was von dessen Schwierigkeit und Undankbarkeit vorsichtige Kenner zu äußern pflegen, nicht unkundig des von minder Bedächtlichen im Allgemeinen darüber ausgesprochenen Verdammungsurteils. Aber jede Forschung ist löblich an sich, den Unterschied macht nur die Art und das Verfahren. Mochte in einer Zeit, da leicht jeder Sache sich gewachsen glaubte, eine neue Wut von Sprachableitungen alles aus allem zu machen und auf wahn-witzige Art auch in der alten Götterfabel alles mit allem zu vermischen bemüht sein: die Untersuchung der Herkunft und Abstammung der Wörter nicht blindlings, sondern kunstmäßig und nach den auch ihr zukommenden Regeln getrieben, wird immer der edelste Teil der Sprachforschung bleiben.\footnote{Die Unsicherheit etymologischer Erklärungen, zumal der Götternamen, kommt hauptsächlich davon, dass jede Gottheit gar mancher und sehr verschiedener Eigenschaften fähig ist. Es müsste sonderbar zugehen, wenn die Etymologie nicht irgendeine Bedeutung jedes Namens herauszubringen wüsste, die mit irgendeiner Eigenschaft der Gottheit übereinstimmte. nötig vor allem also ist, dass der Forschende den Grundbegriff einer Gottheit, gleichsam die Wurzel aller ihrer Eigenschaften, kenne: sonst werden ihm vielleicht Herleitungen in Menge zuströmen, deren keine eine eigentliche Überzeugung mit sich führt, indes er die wahre, auch wie sie ihm gleichsam von selbst sich darbietet, vorübergeht, weil ihm für den daraus sich ergebenden Sinn der Begriff mangelt. Diese Grundbegriffe werden aber nur durch die Stelle bestimmt, welche jede Gottheit im allgemeinen Göttersystem einnimmt: wer also von diesem nicht wenigstens die Grundzüge erkennt, würde nur raten und aufs Geratewohl versuchen, aber ohne zu irgendeiner Gewissheit zu gelangen, noch häufigen Fehlgriffen zu entgehen. So wenn Bochart den Namen Axieros aus dem hebräischen \foreignlanguage{hebrew}{\<'.hzy 'r.S>}, \emph{Achsi-Eres}, Mein ist die Erde, erklärt, so ist dies freilich leicht genug, aber im Grunde sind wir damit nicht mehr gefördert, als der Grieche, wenn er in seiner Δημήτηρ eine Γημήτηρ, Erdmutter, suchte. Ceres ist wohl auch die Mutter-Erde, aber dies ist ein abgeleiteter, nicht der ursprüngliche Begriff. Begnügt man sich aber vollends mit allgemeinen Begriffen, wie \emph{magnipotens, perfecte sapiens} u. ähnl., wo ist noch einige Sicherheit der Erklärung, wo noch eine Spur der Bestimmtheit und Schärfe, die wir in allen Begriffen des Altertums antreffen? Dass man die Sprache, in welcher etymologisiert wird, nicht bloß aus Wörterbüchern, sondern aus den Quellen und von den ersten Wurzeln her kenne, sollte sich von selbst verstehen. Aber auch damit ist nicht auszureichen, ohne die noch feinere Kenntnis dessen, was die Grammatiker die \emph{proprietatem verborum} nennen; denn es kann manchem Wort eine Bedeutung sehr zufällig oder doch nicht in dieser besondern Beziehung zukommen, in welcher sie ihm der gegenwärtigen Erklärung nach beigelegt wird. Nützlich ja nötig wird auch dem etymologischen Erklärer von Götternamen sein, auf die Analogie der Eigennamen in derselben Sprache zu achten, aus der erklärt wird. Inwiefern ich nun selbst diesen Voraussetzungen und Forderungen in den folgenden Erklärungs-Versuchen genügt, mögen Kenner beurteilen.} Das also den drei ersten Gottheiten gemeinschaftlich vorgesetzte Wort können wir als nicht bezeichnend für die besondere Natur einer jeden mit Schweigen übergehen.\footnote{Bochart \emph{Geogr. S. 50. 1. c. 12.} erklärt dieses Wort aus dem hebräischen \foreignlanguage{hebrew}{\<'.hz>}, wobei er für sich hat, dass es wirklich zur Zusammensetzung von Namen gebraucht worden, wie in Achasias 1 \emph{Reg. 22, 4}. Wäre von Einem Namen die Rede, so möchte es hingehen, aber für drei ist der Begriff zu beschränkt. Weit eigentlicher scheint das Wort \foreignlanguage{hebrew}{\<'.h/s>}, welches in Verbindung mit dem Namen jenes alttestamentlichen Perserkönigs Achas-Weros \emph{Esth. 1, 1.} aber auch in andern Zusammensetzungen, wie in \foreignlanguage{hebrew}{\<'.h/sdrpkym>} \emph{Esth. 8, 9.} und: \emph{b. v. 10.} \foreignlanguage{hebrew}{\<'.h/strkym>}, vorkommt, wo es nur eine Bezeichnung des Amts, der Würde oder der Trefflichkeit überhaupt sein kann. Man beruft sich deshalb auf das persische $\symbolAAA$, \emph{dignitas, majestas}, wobei es denn wohl vorerst bleiben mag.} Nach der wörtlichsten Übertragung aber kann der erste Name, Axieros, in phönikischer Mundart nicht wohl etwas anderes bedeuten, als den Hunger, die Armut, und was daraus folgt, das Schmachten, die Sucht\footnote{Die hebräische Wurzel \foreignlanguage{hebrew}{\<yr/s>} hat zwar gewöhnlich die Bedeutung des Besitzens (zumal durch Erbschaft); auch diese wäre nicht zu verschmähen. Allein die Stellen \emph{Prov. 20, 13. 30, 9.}, wo es den Gegensatz vom Sattsein bildet, \emph{Gen. 45, 11.}, wo das \emph{Passivum} die Bedeutung hat: durch Mangel verzehrt worden, sind hinlängliche Beweise, dass es die Bedeutung der verwandten Wurzel \foreignlanguage{hebrew}{\<rw/s>} (wovon \foreignlanguage{hebrew}{\<ry/s>} \emph{paupertas, egestas}) teilt und der Begriff des Mangels, des Hungers der erste ist, dem der des Ansichziehens, Festhaltens, Besitzergreifens erst folgt. Hebräisch geschrieben würde demnach der Name \foreignlanguage{hebrew}{\<'aha:+siUerO+s>} heißen, welches nach der bei Übertragung von Eigennamen immer beobachteten gelinderen Aussprache buchstäblich \emph{Achsieros} lautete. Und so wär' es denn am Ende wohl gar der Name Achas-Weros selbst, nur nach einer andern Mundart \emph{Lud. de Dieu in Annot. ad Esth. 1, 1.} wollte diesen aus dem schon angeführten persischen $\symbolAAB$ und dem Wörtchen $\symbolAAC$ erklären, dass im Persischen bedeute was im Arabischen $\symbolAAD$, also \emph{dominus majestatis}; vielleicht vergaß er in dem Augenblick, dass er ein hebräisches Wort vor sich hatte, denn die Endsylbe \emph{os} mit \emph{Hyde Hist. rel. vet. Pers. (Ed. Ox. 2dæ) p. 43.} wirklich für die ins Hebräische aufgenommene griechische Endigung zu halten, wird schwerlich jemand geneigt sein. Andre nicht genügendere Erklärungen wird man in \emph{Simonis Onomast. 5. T. p. 579.} finden. Die Sylbe \emph{os} gehört unstreitig zur Wurzel und diese kann nur \foreignlanguage{hebrew}{\<wAra+s>} sein, gleich dem arab. $\symbolAAE$, \emph{concupivit, avidus fuit, avide voravit aliquid de cibo}. Die andre Bedeutung von \foreignlanguage{hebrew}{\<yr/s>}, \emph{possedit}, findet sich nach einer sehr gewöhnlichen Teilung in der anderen entsprechenden Wurzel $\symbolAAF$. Die beiden Namen sind also gleichbedeutend und \foreignlanguage{hebrew}{\<yerO+s>} dieselbe Form mit \foreignlanguage{hebrew}{\<werO+s>}. Eine dritte auf die Wurzel \foreignlanguage{hebrew}{\<rw/s>} deutende Form ist das abgekürzte \foreignlanguage{hebrew}{\<'a:ha+s:re+s>} \emph{Esth. 10. 1.} Zum Namen eines Perserkönigs konnte das Wort ohne Rücksicht auf seine Bedeutung eben dadurch werden, dass es Name einer Gottheit war, denn von Göttern nahmen die Perserkönige häufig ihre Namen an, s. \emph{Golius ad Alferg. El. astr. p. 21. Herbelot Bibl. or. voc. Baharan.} Aber wie? von einer weiblichen Gottheit ein männlicher Königsname! Warum nicht? Zunächst wegen der Geschlechts-Zweideutigkeit aller Gottheiten, vermöge der weibliche Gottheiten wohl auch männlich gedacht wurden. Man erinnere sich an den cyprischen Ἀφρόδιτος, Kreuzer 1. 350., der altitalischen \emph{Almus Venus}, Kreuzer 2. 431, den Münzkennern nicht fremden \emph{Deus Lunus}, und, was hieher vielleicht die nächste Beziehung hat, den \emph{Cerus manus} der saliarischen Gedichte, der als männlicher Stellvertreter der weiblichen Ceres nicht zu verkennen ist. \emph{Joseph. contr. Ap. 50. 1. p. 449. ed. Haverc.} erwähnt unter den Königen von Tyrus einen \emph{Astartus}, was wohl nicht statt \emph{Abdastartus} sein kann, da ein andrer dieses Nameus kurz zuvor erwähnt wird. Wie aber ein in den kabirischen Mysterien gebräuchlicher Göttername Name eines Perserkönigs sein konnte; diese Frage gehört in ein ganz anderes Gebiet von Untersuchung. Vgl. inzwischen die 113re Anm.}; eine Erklärung, die auf den ersten Blick wunderlich scheinen mag, aber durch tiefere Betrachtung einleuchtet. Wir wollen nicht mit dem Allgemeinen uns begnügen, dass ein schlechthin erstes Wesen, wenn auch an sich überschwengliche Fülle, doch sofern es nichts hat, dem es sich mitteilen kann, als äußerste Armut als höchste Bedürftigkeit sich selber erscheinen muss. Nicht darauf, dass im Begriff jedes Anfangs der Begriff eines Mangels liegt.\footnote{Merkwürdig ist in dieser Beziehung gewiss folgende Genealogie von Begriffen in der hebräischen Sprache. \foreignlanguage{hebrew}{\<'bh>} \emph{desideravit, concupivit} \foreignlanguage{hebrew}{\<'b>}, \emph{pater}, (also die väterliche, urhebende Kraft,) \foreignlanguage{hebrew}{\<'bywz>}, \emph{pauper, egenus}. Dass wir in der von Ἀξίερος gegebenen Erklärung vom Eingriff des Hungers unmittelbar zu dem der (schmachtenden) Sehnsucht übergehen, kann dem nicht auffallen, der weiß, dass unser jetzt edleres deutsches Schmachten ursprünglich (wie noch im Niederdeutschen und in einigen Zusammensetzungen) mit Hungern ganz gleichbedeutend war, und Schmacht (ein altes Wort) Hunger ist. S. Adelung.} Wir eilen an etwas Bestimmtes zu erinnern, an jene Platonische Penia, die mit dem Überfluss sich vermählend Mutter des Eros wird. Zwar nach Griechen-Art, welche die ältesten Götter im Reiche des Zeus wiedergeboren werden lässt, erscheint diese Penia beim Gastmahl der anderen Götter. Aber es lässt sich nicht zweifeln, dass Plato hier wie anderwärts nur eine schon vorhandene Fabel frei behandelt, und der erste Stoff seiner Erzählung ein Bruchstück ist jener uralten Lehre, nach welcher Eros der erste der Götter aus dem Weltei hervorgeht, vor ihm aber nur die das Ei gebärende Nacht ist. Denn die Nacht sei das Älteste in der ganzen Natur der Dinge, war Lehre aller Völker, die die Zeiten nach Nächten zählen,\footnote{S. \emph{Grotius de ver. rel. chr. 50. 1. §. 16. not. 15.} Solche Völker waren außer den Morgenländern die alten Deutschen, die gallischen und die slavischen Völkerschaften. Von den Athenern s. \emph{Aul. Gell. 3. 2.}} obwohl es Entstellung ist, wenn man dies erste Wesen zugleich als das oberste betrachtet. Aber was ist das Wesen der Nacht, wenn nicht Mangel, Bedürftigkeit und Sehnsucht? denn diese Nacht ist nicht Finsternis, nicht das dem Licht feindliche, sondern das des Lichts harrende Wesen, sie ist die sehnsüchtige, zu empfangen begierige Nacht. Ein anderes Bild jener ersten Natur, deren ganzes Wesen Begehren und Sucht ist, schien das verzehrende Feuer, das, selbst gewissermaßen Nichts, nur ein alles in sich ziehender Hunger nach Wesen ist. Daher der uralte Lehrsatz: Feuer sei das Innerste, also auch das Älteste, durch Dämpfung des Feuers habe sich erst alles zur Welt angelassen. Daher, dass auch Hestia als das älteste der Wesen verehrt worden und die Begriffe der Ceres und Proserpina, der ältesten Gottheiten, mit dem der Hestia vermengt worden.\footnote{\emph{Pausan. Arcad. 8. 9. p. 216.}\\\hspace*{5mm}Μαντινεῦσι δέ ἐστι --- καὶ Δήμητρος καὶ Κόρης ἱερόν πῦρ δε ἐνταῦθα καίεουσι, ποιούμενοι φροντίδα μὴ λαθῃ σφίσιν ἀποσβεσθέν. Pindar \emph{Nem. 11. 7.} nennt die Hestia πρῶταν Θεῶν, doch nur wie es scheint in Bezug auf die Trankopfer, nach dem vom Schol. angef. Sophokleischen Bruchstück ὦ πρῶρα (πρῶτα) λοιβῆς Ἑστία, womit \emph{Cic. de n. D. 2, 27.} zu vergleichen ist und \emph{Schol. Aristoph. Vesp. 842}. "`Ἐν ταῖς σπονδαῖς ἀφ᾽ Ἑστίας ἄρχονται."' Aber eben dieses, dass ihr in den Prytaneen und auch sonst die Trankopfer zuerst ausgegossen worden, deutet wie die so allgemeine Redensart ἀφ᾽ Ἑστίας (vom ersten Anfang) dahin, dass ihr Begriff mit dem der ältesten Natur vermischt war.} Aber wie schon die Weiblichkeit dieses vielnamigen Wesens, wie dunkler oder deutlicher alle Namen dieser ersten Natur auf die Begriffe der Sehnsucht und des schmachtenden Verlangens hindeuten; so zumal gehet das Wesen der Ceres, für welche der alte Geschichtsschreiber die erste samothrakische Gottheit erklärt, ganz auf in Sucht. Ich bin Deo, antwortet sie, zuerst sich kundgebend, den Töchtern des Celeus,\footnote{\emph{Hymn. in Cer. v. 122.}, wo Wolf mit sicherem Gefühle jetzt Δηὼ wieder hergestellt hat. Kein erfreulicher Name wie der von \emph{Ruhnkenius} vorgeschlagne Δωρὶς (die Geberin) oder in demselben Sinn das von einigen beibehaltne Δὼς kann dort stehen, so wenig als ein bekannter oder völlig erdichteter. Δηὼ war der geheime Name der Ceres, der in Demeter verborgen war. Dass \emph{Deo} für \emph{Devo} ist, wie \emph{Dia} für \emph{Diva}, kann mit Sicherheit angenommen werden.} d. h. die Sehnsuchtkranke, die Schmachtende, eine Bedeutung, die der Zusammenhang fordern würde, wenn sie auch nicht aus Sprachforschung sich rechtfertigen ließe.\footnote{Von \foreignlanguage{hebrew}{\<rwh>} \emph{longuit}, woher \foreignlanguage{hebrew}{\<r:Ot>}(dass der zischenden Aussprache des \foreignlanguage{hebrew}{\<t>} \emph{final}. zufolge mit Δὼς ganz gleichlautet, wenn diese Form nur sonst beglaubigt wäre) \emph{languor, præsertim muliebris} und \foreignlanguage{hebrew}{\<rwy>} \emph{languor ex morbo}. Ganz entsprechend unserem deutschen Sucht, wovon \emph{Wachter Gloss. germ. "`Sucht a. morbus v. c. Mondsucht, Fallsucht. b. affectus gravior totum hominem instar morbi occupans. Tales sunt omnes cupiditates.}"' πόθῳ μινυθοῦσα heißt die der Tochter beraubte Ceres \emph{Hymn. v. 305.}, die von Sehnsucht schmachtende, denn schmachten ist \emph{consumi, tabescere, sive inedia, sive situ, sive desiderio, Wachter. Gloss. h. v.} Die Etymologie von \foreignlanguage{hebrew}{\<.twb>}, \emph{bonus}, die \emph{Ignarra ad hymn. 122.} versucht, entbehrt nach der früheren Bemerkung (35) aller Wahrscheinlichkeit.} Wie Isis im Suchen des verlorenen Gottes, wird Ceres im Suchen der verlorenen Tochter ganz die Suchende. Doch liegt der erste Grund des Begriffes tiefer. Alles Unterste, unter dem nichts mehr ist, kann nur Sucht sein, Wesen, das nicht sowohl ist, als nur trachtet zu sein. Darum ist nach ägyptischer Ansicht Ceres Herrscherin der Toten,\footnote{\emph{Herodot. 2. 123.}} deren Zustand allgemein als ein Zustand von Unvermögenheit und kraftlosem Streben nach Wirklichkeit gedacht wird. Die Unterwelt selbst heißt der geizige, der habsüchtige Dis oder Amenthes. Von Alters her wurden die Abgeschiedene von den Athenern Demetrische\footnote{Τοὺς νεκροὺς Ἀθηναῖοι Δημητρείους ἐκάλουν τὸ παλαιόν. \emph{Plut. de fac. in o. l. Opp. 4. p. 546.}} genannt, weil man sich die vom Leib und der äußeren Welt Getrennte in einen Zustand lauterer Sucht versetzt dachte, aus demselben Grund also, warum die Manen in hebräischer Sprache die sich sehnenden, die verlangenden hießen.\footnote{Nämlich \foreignlanguage{hebrew}{\<'owbOt>}; eine Bedeutung, die der Genealogie Anm. 32. noch beigefügt werden kann.} Damit aber nicht jemand die Worte des weniger tiefsinnigen als witzigen römischen Dichters auch hieher anwende: Nimmer ja gehen Hunger zusammen und Ceres,\footnote{\emph{--- --- --- neque enim Cereremque Famemque Fata coire sinunt. ---} \emph{Ovid.} Met. 8. 19.} genügt zu erinnern, dass wir nicht bloß von einer fruchtbringenden, sondern auch von einer Ceres-Erinnis wissen, und wie die Erinnien überhaupt zu den älteren Gottheiten gehören,\footnote{\emph{Aeschyl. Eum. 145.} γραῖαι δαίμονες \emph{oppos.} τῷ νέῳ θεῷ (dem Apollo) \emph{ib.} und τοῖς νεωτέροις θεοῖς \emph{v. 157.}} so ist eben die fruchtbare Ceres die ältere; denn der gestillten Sucht muss die brennende vorausgehen, überschwenglicher Fülle der Fruchtbarkeit die größte Empfänglichkeit also verzehrender Hunger. Ihre volle Bedeutung erhält dadurch erst die Strafe des Erisichthon, den die zürnende Ceres mit unersättlichem Heißhunger\footnote{Βουβρῶστις \emph{Callim. Hymn. in Cer. v. 103.} vergl. \emph{Iliad. 24, 532.}, wo \emph{Heyne (8. 707.) "`Famem suum fanum habuisse memini lectum."'} Dass der Name Erysichthon selbst bedeutend ist und vielleicht an die gleiche Wurzel mit Axieros erinnert, wollen wir nicht einmal behaupten.} heimsucht. Denn es ist auch sonst dem tiefer Forschenden nicht fremd, dass die Götter durch Verstoßung in eben den Zustand strafen, der durch ihre Gunst überwunden worden. Darum leiden die Uneingeweihten in der Unterwelt die besondere Strafe, dass sie ein unfüllbares Gefäß rastlos zu füllen sich bestreben.\footnote{\emph{Zenob. Cent. 2. Prov. 6.} ἄπληστος πίθος λέγεται οὗτος ἐν ἄδου εἶναι οὐδέποτε πληροὐμενος, πάσχουσι δὲ περὶ αὐτὸν αἱ τῶν ἀμυήτων ψυχαί. Die Töchter des Danaus sollen die Thesmophorien aus Ägypten gebracht und darinn die Pelasgischen Weiber unterrichtet haben, \emph{Herodot. 2. 171.}} Diese Nachweisungen könnten hinlänglich scheinen zur Begründung der gegebenen Erklärung. Doch glauben wir, sie der Gewissheit näher bringen zu können. Es sind uns verschiedene Bruchstücke phönikischer Kosmogonien erhalten. Eine derselben setzt über alle Götter die Zeit, die, weil das gemeinschaftlich Befassende und gleichsam Tragende aller Zahlen, selber nicht zählt, noch für eine Zahl gilt; ihr zunächst aber, also als erste Zahl, nennt sie die schmachtende Sehnsucht.\footnote{\emph{Excerpt. ex Damasc. de princ. in Wolfii anecd. græc. T. 3. p. 259.} Σιδώνιοι δὲ κατὰ τὸν αὐτὸν συγγραφέα (Εύδημον) πρὸ πάντων χρόνων ὑποτίθενται, καὶ ΠΟΘΟΝ καὶ Ὁμίχλην. Die Zeit hat hier offenbar dieselbe Bedeutung, wie \emph{Zeruané akhereré}, die Zeit ohne Grenzen, im Parsischen System. Weil die Götter in einer Folge hervortraten, sind sie selber nur Kinder der allgewaltigen Zeit. Nach einem merkwürdigen Bruchstück ebenfalls bei \emph{Damasc. l. c.} wurde diese Zeit ohne Grenzen als das an sich Gleichgültige (Indifferente) betrachtet, das eben darum Alles ist; ob wohl als solches nur mit dem Verstand, nur im Denken zu fassen (dies ist der Sinn des τὸ νοητὸν ἅπαν καὶ τὸ ἡνώμενον, welches in der Folge durch ἡ ἀδιάκριτος Φύσις vollkommen erklärt wird). Aber diese selbe Zeit ist in ihrem Wirken das Setzende aller Verschiedenheit, oder, wie es in einer persischen Urschrift ausgedrückt wird: "`der wahre Schöpfer ist die Zeit, die keine Schranken kennt, nichts über sich hat, keine Wurzel, ewig gewesen ist und ewig sein wird."' S. Zend-Avesta von Kleuker T. 3. S. 55. Anm. In unsrer Sprache also würden wir sagen: die Zeit ohne Grenze ist das, in welchem nach alter Parsen-Lehre die Einheit und die Verschiedenheit selbst als Eins gesetzt sind. Darnach muss erklärt werden, wenn das Hervortreten der Verschiedenheit in jener Stelle als eine διάκρισις erklärt wird, "`ἐξ οὗ (τοῠ ἡνωμένεου) διακριθῆναι (Φασὶ) καὶ θεὸν ἀγαθὸν καὶ δαίμονα κακὸν ἢ Φῶς καὶ σκότος πρὸ τούτων (\emph{scil.} δαιμόνων ὄντα)."' Dass diese Zeit ohne Grenzen kein \emph{summus Deus} ist, wird jedermann, der den Begriff versteht, mit Tychsen, \emph{Comment. Soc. Gott. Vol. 11. p. 130.}) gegen Anquetil und Kleuker behaupten. Selbst ein \emph{principium superias} kann sie nicht heißen, denn sie geht durch alles hindurch. Aber die bloße Ewigkeit, was man nach jetzigen Schulbegriffen so nennt, ist sie doch auch nicht, so wenig als der Satz: "`Ormusd und Ahriman, beide gab die grenzenlose Zeit,"' nur so viel heißen kann: "`Beide sind oder waren von aller Ewigkeit."'} Ein anderes Bruchstück phönikischer Kosmogonie, dem das Zeichen hoher Altertümlichkeit an der Stirne geschrieben steht, drückt sich so aus: Zuerst war der Hauch einer finstern Luft und ein trübes Chaos, dies alles für sich grenzenlos. Als aber der Geist von der Liebe gegen die eigenen Anfänge entbrannte und eine Zusammenziehung entstand, wurde dieses Band Sehnsucht genannt, und dies war der Beginn der Erschaffung aller Dinge.\footnote{\emph{Euseb. Præp. ev. 50. 2. c. 10.} wie Σύγκρασις durch Mischung übersetzt, erweckt leicht einen falschen Begriff. Ich übersetzte es: Zusammenziehung in dem Sinn, wie zwei Vokalen zusammengezogen werden. Auch Verschmelzung wäre gut; das Wort bedeutet überhaupt eine Verbindung, in der das eine durch das andere gemäßigt wird, \emph{temperamentum}. Ob Πόθος für Ἔρως gehalten werden könne, s. Anm. 47. Im Phönikischen war es sicher kein von \foreignlanguage{hebrew}{\<'hb>}, das nur lieben bedeutet, sondern ein von \foreignlanguage{hebrew}{\<'AwAh>} oder \foreignlanguage{hebrew}{\<'AbAh>} abgeleitetes Wort, das hier durch Πόθος ausgedrückt wird. Vgl. über die eigentliche Bedeutung dieses Worts Anm. 36.} Hier wird der Anfang in ein Entbrennen gegen sich selbst, ein sich selber Suchen gesetzt, das hieraus entstehende Band ist wieder, nur die gleichsam verkörperte, Sehnsucht und der erste Anlass zu Erschaffung aller Dinge. Einheimisch in phönikischen Kosmogonien war also die Vorstellung der Sehnsucht als Anfangs, als ersten Grundes zur Schöpfung. Aber war sie darum auch samothrakisch? Hierauf antwortet eine Stelle des Plinius, der unter den Werken des Skopas die Venus, den Pothos, d. h. die Sehnsucht und den Phaëton nennt, Gottheiten (setzt er hinzu), die in Samothraki mit den heiligsten Gebräuchen verehrt werden.\footnote{\emph{Is fecit Venerem et POTHON et Phaëthontem, qui Samothraki sanctissimis cærimoniis coluntur. H. N. 50. 36. c. 4 p. 727.}} Gewiss also ist, dass unter den samothrakischen Gottheiten eine war, mit der der Begriff: Sehnsucht, verbunden wurde. Wir kennen mit ziemlicher Zuverlässigkeit alle samothrakischen Gottheiten, aber es ist keine, welcher schmachtende Sehnsucht so eigen, so ganz angemessen wäre, als der, welche der alte Geschichtsschreiber für Demeter erklärt, der also, welche Axieros genannt wurde.\footnote{Weil Varro die dem Kabirensystem zu Grunde liegende Zweiheit als \emph{Cœlum et Terra} ansieht, glaubt \emph{Sainte-Croix l. c. p. 29.} aus Phaethon den Himmel machen zu können, oder (was doch so einerlei nicht ist) \emph{la lumière, qui l'éclaire}, dieses sei dann (warum?) Axieros, Venus sei Axiokersa und Pothon (Pothos), oder Cupido, der junge Cadmillus. Vorsichtiger drückt sich Kreuzer aus 2. 303. "`Auf jeden Fall war wohl Phaethon kein andrer, als der Lichtbringer Axieros (Phthas, Hephästos), und Pothos war der dienende Dämon Eros, wie ihn auch Platon kennt."' Gesetzt selbst, der Pothos wäre Eros, so würde er, weil Ἔρως oder nach der alten Form Ἔρος doch am Ende von derselben Herkunft mit Ἀξίερος sein möchte, immer natürlicher in diesem Namen als in Kadmilos gesucht. Insoweit ist die Bedeutung von Ἔρως, \emph{Cupido}, nur eine Bestätigung der von Ἀξίερος gegebenen Erklärung; die Begriffe des Sehnens, Verlangens, Begehrens sind die einzigen, welche bei übrigens so verschiedenen Gottheiten den Gleichlaut der Namen erklären können. Aber dem Sprachgebrauch nach ist Πόθος sehr bestimmt von Ἔρως unterschieden. Den eigentlichen Begriff des ersten zeigt die obige Anführung aus \emph{Hymn. in Cer.} und eine größere Zahl von Nachweisungen bei \emph{Kreuzer. ad Plotin. de pulcrit. p. 213.} Πόθος ist Sehnsucht nach einem verlorenen oder doch jetzt abwesenden Gut. Wie Πόθος sich auf Vergangenheit bezieht, so Ἵμερος auf das Gegenwärtige, Anwesende (s. \emph{Plat. in Cratyl. p. 304. Bip.}); Ἔρως ist das erste Entbrennen, die Begierde, die dem Besitz vorausgeht, also nach dem noch Zukünftigen strebt (vergl. den Sprachgebrauch in \emph{Plat. Sympos.} ὁρᾶτε ἐι τούτου ἐρᾶτε \emph{p. 208. Bip.} u. a.), darum passt der Begriff Πόθος unter den samothrakischen Gottheiten nur auf Ceres, denn sie allein schmachtet oder sehnt sich nach einem Verlorenen, es sei nun die Tochter oder vielmehr der Gott, den sie wie Isis sucht. Jedes Sehnen irgendeiner Natur, auch dieses erste und uranfängliche deutet nach alter Lehre auf ein vormaliges Einsgewesensein mit dem, wonach sie sich sehnet (vgl. die auch von Kreuzer angef. Worte des Aristophanes in Plat. \emph{Sympos. p. 204.}). Auch jene erste Natur ist nur durch eine vorhergegangene Scheidung in jenen Zustand der Einsamkeit, also des Mangels, der Bedürftigkeit gesetzt worden, in dem sie als Sehnsucht erscheint. Aber nicht weniger im Kunstbegriffe war Πόθος von Ἔρως unterschieden. Wenn auch nach Kreuzers Bemerkung (\emph{ad Plot. p. 214.}) späterer Sprachgebrauch den Unterschied weniger genau beobachtet haben sollte, so hatte der samothrakische Pothos des Skopas seinen Namen vom Ursprung her, und damals gewiss war mit Pothos ein ganz andrer Kunstbegriff verbunden, als mit Ἔρως. Beweis die Erzählung des Pausanias, \emph{Attic. 100. 43. p. 105.} In Megara sah man von der Hand desselben Skopas drei Werke, Eros, Himeros und Pothos, von denen gesagt wird: ἔιδη διάφορά ἐστι κατὰ τἀυτὰ τοῖς ὀνόμασι καὶ τὰ ἔργα σφίσιν, eine Brachyologie, die nur so aufzulösen ist: "`Es sind Gestalten verschieden (gebildet) nach den einem jeden zukommenden Werken, die sich ihren Namen gleich und auch so (verschieden) wie diese verhalten."' Die in den drei Gestalten gedachte Fortschreitung konnte keine andre sein, als die oben angegeben worden. Beweis genug, dass die drei keine bloßen Eroten oder \emph{Cupidines} waren, die der tändelnde Geschmack auch da sieht, wo sie nicht sind. Die dritte Gestalt, die nach dem verlorenen Gegenstand schmachtende Sehnsucht, kann man sich auch hier kaum anders, als weiblich denken. Dem sei, wie ihm wolle, verschieden waren übrigens die beiden Reihen. Der von Plinius erwähnte Pothos bildete mit Phaethon und Venus grade eben so eine plastische Trilogie, wie der von Pausanios mit Himeros und Eros ein zusammengehöriges Ganze ausmachte. Der Pothos bei Plinius wird bestimmt durch die Vorstellungen der Venus und des Phaethon, samothrakischen Gottheiten, mit denen er ein Ganzes bildet; der bei Pausanias durch Himeros und Eros, mit denen er Einen Kunstkreis erfüllt. Die Trilogie bei Pausanias scheint, ganz aus dem Geiste des Meisters gekommen, ein künstlerisch-freies Spiel gewesen zu sein, ob ihn gleich vielleicht nicht der spitzfindige Gedanke, die Abstufungen einer bloßen Empfindung darzustellen, sondern etwas Poetischeres und Symbolischeres begeisterte. In der andern hatte er sich freiwillig an etwas Gegebenes gebunden, er wollte nicht eine Venus überhaupt, sondern eine Venus mit der bestimmten Vorstellung der samothrakischen, so nicht einen Pothos überhaupt, sondern die Gottheit bilden, welche in Samothraki als Sehnsucht verehrt wurde. So weit also, aber gewiss nicht weiter, waren die beiden Pothos verschieden.} Hiedurch glauben wir die gegebne Erklärung zu dem in solchen Untersuchungen möglichen Grad der Gewissheit gebracht. Was die folgenden Namen der zweiten und dritten Persönlichkeiten betrifft, Axiokersa und Axiokersos; so möchte man sich zunächst darüber wundern, dass keiner der bisherigen Forscher in ihnen die Spur der uralten Wurzel des Ceres-Namens erblick hat, da doch in diesem Zusammenhang alles aus Cerealischen Dienst und Lehre hindeutet. Wirklich ist Kersa nur nach einer andern Mundart dasselbe, was Ceres (in der alten Aussprache Keres).\footnote{Ceres nämlich ist das hebr. \foreignlanguage{hebrew}{\<.hr/s>}, \emph{Kersa} nur das chald. \foreignlanguage{hebrew}{\<.hr/s'>}. Dass Ceres nichts anders als das hebr. \emph{Cheres} ist, lässt sich kaum bezweifeln, wenn man auch nur die gewöhnliche Bedeutung dieses Worts und der davon abstammenden kennt, \foreignlanguage{hebrew}{\<.hAra+s>} \emph{aravit}, \foreignlanguage{hebrew}{\<.hore+s>} \emph{sata}, \emph{Es. 17, 9.} arab. $\symbolAAG$, \emph{cultura fundi, aratio, satio, ager, satum.} Wer die sonst versuchten Ableitungen kennen lernen will, findet sie in \emph{Villoison Eclaircissements} zu \emph{Sainte-Croix p. 523.}, bei \emph{Ignarra ad hymn. Cer. 122.}, auch bei Kreuzer 4. 338., der eine morgenländische Wurzel erwartet für Ceres, so wie für das alte nach Varro für \emph{creo} gebrauchte \emph{cereo}, wovon \emph{Cerus manus}, das Festus durch \emph{creator bonus} erklärt.} Und da, nach einmal erwiesener Bedeutung der Axieros, daran, dass Axiokersa die Persephone ist, nicht zu zweifeln steht, so dient dieser Name nur als neuer Beleg des auch sonsther bekannten, dass Proserpina nur Ceres, die Tochter nur die Mutter ist in einer andern Gestalt, und auch wohl ihre Namen, wie oft ihre Bilder verwechselt worden.\footnote{\emph{Spanhem. ad Call. hymn. in Cer. 113.} Kreuzer 4. 10. 236. 253. Bei Euripides, \emph{Phœn. 689.}, heißen Ceres und Proserpine die διώνυμοι θεοί.} Zauber aber oder Zauberin (denn dies bedeuten die Wörter\footnote{Diese den aramäischen Mundarten ganz gewöhnliche Bedeutung des Worts \foreignlanguage{hebrew}{\<hr/s>} wurde bei den bisherigen Anwendungen auf Erklärung des Ceresnamens übersehen, vielleicht weil sie im Hebräischen seltener ist, denn dass sie auch dieser Mundart nicht fehlt, zeigt T. 7, 3. und der Name Thal der Charasim \emph{Neh. 11, 35. 1 Par. 4, 14.}, wo beigesetzt wird, "`denn sie waren Charasim, d. h. Zauberer (s. \emph{Sim. Onom. p. 166.}), etwa wie die wegen Wahrsagekunst berühmten Einwohner von Telmassus und die wegen Zauberei berüchtigten Männer und Weiber Thessaliens. Aus \emph{Esr. 2, 59. Neh. 7, 61.} lernen wir den Namen eines Orts \foreignlanguage{hebrew}{\<.hl .hr/s'>} kennen in Chaldäa, wo auch der mit Axieros gleichlautende Name vorkommt (\emph{Dan. 9, 1.}). Gewöhnlich erklärt man \emph{Tumulus aratioris}, sehr flach; ich zweifle nicht, dass \foreignlanguage{hebrew}{\<.hr/s'>} hier Eigenname und zwar einer Gottheit ist. Wie der Begriff des Ackerbaus und des Zaubers sowohl in jenem Wort als im Begriff der Ceres zusammenhänge, leidet noch eine tiefere Erforschung.}), kann sowohl Demeter als Persephone genannt werden. Denn als der Hunger nach Wesen, den wir noch als das Innerste der ganzen Natur erkennen, ist Ceres die bewegende Kraft, durch deren unablässiges Anziehen aus der ersten Unentschiedenheit alles wie durch Zauber zur Wirklichkeit oder Gestaltung gebracht wird. Aber die ursprünglich gestaltlose, darum in ihrem Tempel zu Rom als Vesta bildlos und in der reinen Flamme verehrte, Gottheit\footnote{\emph{Ovid. Fast. 6. 295. ss.} Auch in einem Tempel des Peloponnes, \emph{Paus. Cor. c. 35. ir.} Diess hinderte nicht Bilder der Vesta außer ihrem Tempel.} nimmt in Persephone Gestalt an, und diese wird erst eigentlich der lebendige Zauber, gleichsam das Mittel und Gebild, an welches der unauflösliche Zauber geknüpft ist. Doch über diese Bedeutung kurz zu sein, erlauben uns die gelehrten Zusammenstellungen Kreuzers, die schwerlich einen andern obersten Verbindungsbegriff zulassen, als den der Zauberin, in dem ohnehin auch der der Künstlerin gedacht wird. Zauberin ist Persephone als erster Anfang zum künftigen leiblichen Dasein, als die, welche dies Kleid der Sterblichkeit webt und das Blendwerk der Sinne hervorbringt, überhaupt aber als erstes Glied der vom Tiefsten bis ins Höchste gehenden, Anfang und Ende verbindenden Kette.\footnote{Kreuzer 3. 455. ff. 533. ff. 4. 247. u. a.} Persephone heißt auch Maja, ein Name, der an Magia vielleicht mehr als nur erinnert.\footnote{Die ursprüngliche Bedeutung des Worts \emph{Magia, Magus} ist verloren. Die persische Sprache selbst hat kein Wort, von dem ihr $\symbolAAH$ oder $\symbolAAI$ abstammen könnte, daher es Hyde für radikal erklärt. Ebenso gut könnte aber geschlossen werden, es sei ein der persischen Sprache selbst ursprünglich fremdes Wort. Die arabische mag ihr $\symbolAAJ$, \emph{magum effecit}, genommen haben, wo sie will, so zeigt es, wie leicht in morgenländischen Sprachen bei fremden Wörtern die Wurzelbuchstaben sich ändern. Die indische Maja, welche durchaus nichts anderes ist, als Zauberin (\emph{præstigiatrix}) und zwar in demselben Sinn, wie Persephone, wird im Persischen $\symbolAAK$ geschrieben. S. \emph{Langlès Notes} zu \emph{Recherches Asiat. T. 1. p. 219}. Hierinn also könnte die Hinweisung auf die wahre Bedeutung des Worts liegen.} Auch Artemis sei was Persephone, soll schon durch Aeschylus verlautet haben,\footnote{Kreuzer 4. 13.} und auch Artemis heißt die Zauberin nach der natürlichen Ableitung.\footnote{Der Beweis hievon wird für eine andre Gelegenheit vorbehalten.} Aber überhaupt allen weiblichen Gottheiten liegt der Begriff des Zaubers zu Grunde, und wie die Götterlehre der alten Deutschen, innerlicher, als geahndet wird, verwandt jener samothrakischen, dem Othin die Freya zugesellt und beiden mächtige Zauberkräfte zuschreibt,\footnote{Arnkiel's kimbrische Heyden-Religion 1. S. 62. "`Alle Zauberei hat in der nordischen Welt von ihm (Othin) ihren Ursprung u. s. w."' Aus \emph{Snoro Sturles. Chron. Norwag.} Ebendas. S. 61. heißt es: "`Wenn seine Völker in Nöten und Gefährlichkeiten waren zu Wasser oder zu Lande, riefen sie seinen Namen an und vermeinten Hülfe von ihm zu haben; deßwegen war er all' ihr Trost."' Wegen der Freja, \emph{frie, fri} bedarf es nicht einmal der Erinnerung an die persischen Peris ($\symbolAAL$) oder Feen.} so sind Axiokersa und Axiokersos durch den gemeinschaftlichen Begriff des Zaubers vereint. Denn diese dritte Gestalt ist wirklich kein anderer, als der den Ägyptern Osiris, den Griechen Dionysos, den Deutschen Othin war.\footnote{Die Einerleiheit von Osiris und Dionysos weiß jeder aus Herodot und Plutarch. Die Ähnlichkeit der Züge in den Erzählungen von Osiris und Othin muss jedem auffallen, der auch nur den Anfang von \emph{Plut. de Is. et Os. c. 13.} liest: "`Osiris, wird erzählt, habe gleich zuerst die Ägypter von der tierischen Lebensweise befreit, indem er ihnen die Früchte gezeigt, Gesetze gegeben und die Götter ehren gelehrt. Darauf habe er das ganze Land, dessen Sitten zu mildern, durchzogen, am wenigsten der Waffen sich bedienend, sondern die Meisten mit Überredung, Wort, allerlei Art Gesang und Tonkunst gesänftigt."' Von Othin sagt Arnkiel S. 63. 62. "`Diess alles hat er ausgerichtet durch Reim und Gedichte, welche Galdrer oder Schaldrer heißen. Daher die Asiatischen Schaldmeister und Runmeister genannt worden. Was er redete, brachte er reinweiß für, nach der Tichter Kunst, also dass man ihm mit Lust zuhöre."' Zur Vermeidung jedes Missverstands bemerke ich, dass Othin mit Wodan nicht Einer ist. Diesen bezeichnet der über die Urzeiten unseres Volks glaubwürdigste Schriftsteller Tacitus mit Recht durch den Mercur.} Zwar der griechische Geschichtsschreiber erklärt Ariokersos als Hades und alle Ausleger verstehen dies eigentlich, von Pluto nämlich oder dem stygischen Jupiter. Aber Hades und Dionysos sind dieselben, lehrte schon Heraklit\footnote{Ἅδης καὶ Διόνυσος ὁ αὐτός. \emph{Plut. de I. et O. c. 28. p. 333.}} und Osiris-Dionysos ist König über die Abgeschiedenen,\footnote{\emph{Ib. c. 79.} ἄρχει (\emph{Herodot. 2. 123.} ἀρχηγετέυει) καὶ βασιλεύει τῶν τεθνηκότων.} wie unser deutscher Othin wohltätiger Gott, erster Überbringer der fröhlichen Botschaft, zugleich Herr im Todtenreich ist. Diese Lehre, der freundliche Gott Dionysos sei der Hades, war unstreitig die beseligende Überzeugung, welche die Geheimlehren mitteilten. Nicht zu dem strengen unterirdischen Zeus abwärts, sondern zu dem milden Gott Osiris aufwärts gehen die Seelen, dies war der verborgenste Sinn der Lehre, dass Dionysos der Hades sei. Deutlich erhellt dies aus einer Stelle des Plutarch,\footnote{"`Auch das, was die jetzigen Priester mit heiliger Scheu und Umhüllung und Vorsicht äußern, dieser Gott sei Herrscher der Toten und eben der, der bei den Griechen Hades und Pluton genannt wird, stört, weil unvollkommen gewusst, die Mehresten, welche meinen, in und unter der Erde wohne wahrhaft jener heilige Osiris. Aber dieser ist weit von der Erde entfernt, unbefleckt und rein von jeder des Untergangs und Todes empfänglichen Natur."' Ebendas.} wie aus jenem, selbst auf römischen Grabmälern, so häufigen Nachruf: Lebe selig mit dem Osiris.\footnote{Εὐψυχεῖ μετὰ τοῦ Ὀσίριδος. S. \emph{Zoëga de obel. p. 305.} Dagegen: Fahr' zu Oden! ist eine nordische Verwünschung. Arnkiel S. 66.} In diesem Zusammenhang war Persephone nicht des Hades, sondern als Kore und Libera des Dionysos Gattin.\footnote{Kreuzer 3. 396.} Dabei blieb aber im öffentlichen Gebrauch der Hades wenigstens im Besitz des Namens, und so hieß nun Dionysos selbst Hades. Dionysos also oder Osiris ist Axiokersos, wie ja Axiokersa-Persephone auch Isis ist.\footnote{\emph{Plut. de I. et O. c. 27. p. 333.}} Was aber der Name aufs genaueste ausdrückt, ist schwer zu sagen, da wir ihn nicht in seiner ursprünglichen Gestalt kennen. Heißt diese Persönlichkeit Axiokersos bloß als Gemahl der Axiokersa? Oder ist er Zauberer in einem höheren Sinn, als der, welcher jenen Zauber der Persephone überwindet, ihre Strenge mildert, jenes Urfeuer (denn auch sie ist Feuer) dämpft und beschwört?\footnote{Es versteht sich, dass das Letzte unsere Meinung ist. Axiokersa und Axiokersos erbauen zusammen das Weltall durch einen doppelten Zauber, da der spätere den früheren nicht aufhebt oder vernichtet, sondern überwindet. Dem wäre so, auch wenn der Name bloß den allgemeinen Begriff des Zauberers ausdrückte. Doch ist zu vermuten, nicht \emph{Kersos}, sondern \emph{Kersor} sei das ursprüngliche, wie \emph{Amilcar} im Griechischen, Ἀμίλκας lautet und \emph{Barthélemy, Reflexions sur quelques monumens Phéniciens (Mém. de l'Ac. des Inscr. T. 30. pag. 410.)} bemerkt: \emph{Les Grecs paroissent avoir terminé en os les noms phéniciens, qui terminoient en ορ, par la même raison, que les mots Lacedémoniens terminés en ορ, avoient chez les autres peuples de la Grece une terminaison en ος, Τιμόθεορ, Τιμόθεος, Μιλήσιορ, Μιλήσιος etc.} Der Name Κέρσορ aber, oder Κερσώρ, würde an den Χρυσώρ des Sanchuniathon erinnern, von dem gesagt wird, er sei der Hephästos, \emph{Euseb. Pr. ev. 50. 1. p. 35. C.} Das Letzte nun dürfte nicht irren. Denn die ersten Kabiren alle sind Hephäste (s. §. 12. des Textes und die dazu gehörigen Anm.). Überdies wird hinzugesetzt, λόγους ἀσκῆσαι (τὸν Χρυσὼρ) καὶ ἐπῳδὰς καὶ μαντείας, wodurch er wieder zum Zauberer wird und Eigenschaften erhält, die dem gewöhnlich sogenannten Hephästos nicht zukommen. Dass er dennoch durch Hephästos erklärt wird, zeigt auf die wahre Bedeutung. Er ist der Feuer-Gott, denn auf jeden Fall hat er mit Feuer zu tun. Er heißt Hephästos, wie der ägyptische Phthas, auch bei Eusebius, 3. 2. p. 115. und bei Suidas, T. 3. p. 615. \emph{voc.} Φθᾶς, für Vulcan ausgegeben wird, obgleich derselbe Suidas, \emph{voc.} Ἀφθὰς T. 1. p. 396., richtiger und unstreitig aus irgendeiner alten Quelle sagt: Ἀφθὰς. Ὁ Διόνυσος. τὸ ἀ ἐπιτατικόν. --- --- καὶ παροιμία • Ὁ Ἀφθάς σοι λελάληκεν. ἦν δὲ χρησμολόγος. Auch er (Phthas) ist nur Hephästos inwiefern das männliche, oder eröffnende, aufschließende Feuer. Umso mehr Aufmerksamkeit verdient, auch nach Akerblads Widerspruch, was \emph{Sylvestre de Sacy, Lettre au sujêt de l'Inscription Egyptienne du monument trouvé à Rosette p. 22. ss.}, behauptet, auf der Inschrift werde Hephästos von Phthas unterschieden, dieser Name sei nicht dem Vulcan eigen, sondern Name aller Götter (oder doch gewiss mehrerer); und wenn nach der Bemerkung desselben scharfsinnigen Gelehrten das am Ende nicht zum altägyptischen Wort gehört und der wahre Name, wie in der griechischen Inschrift, ΦΘΑ lautete: so dürfte sich statt der misslungenen ägyptischen Etymologien von Jablonski und La Croze vielleicht eine hebräische anbieten. Nach derselben wäre φθὰς der Eröffner (\foreignlanguage{hebrew}{\<pote.ha>}), eine Bedeutung, die mit allen seinen Eigenschaften übereinstimmen würde. Dieses nun auch darum, weil Zoega u. a. in Axieros diesen vermeinten höchsten Gott des ägyptischen Systems sehen wollten! Dem sei, wie ihm wolle, auf Feuer bezieht sich der Name Χρυσώρ und so hat wohl Bochart \emph{G. S. 50. 2. c. 2.} ganz richtig ins Phönikische zurückübersetzt; nach ihm ist \emph{Chrysor} \foreignlanguage{hebrew}{\<.hr/s 'wr>}. Da aber \foreignlanguage{hebrew}{\<.hr/s>} im Sinn von \emph{fabricare} transitive Bedeutung hat und das eigentliche Wort für Bearbeitung von Metallen ist (\emph{Genes. 4, 22.}), so würde \foreignlanguage{hebrew}{\<.hr/s 'wr>} kaum etwas anderes bedeuten können, als der das Feuer selbst hämmert. Wahrscheinlicher also, dass das Wort in diesem Namen seine andre Bedeutung, des Beschwörens, hat. Aber auch so den Namen recht zu verstehen, würde eine Kenntnis erfordert, der geheimeren, auch den Hebräern bekannten, Feuerlehre. Das Wort \emph{Ur} (wovon unser Ur in Ur-Bild u. ähnl.) ist durchaus verborgeneren Sinns; es ist nicht das äußere Feuer (das \foreignlanguage{hebrew}{\<'e+s>} heißt), sondern das innere, gleichsam was im Feuer das Feuer ist: in solchem Verhältnis stehen \foreignlanguage{hebrew}{\<'wr>} und \foreignlanguage{hebrew}{\<'/s>} zusammen \emph{Es. 50, 2.} Doch kann, das Wort im angegebenen Sinn genommen, \foreignlanguage{hebrew}{\<.hr/s 'wr>} kaum etwas andres heißen, als Feuer-Beschwörer, Besänftigter, \emph{incantator ignis}. Die transitive Bedeutung des Worts in diesem Sinn ist zwar durch keine mir bekannte Stelle erweislich, aber \foreignlanguage{hebrew}{\</s.hr>}, welches nach der in den morgenländischen Sprachen so häufigen \emph{Metathesis} dasselbe mit \foreignlanguage{hebrew}{\<.hr/s>} ist, hat wenigstens im arabischen $\symbolAAM$ transitive Bedeutung im Sinn von \emph{incantare}. So in \emph{Geogr. Nub.} bei \emph{Bochart. Hieroz. 2. 386.} $\symbolAAN\:\symbolAAO\:\symbolAAP$ \emph{incantant animalia noxia}; vgl. die von \emph{Castell. Lex. heptagl. 2. 1508.} angef. Stellen des Koran. Dann wäre ja jener \emph{Chores-Ur, Chrysor} oder \emph{Kersor} auch dem Namen nach gleichbedeutend mit dem \emph{Oser-Es, Osiris}; ein Name, dem man nach so vielen meist auf höchst allgemeine Begriffe hinauslaufenden Erklärungen sehr geneigt sein könnte, für \foreignlanguage{hebrew}{\<'sr '/s>} oder nach der wahrscheinlich älteren Schreibung \foreignlanguage{hebrew}{\<'sr 'y/s>} Feuer-Bändiger, Feuer-Beschwörer zu erklären. Denn die morgenländischen Wörter, die \emph{sign. ligandi}, haben meist auch \emph{sign. incantandi}; wegen \foreignlanguage{hebrew}{\<'sr>}. \emph{Targ. Jon. Deut. 18, 2.}, wo \foreignlanguage{hebrew}{\<.hober .hAbEr>} durch \foreignlanguage{hebrew}{\<m:.hab*:riyN wi'As:riyN hewiyN>} übersetzt wird. Dieser Erklärung des Osiris-Namens kommt die bekannte von Barthelemy sinnreich erklärte phönicisch-griechische Inschrift von Maltha gewissermaßen zu Statten. S. die Abbildung \emph{Pl. 1. p. 424.} in \emph{Mém. de l'Ac. des 1. T. 30.} Dort entspricht \emph{lin. 2.} dem griechischen Διονύσιος das phönic. \foreignlanguage{hebrew}{\<.sbd 'sr>}, Diener Asars, ohngefähr wie auf dem von Akerblad, \emph{Comm. Gott. Vol. 14.}, bekannt gemachten Stein Heliodorus durch Diener der Sonne ausgedrückt ist. Von der andern Seite wird ein Teil der Erklärung durch sie zweifelhaft, da \emph{Osiris} blos als \emph{Oser} genommen und is als griechische Endigung behandelt ist. Eine andre phönikische Inschrift, die des Basreliefs von Carpentres, enthält dreimal den Namen Osiris und zwar jedesmal \foreignlanguage{hebrew}{\<'wsry>} \emph{Oseri, cum Jod quasi gentilitio}, wie in dem hier ganz analogen \foreignlanguage{hebrew}{\<yid*.sokiy>}; dass \foreignlanguage{hebrew}{\<'zsyry>} geschrieben sei, ist bloßes Vorgeben von Hug, über den Mythus der alten Welt S. 62. Anm., die Inschrift und \emph{Barthelemy, Mém. de l'Ac. d. Inscr. 32. p. 728.} weiß nichts davon. Auch diese Inschrift führt daher auf Oser zurück und schneidet nur die Möglichkeit ab, auch etwa \foreignlanguage{hebrew}{\<'Asir>} zu lesen, was einerlei Form wäre mit \emph{Kabir}. Nehmen wir also \emph{Oser, Oseri} als das richtige an, so hindert nichts, dies Wort auch so, schlechthin gesetzt, durch Beschwörer, Bezauberer zu erklären, zumal das hinzudenkliche Hauptwort wieder von derselben Wurzel und der vollständige Ausdruck \foreignlanguage{hebrew}{\<'EsAr 'osr>} sein könnte, \emph{ligans ligationem} (das letzte Wort im Sinn des griech. κατάδεσμος), wie \foreignlanguage{hebrew}{\<.hober .hAbEer>} \emph{Deut. 18, 11.}, welches die samaritanische Übersetzung wirklich durch \foreignlanguage{hebrew}{\<'sr 'sr>} ausdrückt. Und wenn jemand damit das Etruskische \emph{Aesar} in Verbindung setzt, "`\emph{quod AESAR Etrusca lingua Deus vocaretur}"' \emph{Sueton. in Oct. p. 229. Wolf.}, könnte man's aber auch nicht unbedingt tadeln. So möchte man sich denn auch für Ἀξιοκέρσος mit dem einfachen \foreignlanguage{hebrew}{\<.hr/s>} begnügen. Er wäre fast wörtlich jener \emph{Cerus manus} Anm. 31. oder \emph{Creator bonus}; \foreignlanguage{hebrew}{\<.hr/s>} bliebe in seiner gewöhnlichen Bedeutung als \emph{fabricator} (Demiurg), die indes den Begriff von \emph{magus} nicht ausschließt, so wenig \foreignlanguage{hebrew}{\<.hr/sh>} in der Bedeutung von \emph{mago, præstigiatrix} den von \emph{fabricatrix} (\emph{rerum natura, Lactant. Epit. 68.}) ausschließt. Eine Frage ist, wie weit man das Ansehen der beiden Inschriften gelten lässt. Bei der von Maltha würde etwas auch darauf ankommen, ob es eingeborne Tyrier sind, deren Namen ins Griechische, oder geborne Griechen, deren Namen ins Phönikische übersetzt worden. Verschiedene Umstände sprechen für das Erste. Dann wär' es eigentlich mir der griechische Übersetzer, der den Namen \emph{Abdasar} durch Διονύσιος erklärt hätte. Andre Bewandtnis hat es mit der von Carpentres, dort ist \emph{Oseri} als Name des Gottes Osiris nicht zu verkennen, das Basrelief selbst enthält ägyptische Vorstellungen, unter diesen den Osiris. Seine Herkunft, ob aus Ägypten selbst oder aus irgendeiner der phönikischen Niederlassungen, ist unbekannt, wie sein Alter. Aus welcher Zeit es aber sei, beweist es doch mir, dass man damals den Osiris-Namen durch Oseri vollkommen ausdrücken zu können meinte. Wird nun dieser Schreibung Urkundlichkeit zugestanden, so weiß man ebendieselbe auch für die Ableitung von \foreignlanguage{hebrew}{\<'sr>} zu geben, und so wäre \emph{Oser} oder \emph{Oseri} doch nur der kürzere Name; Χρυσώρ und Κερσώρ der vollständigere. Denn was die Wahrscheinlichkeit der zuerst gegebenen Erklärung noch erhält, ist ein anderer aus der Kosmogonie des Phönikier Mochos angeführten Namen, Χουσωρὸς, den entweder dieser selbst oder doch Damascius als den ersten Eröffner, ἀνοιγέα πρῶτον erklärt, \emph{Wolf anecd. gr. 3. 260.} Hier hätten wir also zum drittenmal jenes bedeutende \emph{Or}; der seltsamste Zufall müsste walten, wenn nur zufälliger Weise dieses Wort auch wieder den Feuer-Bändiger bedeutete, von dem chald. \foreignlanguage{hebrew}{\<.hws>}, eigentlich \emph{propitium, clementem esse}, wovon \foreignlanguage{hebrew}{\<.hA'es>}, dessen sich die chaldäische Übersetzung für das hebräische \foreignlanguage{hebrew}{\<m:kab*Eh>} bedient, in der bedeutenden Stelle \emph{Jer. 4, 4.} "`dass nicht mein Zorn entbrenne und kein Besänftiger sei"' \foreignlanguage{hebrew}{\<wl' y:hy .h's>}, und in demselben Sinn \emph{Jer. 7, 20. Es. 1, 30.} Vgl. \emph{Buxtorf. lex. chald. talm. p. 721.} Noch seltsamer wenn sich zu diesem Χρυσώρ, Κερσώρ, Χουσώρ, endlich noch Dionysos selbst, mit der gleichen Bedeutung gesellte! Doch davon jetzt nicht! Kreuzer 4. 75. Anm. hat schon den Χουσωρὸς mit dem Χρυσώρ zusammenzubringen gesucht, so wie mit dem hesiodischen Chrysaon und dem \emph{Adj.} χρυσάορος, das als Beiwort der Ceres \emph{Hymn. v. 4.} auf keinen Fall so schnell als von \emph{Ruhnkenius} u. a. verworfen werden sollte und auch von Wolf beibehalten ist. Weder der Ceres, noch (was häufiger) dem Apollo, \emph{Il. 5, 509. 15, 256.}, der so viel mit Dionysos gemein hat, noch dem Orpheus (bei Pindar in \emph{Villois. Schol. ad 50. Il. prox. cit.}) will es nach der aus griechischer Etymologie genommenen Erklärung "`der mit goldenem Schwert"' recht zusagen. Es ist eines der alten Wörter, die an die Griechen ohne Kenntnis ihres wahren Sinns gekommen waren, und wurde nur in Folge von Überlieferung mit gewißen Gottheiten verbunden. Doch genug der sprachlichen Untersuchungen, um endlich zu fragen, wie denn Dionysos oder Osiris Feuerbeschwörer, Feuerbesänftiger heißen könne, und wie damit der Begriff eines ersten Eröffners zusammenhänge? Statt jeder tiefer eingehenden philosophischen Erläuterung stehe hier der uralte Lehrsatz: "`Κόσμος --- --- πῦρ ἀειζῶον, ἀπτόμενον μέτρα (\emph{Euseb.} μέτρῳ) καὶ ἀποσβεννύμενον μέτρα"' \emph{Heracl. ap. Clem. Alex. Strom. 5. p. 711. ed. Pottos.} "`Die Welt ein ewig lebendes Feuer, das in Pausen (so erkläre ich μέτρα \emph{subirt.} κατὰ) entbrennt und gelöscht wird."' Eine Kraft also, die es entzündet (das ist Ceres, Isis, Persephone oder wie man sonst die erste Natur nenne), eine, die es löscht (vgl. Anm. 66.), besänftiget und dadurch erster Eröffner der Natur wird, sie in mildes Leben und sanfte Leiblichkeit aufschließend, diese ist Osiris oder Dionysos. Τοῦ πυρὸς κατασβεννυμένου κοσμοποιεῖσθαι τὰ πάντα sagten Heraklit und Hippasus (\emph{Plut. de pl. phil. Opp. 4. 355. Euseb. pr. ev. p. 749.}), darum war auch Dionysos (Anm. 80.) Demiurg.} Diess könnte auf jeden Fall nur durch Untersuchungen ausgemacht werden, die sich für diesen Vortrag nicht eignen. Aber welchen bestimmteren Sinn man außer dem allgemeinen dem Namen gebe, ein zauberischer Gott ist Dionysos, sei es dass man an die Schreckbilder denke, mit denen er die tyrrhenischen Schiffer strafte, oder an sein Amt als Eröffner der Natur, als alles mildernder Gott,\footnote{Εὐεργέτης, ἀγαθοποιός heißt Osiris bei \emph{Plut. p. 317.} Derselbe \emph{c. 42} sagt, Osiris bedeute zwar sehr vieles, aber doch vorzüglich κράτος εὐεργετοῦν (nach Markland's Verbess.) καὶ ἀγαθοποιόν.} als der feuchte, der dem trocknenden Feuer wehrt.\footnote{\emph{Id. c. 33.} Sanchuniathon bei \emph{Euseb. p. 35.} sagt: καλεῖσθαι αὐτὸν καὶ Διαμίλιον, welches Kreuzer 4. 75. durch \emph{Jovem penetralem} erklärt. Hellanicus wollte den Namen als Ὕσιρις gehört haben; von der Befeuchtung, und Hyes, sei Dionysos genannt worden, ὡς κύριος τῆς ὑγρᾶς Φύσεως, sagt \emph{Plut. c. 34.} Eben dieses Amt des Feuer-Löschenden übt er auch im andern Leben. Daher jener fromme Wunsch auf Grabmälern: Osiris gebe dir das kühle Wasser! Vgl. \emph{Luc. 16, 24.} Auch dort ist er der beseligende Gott, weil durch ihn das Feuer jener unauslöschlichen Sucht gestillt wird, mit dem die Seelen der Ungeweihten erfüllt sind.} So bilden also die drei ersten samothrakischen Götter dieselbe Folge und Verkettung, in der wir auch sonst überall Demeter, Persephone und Dionysos finden. Es folgt die vierte Gestalt, Kasmilos genannt, gewöhnlicher Kadmilos, auch Camillus. Über diesen Namen nun sind alle Erklärer so weit einstimmig, dass er einen dienenden Gott bedeute; wie auch schon aus der Verrichtung des etruskisch-römischen Camillus erhellen würde. Aber welches Gottes oder welcher Götter Diener? Die nicht minder einstimmige Meinung ist, eben jenen ihm vorangehenden Göttern sei er als Diener beigesellt und zwar mit dem bestimmten Begriffe der Unterordnung.\footnote{\emph{Sainte-Croix p. 27. 28. une quatrième divinité Cadmillus prit encore place parmi elles, mais il n'eut que le dernier rang.} Noch besser, ein andrer in den \emph{Mém. de l'Ac. des Insc. T. 27. p. 14. qui n'étoit employé, qu'à exécuter les ordres des trois autres.} Kreuzer, da ihm Axieros die höchste Gottheit ist, muss im Ganzen damit einstimmen, doch sucht er 2. 297. ff. andre Verknüpfungen, deren Absicht fast scheint dem Kadmilos eine andre Bedeutung als die des Hermes zu verschaffen (vgl. S. 317.), welches auch wohl sein müsste, wenn er der den drei andern untergeordnete sein sollte.} Also der Ceres, der Proserpina, dem Bacchus untergeordnet wäre Kadwilos oder Hermes? Denn auch das ist unbezweifelt, dass er Hermes ist. Mercur Diener dieser Gottheiten, der sonst vorzugsweise Bote des höchsten der Götter, des Zeus, heißt? Zwar Er ruft Proserpina aus der Unterwelt zurück, aber nicht im Dienste der Ceres, nur auf Jupiters Geheiß.\footnote{\emph{Hymn. in Cer. 336.}} Es findet sich zwar bei Varro der Ausdruck: Camillus, ein Gott, der großen Götter Diener.\footnote{\emph{Casmillus nominatur in Samothrakis mysteriis Dius quidam administer Diis magneis. De lingu. lat. 50. 6. p. 88. ed. Bip.}} Aber welcher Götter Diener, ist auch dadurch nicht bestimmt, vorausgesetzt selbst dass die kabirischen Gottheiten ohne Unterschied die großen genannt werden. Denn ihre Zahl wird sehr bestimmt auf sieben angegeben, denen ein achter beigesellt ist. Also Diener der großen Götter ist Camillus, nicht notwendig Diener jener drei ersten. Gesetzt aber, er diente zugleich den unteren und den oberen Göttern, so diente er jenen doch nur, sofern er der Mittler zwischen ihnen und den oberen, also selbst höher war denn sie, und dieses, gleichsam die leitende Verbindung zu sein zwischen den oberen und unteren Göttern, ist ja des Hermes eigentlichster Begriff.\footnote{--- --- --- --- \emph{superis Deorum}\\\hspace*{5mm}\emph{Gratus et imis.}\\\hspace*{15mm}\emph{Horat.} Od. 1. 10. \emph{extr.}} Er würde also den oberen und untern in sehr verschiedenem Sinne dienen, jenen als ein wirklicher Diener, als ein gehorchendes Werkzeug, diesen aber als ein wohltätiges und über sie erhabenes Wesen. Sehr zu fürchten demnach ist, dass durch die zu leicht angenommene Meinung, den drei ersten Gottheiten sei Kadmilos als Diener zugesellt, das ganze samothrakische System in ein falsches Licht gestellt worden. Entschieden bestätiget werden jene Zweifel durch die Namen selbst. Denn Kadmilos mit der griechischen Endigung, mit der ursprünglichen Kadmiel, heißt wörtlich: der, der vor dem Gott hergeht,\footnote{Ganz unnötig ist Bocharts Erklärung \emph{G. S. 50. 1. p. 395.} aus \foreignlanguage{hebrew}{\<.hdm>} und der vom Arabischen hergeholten Bedeutung \emph{ministrare}. Καδμίλος ist ganz einfach \foreignlanguage{hebrew}{\<qad:miy 'el>} von \foreignlanguage{hebrew}{\<qdmy>}, \emph{prior, antecedens}. Der Name Kadmiel, ebenso geschrieben, kommt in den späteren Büchern des A. T. und zwar als Name eines Priesters, eines Leviten vor. S. \emph{Esr. 2, 40. 3, 9. Neh. 7, 43. al.} Sicher bedeutet er nicht, wie gewöhnlich erklärt wird (s. \emph{Simonis p. 509.}, denn \emph{Hiller. Onom. Sacr.} musste ich leider bei dieser ganzen Arbeit enrbehren), \emph{quem Deus beneficiis prævenit}, sondern einen, der "`vor Gott steht"' (denn so wird der Begriff von \emph{ministrare} ausgedrückt, z. B. \emph{Gen. 18, 8.}, wo Abraham als ein wahrer Camillus vor den drei Männern sieht, vergl. \emph{Neh. 12, 44. Jer. 52, 12.} und dass röm. \emph{præminister} (\emph{Deorum Macr. Sat. 1. 8.}), welches denselben Nebenbegriff ausdrückt), oder einen, der "`Herold, Bote, Verkünder Gottes ist,"' (wovon in der folg. Anm.) oder "`der das Angesicht Gottes sieht,"' denn mit dieser Redensart würden \emph{Ministri} (auch der Könige) allgemein bezeichnet. Vgl. die selbst für die Etymologie des Worts nicht unwichtige Stelle \emph{Esth. 1, 10.} Die chaldäische Übersetzung des A. T. befleißiget sich \foreignlanguage{hebrew}{\<biN qadam yAy>} zu sagen, wo im Hebräischen blos steht \foreignlanguage{hebrew}{\<myhwh>}. S. \emph{Buxt. Lex. p. 19. 70.} Selbst die etruskische Zusammenziehung (\emph{Camillus}) ist hebräisch und besonders dem hierosolymitanischen Dialect des Chaldäischen eigen. Dort wird allgemein für \foreignlanguage{hebrew}{\<qdm>} und \foreignlanguage{hebrew}{\<qdmy>} blos \foreignlanguage{hebrew}{\<qm>} und \foreignlanguage{hebrew}{\<qmy>} gebraucht, s. \emph{Buxt. p. 1971.} Hebräisch ist die Zusammenziehung, denn sie findet sich im Namen \emph{Kemuel Gen. 22, 21. 1 Par. 27, 17.}, der gewiss unrecht durch \emph{grex Dei} (\emph{Sim. p. 509.}) erklärt wird, er ist statt \emph{Kemiel}, wie \emph{Genes. 32, 30. 31.} \emph{Peniel} und \emph{Penuel} unmittelbar hintereinander verwechselt werden, und dieses statt \emph{Kadmiel}.} und dieses wieder heißt nach morgenländischem Redegebrauch nichts anderes, als der Verkündiger, der Herold des kommenden Gottes. Er verhält sich insofern zu dem unbekannten Gotte, wie sich zu dem alttestamentlichen Jehovah der sogenannte Engel des Angesichts verhält.\footnote{Der \foreignlanguage{hebrew}{\<ml'k: hk*kym>} \emph{Es. 63. 9.}, auch \foreignlanguage{hebrew}{\<yhwh ml'k:>} schlechthin \emph{Exod. 23, 20. sq.} Eine ausführliche Erklärung dieses Begriffs findet sich im 1sten Teil der Weltalter. Wen muss es nicht verwundern, dieses Verhältnis durch die ganze heilige Geschichte beobachtet zu sehen, wie wenn Aaron Mosis Mund, also recht eigentlich sein Mercur (ἡγμούενος τοῦ λόγου \emph{Act. 14, 12.}) wird, Christo Johannes vorangeht, ihm den Weg bahnend, daher von einem Kirchenvater (\emph{Tertull. de orat. 1.}), der wahrscheinlich auf den Begriff des Camillus (s. Anm. 71.) anspielt, \emph{præminister domini} genannt. Was im A. T. der Engel des Angesichts, was in den griechischen Geheimlehren der Kadmilos, in etruskischer Religion Hermes-Camillus ist, das ist der späteren jüdischen Philosophie der \emph{Metatron}, ein sonderbarer Name, von dem vielleicht bei andrer Gelegenheit! Er ist der vornehmste Engel und ebenso erhaben über alle Engel, d. h. alle Naturen, die nur Boten, Werkzeuge der höchsten Gottheit sind, als nach unserer Ansicht der Kadmilos über die ersten Kabiren. Er heißt auch der Bote, der Gesandte, \foreignlanguage{hebrew}{\</sly.h>}, Eisenmengers entdecktes Judentum T. 2. S. 395., er ist auch der "`Fürst des Angesichts,"' der immer das Angesicht sieht des gebenedeieten Königs, das. S. 396. Von demselben sagen sie, er sei \foreignlanguage{hebrew}{\<zqn>} und \foreignlanguage{hebrew}{\<k.sr>}, zugleich alt und jung, er ist alt, als einer, der auffährt über die Himmel zum Thron der Herrlichkeit, jung, wenn er in die Welt der Formierung zurückkommt, d. h. Dienste als Camillus verrichtet, ebend. S. 397. Der etruskische Camillus war bekanntlich ein Knabe. "`Der Metatron, sagt ein jüdisches Buch, wird \emph{Naar}, d. i. ein Knabe genannt, weil er vor der Schechinah (der göttlichen Majestät) eines Knaben Dienste versieht,"' ebend. Die Etrusker haben ihre Vorstellung nicht von diesen späteren Juden, diese die ihrige eben so wenig von den Etruskern. Die gemeinschaftliche Quelle ist \emph{Prov. 8, 30.}, wenn man \foreignlanguage{hebrew}{\<'mwn>} richtig übersetzt; so wie ebendas. v. 22. Der Grund liegt vom Metatron als "`Anfang der Wege Gottes, Eisenm. \emph{l. c.}, und von Hermes als Gott der Wege.} Denn das Angesicht bedeutet dasselbe was Kadmi, nämlich das Vordere; der Engel des Angesichts also ist der Bote, der gleichsam das Vordere, das Vorausgehende der Gottheit ist. Nicht also der ihm Vorangehenden Götter, sondern eines kommenden, noch zukünftigen Gottes Diener ist Kadmilos. Auf einen nicht ihm vorangehenden, sondern ihm folgenden Gott deutet ebenso der andre, nicht minder urkundliche Name. Kasmilos heißt nicht überhaupt nur ein Ausleger der Gottheit, wie gewöhnlich erklärt wird, sondern bestimmt der von der Gottheit weissagt, sie voraus, die kommende verkündet.\footnote{Nicht blos \emph{interpres}, sondern \emph{augur quasi}, \emph{divinator Dei}. \emph{Es. 3. 2.} steht \foreignlanguage{hebrew}{\<qsm>} neben \foreignlanguage{hebrew}{\<kby'>}. Ist es wahr, wie erzählt wird (\emph{Plut. pl. phil. 50. 2. in.}), dass Pythagoras zuerst den Inbegriff aller Dinge κόσμος genannt, so sieht es zweideutig aus um die gewöhnliche Erklärung dieses Worts. Nach der Urlehre, aus deren Quellen Pythagoras schöpfte, ist die ganze Welt nur ein \emph{Kesem}, ein \emph{augurium Dei}. Ich bemerke, dass Kasmilos auch wohl noch in anderer Beziehung \emph{augur Dei} heißen kann; doch dies ist tieferer Erforschung, und das Verhältnis des vorangehenden bleibt dabei dasselbe. Die Herleitung von Kasmilos aus \foreignlanguage{hebrew}{\<qos:miy'el>} gibt schon Bochart, Hieroz. 2. 36. Wenn aber Münter in der angef. Abh. für die phönikische Erklärung von Kasmilos, dagegen für die ägyptische Erklärung der drei ersten Gottheiten spricht, so wäre unstreitig folgerichtiger zu schließen, dass weil Kadmilos, Kasmilos, Camillus unwidersprechlich und unbestreitbar hebräische Wörter sind, die andern, derselben Lehre und demselben Geheimdienst angehörigen, auch aus dieser Sprache sein müssen. Freilich meint Münter, aber ohne allen Grund, die drei ersten Kabiren seien aus Ägypten gekommen, der vierte erst von den Phönikiern eingeführt worden. \emph{Sainte-Croix} dagegen hält grade den Kadmilos für ägyptisch. Bemer, kenswert ist noch, dass von allen griechischen Völkerschaften grade die Böotier den Hermes Kadmilos nannten, dieselbe Völkerschaft, unter der (s. \emph{Larcher} zu \emph{Herod. 2. 49.}) die Nachkommen des Tyriers Kadmos und der Phönikier lebten, die dortin mit ihm gekommen waren. Auch bloß Kadmos heißt oft der Kadmilos.} Also durchaus auf einen zukünftigen Gott deuten die Namen, auf einen Gott, zu dem sich Kadmilos oder Hermes, notwendig also auch die ihm vorausgehenden Götter, nur als untergeordnete, nur als Diener, als Herolde, Verkündiger verhalten. Bewiesen wäre daher aus der Natur der einzelnen Persönlichkeiten selbst, dass weder die erste, Axieros, als Einheit und Quelle, der Götter und der Welt, vorangestellt, noch in der Kabirenlehre überhaupt ein Emanationssystem in ägyptischem Sinn enthalten ist.\footnote{So erklärt sich Kreuzer, Symb. und Myth. 2. 333. Es scheint diesem ausgezeichneten Werk überhaupt nicht vorteilhaft zu sein, dass zufolge einer sehr particularen philosophischen Ansicht, die man am Ende des 4ten Teils entwickelt findet und die dem Christentum, wie dem Altertum, nur gewaltthätig aufzudrinaen ist, allen Erklärungen die Emanations-Theorie zu Grunde gelegt worden. Indes kann diese Ansicht als etwas Fremdartiges rein abgeschieden werden von dem Werk, dessen unschätzbares Verdienst, durch höhere Ideen im Verein mit umfassender Gelehrsamkeit den Weg für eine tiefere Erkenntnis der ganzen Mythologie gebrochen zu haben, dadurch unangetastet bleibt. Insbesondre halte ich für recht, hier zu erwähnen, was eigentlich früher hätte erwähnt werden sollen, dass Kreuzer durch das Licht, in welches er die Ceres- und Proserpina-Lehre gesetzt, die ersten Mittel zu der Ansicht gegeben, die in gegenwärtiger Abhandlung entwickelt wird. Er hat, besonders 4. §. 39., unwiderleglich dargetan, dass Ceres das erste der Wesen ist, und dieser Lehrsatz, recht verstanden, das erste Wesen nämlich nicht mit Kreuzer zugleich für das oberste, sondern als das allem zu Grunde liegende genommen, ist das Fundament, auf welchem dieses Erklärungssystem ruht. Wenn daher derselbe geistvolle Gelehrte in Erklärung der samothrakischen Geheimnisse Zoega nachgibt, und mit ihm Axieros für die höchste Gottheit des ägyptischen Systems hält, so streitet dies gegen die Analogie der von ihm selbst anderwärts aufgestellten mythologischen Grundsätze.} Weit entfernt sich in herabsteigender Ordnung zu folgen, folgen sich die Götter in aufsteigender; Axieros ist zwar das erste, aber nicht das oberste Wesen, Kadmilos unter den vieren das letzte, aber das höchste. Natürlich ist dem sinnigen Forscher die Neigung, alles Menschliche so viel möglich menschlich zu begreifen, natürlich also auch, in Erforschung der alten Götterlehre ein Mittel zu suchen, wodurch die Vielheit göttlicher Naturen sich mit dem menschlich notwendigen und unaustilgbaren Gedanken der Einheit Gottes vereinigen ließe. Aber die Vorstellung der verschiedenen Götter als bloßer Ausflüsse Einer in ihnen, wie in verschiedenen Strahlen, sich fortpflanzenden Urkraft, ist weder an sich volksgemäß und deutlich,\footnote{Daher auch die Einschränkung a. a. O. "`Dieses Hervorgehen und Zurückkehren aus Einem Wesen und in dasselbe ward ohne Zweifel dem Gebildeteren als Grundlehre vorzutragen, die freilich der rohe Pelasger nicht zu fassen im Stande war. Ihm gab man dafür eine Reihe von Sterngöttern und ihnen entsprechende Baethylien, Idole von der Sternenkraft infuliert und magisch wirksam u. s. w."'} noch vermöge ihrer Unbestimmtheit und Grenzenlosigkeit verträglich sowohl mit der Bestimmtheit und Schärfe der Umrisse jeder einzelnen Gestalt als mit der geschlossenen Zahl dieser Gestalten. Allein auch mit menschlicher Denkweise lässt sie sich nicht wohl vereinen. Denn wer einmal zum Gedanken Eines höchsten Wesens sich erhoben, von dem alle übrigen Naturen nur Ausströmungen sind, wird sich schwerlich entschließen, diesen Ausflüssen seine Verehrung, geschweige jene aufrichtige, gefühlte Frömmigkeit zuzuwenden, die wir an Manchen der Weisesten und Besten, die entweder in die Mysterien oder in die Lehren der Philosophen eingeweiht waren, an einem Xenophon z. B., wahrnehmen. Ganz anders verhält es sich, wenn die verschiedenen Götter nicht abwärts gehende, immer mehr sich abschwächende Ausflüsse einer höchsten und obersten Gottheit, wenn sie vielmehr Steigerungen einer untersten, zu Grunde liegenden Kraft sind, die sich endlich alle in Eine höchste Persönlichkeit verklären; alsdann nämlich sind sie wie Glieder einer vom Tiefsten ins Höchste aufsteigenden Kette, oder wie Sprossen einer Leiter, deren tiefere nicht übergehen darf, wer die höheren erklimmen will; dann, weil sie dem Menschen Mittler sind zwischen ihm und der höchsten Gottheit und nur Boten, Verkündiger, Herolde des kommenden Gottes, gewinnt die Verehrung derselben einen Schein, der sich auch mit der besseren Menschheit verträgt und der allein erklärt, wie die den vielen Göttern erzeigte Verehrung so tiefe und fast unausreissbare Wurzeln schlagen, so lange sich erhalten konnte. Weder überhaupt zur Erklärung alter Götterlehre, noch zur Erklärung der samothrakischen insbesondere geeignet scheint also die Vorstellung der Emanation. Hier scheitert sie an dem recht verstandenen Begriffe des Kadmilos. Eine von unten wie Zahlen aufsteigende Reihe bilden die vier uns urkundlich bekannten samothrakischen Gottheiten. Kadmilos, nicht den drei andern untergeordnet, steht vielmehr über ihnen. Diese Einsicht verwandelt auf einmal die ganze Folge in eine lebendig fortschreitende, und öffnet uns die Aussicht in eine weitere Entwickelung der bis zur vierten Zahl bekannten Reihe. Die nächste Frage ist unstreitig, welcher Natur jener El, jener Gott sei, dessen Verkündiger und Diener zwar alle vorangehende Gottheiten, zunächst aber und unmittelbar Kadmilos ist? Unstreitig beginnt mit diesem Gott eine neue Reihe von Offenbarungen, durch die sich die Folge der Persönlichkeiten bis in die Sieben- und Achtzahl fortsetzt. Doch die vollständige Entwickelung dieser Reihe, die noch andere Mittel fodert als in den samothrakischen Überlieferungen für sich liegen, ist nicht unser Zweck. Es genügt uns, soweit dies möglich ist, über die Natur der dem Kadmilos folgenden Gottheit einigen Aufschluss zu geben. Zunächst also ist klar, dass jene ersten Gottheiten diejenigen Kräfte sind, durch deren Wirken und Walten vorzugsweise das Weltganze besteht; klar also, dass sie weltliche, kosmische Gottheiten sind. Denn sie heißen allesamt Hephäste,\footnote{Kreuzer 2. 321.} in keinem andern Sinn, als Alexander der Große sagte, auch Parmenio sei Alexander. Hephästos selbst ist in keiner Kabirenreihe, so wenig als sein Name unter denen der sieben Planeten oder in dem Kreis der Wochentage vorkommt, dem Schlüssel, wie ich einst zu zeigen hoffe, aller Göttersysteme. Sie alle zusammen, diese vorangehenden oder wie wir auch sagen können, dienenden Gottheiten, sind Hephästos.\footnote{Der mögliche Einwurf, dass Dionysos als höherer Demiurg dem Hephästos entgegengesetzt wird (Kreuzer 3. 414.) und doch nach obiger Ansicht selbst ein Hephästos ist, wird sich durch Anm. 80. erledigen.} Die Schöpfung des Hephästos ist die Welt der Notwendigkeit. Er ist es, der in strengem Zwange das All hält.\footnote{Kreuzer am eben angef. O.} Er aber ist es auch, der den Göttern, den höheren unstreitig als er selbst, die innerweltlichen Sitze bildet.\footnote{Τὰς ἐγκοσμίους ἕδρας. Ebenders. ebendas. aus \emph{Procl. in Plat. Theol. 6, 22.}} Eben dieses tun also jene dienenden Gottheiten, die sich auch dadurch wieder als die, nur die Epiphanie, die Offenbarung der höheren Götter vorbereitenden darstellen. Man könnte von ihnen sagen, sie sei'n nicht sowohl göttliche, als gottwirkend, theurgische Naturen, und als theurgisch stellt sich je mehr und mehr die ganze Verkettung dar. Wenn also jene vorangehenden Persönlichkeiten weltliche Gottheiten sind, so ist der Gott, zu dem sie die Führer und Leiter sind, und dem unmittelbar Kadmilos dient, der überweltliche Gott, der Gott, der sie beherrscht und dadurch Herr der Welt ist, der Demiurg oder im höchsten Sinne Zeus.\footnote{Aber auch Dionysos ist Demiurg und zwar der den Hephästos gewissermaßen überwindende Demiurg, der die Schöpfung aus den Banden der Notwendigkeit erlöst und in freie Mannichfaltigkeit auseinandersetzt. Dieser scheinbare Widerspruch löst sich schon durch die allgemeine Bemerkung, dass ein Wesen oder Princip, das höher steht als ein andres und insofern sein Gegensatz (sein Überwindendes) ist, dennoch gegen ein noch höheres mit jenem zu Einer Gattung gehören kann. Für Solche, die aus Andeutungen ein Ganzes verstehen, sei Folgendes! Auch Zeus ist wieder Dionysos, wie ja auch mitunter ausdrücklich gelehrt wurde. (S. die Anführungen von Kreuzer 3. 397. vgl. mit 416.) Nämlich Zeus verhält sich zu den drei ersten Potenzen wieder, wie sich die zweite zu der ersten verhält. Ich sage zu den drei ersten, obschon wir bisher vier zählten. Denn tiefer angesehen ist Ceres keine arithmetische Zahl. Sie ist die Mutter der Zahlen, die intelligible Dyas, mit der nach Pythagoreer-Lehre die Monas aller wirklichen Zahlen erzeugt. Persephone ist die erste Zahl (πρωτόγονος), die arithmetische Eins. Also Zeus verhält sich zu 1. 2. 3. wieder wie sich 2 zu 1 verhält und umgekehrt 2 verhält sich zu 1 nicht anders, als sich 4 zu 1. 2. 3. verhält. Des Zeus Zahl ist immer die vierte Zahl. Außerdem aber kehrt Dionysos noch einmal in höherer Potenz zurück. Axiokersos ist Dionysos in der tiefsten Potenz.} So hieß in Eleusis der, welcher den Hermes oder Kadmilos vorstellte, der heilige Herold, aber der höchste Priester, welcher den höchsten der Götter vorstellte, war das Ebenbild des Welterbauers und als solcher geschmückt.\footnote{Ἐν δὲ τοῖς κατ᾿ Ἐλευσῖνα μυστηρίοις ὁ μὲν Ἱεροφάντης ἐις ἐικόνα τοῦ Δημιουργοῦ ἐνσκευάζεται --- --- ὁ δὲ Ἱεροκήρυξ Ἕρμου. \emph{Eus. pr. ev. 3. p. 117.} Auch Samothraki hatte seinen Hierophanten. Er hieß Κόης, Κοίης. Bochart, \emph{G. S. p. 397.} leitet es, nicht eben unwahrscheinlich, von \foreignlanguage{hebrew}{\<khn>} ab. Da indes der Hierophant von Eleusis auch Προφήτης hieß (von τελετῶν κατάρχουσι προφήταις wird auch \emph{Euseb. l. c. p. 39. C.} gesprochen), so halte ich für wahrscheinlicher, dass das Wort so viel als \foreignlanguage{hebrew}{\<.hOzEh>}, Seher, ist, welches griechisch wohl nur durch Κόης oder Κοίης auszudrücken war. Das Wort scheint weniger allgemein als \foreignlanguage{hebrew}{\<kby'>}; dieses drückt die Eigenschaft, jenes das Amt aus (s. \emph{2 Sam. 24, 2. 1 Par. 21, 9. 25, 5.}) und von dem ist hier die Rede.} Der etruskisch römische Camillus war keineswegs ein jedem Priester ohne Unterschied dienender Gehülfe, er war, was bisher nicht beachtet worden, der ausdrücklichen Erklärung alter Schriftsteller zufolge, der dem Priester des Jupiters dienende Knabe.\footnote{\emph{Plut. in Num. c. 7. extr.} τὸν ὑπηρετοῦντα τῷ ἱερῷ (aber Reiske schon verbessert ἱερεῖ) τοῦ Διὸς ἀμφιθαλῆ παῖδα λέγεσθαι κάμιλλον, ὡς καὶ τὸν Ἕρμην οὕτως ἔνιοι τῶν Ἑλλήνων ἀπὸ τῆς διακονίας προσηγόρευον. Damit übereinstimmend \emph{Macr. Sat. 3. 8.}, "`\emph{Romani pueros puellasve nobiles et investes Camillos et Camillas appellant, flaminicarum et flaminum præministros.}"' Daher ist "`\emph{Festus de Verb. sign. p. 149. ed. in us. D.}"' \emph{Flaminius Camillus puer dicebatur ingenuus patrimus et matrimus, qui Flamini Diali ad sacra præministrabat},"' nicht so zu verstehen, als wolle er nur das Beiwort \emph{Flaminius} erklären; sondern der dem Jupiters-Priester dienende Knabe hieß ursprünglich und vorzugsweise \emph{Camillus}. Dass er ἀμφιθαλὴς sein musste, ein Knabe, des beide Eltern lebten, war nicht weniger bedeutend.} Da also dieser den Zeus selbst vorstellte, so war der Camillus zu ihm in demselben Verhältnis, in welchem nach der gegebenen Ansicht der kabirische Kadmilos zum höchsten Gott ist. Die aufsteigende Reihe verhält sich daher jetzt so: Das tiefste Ceres, deren Wesen Hunger und Sucht, und die der erste entfernteste Anfang alles wirklichen, offenbaren Seins ist. Die nächste Proserpina, Wesen oder Grundanfang der ganzen sichtbaren Natur; dann Dionysos, Herr der Geisterwelt. Über Natur und Geisterwelt das die beiden sowohl unter sich als mit dem Überweltlichen vermittelnde, Kadmilos oder Hermes.\footnote{--- --- --- --- \emph{commune profundis}\\\hspace*{5mm}\emph{Et superis numen, qui fas per limen utrumque}\\\hspace*{5mm}\emph{Solus habes geminoque facis commercia mundo.}\\\hspace*{15mm}\emph{Caudian} de R. Pr. 1. 89. ss.} Über diesen allen der gegen die Welt freie Gott, der Demiurg.\footnote{Merkwürdig genug bricht der Geschichtsschreiber Mnaseas schon mit Dionysos ab, sei es, dass er selbst keine höhere Weihe empfangen, sei es, was wahrscheinlicher, dass heilige Scheu ihn zurückhielt, das letzte Geheimnis auszusprechen. Einige, sagt der Scholiast, setzen den vierten hinzu. Nicht alle also gelangten bis zu dieser Zahl (des Kadmilos), mit der sich der Sinn des Ganzen erst aufschloss. Über diese Zahl hinaus führt kein Schriftsteller die Reihe; nur außer der Ordnung, einzeln werden Zeus, Venus, Apollo u. a. genannt. Umso natürlicher ist, den bei dem alten Scholiasten mit Kadmilos abreißenden Faden durch andere Bruchstücke fortzusetzen, die sich unter den Trümmern phönikischer Kosmogonien finden. Es kann umso weniger nötig sein, die Streitfrage über die Ächtheit oder Unächtheit dieser Bruchstücke aufzunehmen, da man ohnedies von beiden Seiten her angefangen, in den Mittelweg einzulenken. Doch ist vielleicht folgende Bemerkung an ihrer Stelle. Sanchuniathon erklärt sich als Feind jeder tieferen, wie er sie nennt, allegorischen, oder wie man heutzutage sagen würde, mystischen Bedeutung, als Eiferer für den rohen buchstäblichen Verstand der alten Göttergeschichten, die bei ihm völlig verwildert erscheinen. So roh und ohne alle Ahndung tieferen Verstands treiben sich in seinem wunderlichen Chaos auch die Trümmer um, von denen wir hier reden wollen. Ein Betrug, wie ihn Mosheim dachte, und mit solchem Zweck, hätte sich mit solchen Einschiebseln schwerlich Genüge getan. Nachdem also Sanchuniathon von den Korybanten und Kabiren gesprochen, fährt er fort: "`Zur Zeit derselben ward geboren ein gewisser Eljun mit Namen, der Höchste."' Durch leichte Änderung wäre der Sinn herauszubringen: Nach denselben; aber es ist für unsern Zweck unnötig; umso mehr da man diesem Bruchstück, wenn keinen höheren Ursprung zugestehen, doch in dem Mythus von der durch die Kureten und Korybanten (auch merkwürdig!) bewachten Geburt des Zeus seine Wurzel anweisen könnte. Eljun ist der wirkliche Name des höchsten Gottes \emph{Genes. 14, 18.}, des Priester jener aus dem Dunkel der Urzeit wunderbar hervortretende Malki-Sedek ist, Name des Gottes, der "`Himmel und Erde"' (so würde ja auch die kabirische Zweiheit ausgedrückt) besitzet, also des Weltherrn, des Demiurgen. Darf man die, vorzüglich von Kreuzer geltend gemachte Bemerkung auch hier anwenden, dass der Priester den Gott vorstellt und auch wohl dessen Namen trägt: so ist Malki-Sedek der Name des höchsten Gottes selbst, wofür auch spricht, dass schon die ältesten jüdischen Schriften, die hierinn sicher Überlieferungen folgten, z. B. das Buch Sohar, Sopher Jetziro, Beresit Rabba (s. \emph{Boch. G. S. p. 707.}), den Namen Zeus durch \foreignlanguage{hebrew}{\<.sdq>}, Sedek, ausdrücken. Jeder mit hebräischem Sprachgebrauch Bekannte weiß aber, dass Malki-Sedek nichts anderes bedeutet, als der vollkommene König, der vollendete Herrscher, also eben das, was \emph{1 Tim. 6, 15.} ὁ μακάριος (auch dies im Sinn von vollendet) καὶ μόνος δυνάστης, ὁ βασιλεὺς τῶν βασιλευόντων καὶ κύριος τῶν κυριευόντων heißt. Die anderen, nächst ihm vollkommensten, Naturen, herrschen zwar auch, aber sie herrschen nur als Werkzeuge; wie Diener eines irdischen Königes nicht als Selbstherrscher, sondern als Stellvertreter. Zu dem allem kommt Folgendes. Die sieben Söhne Sydyks (bei Damascius Sadiks) heißen urkundlich die Kabiren, \emph{Euseb. p. 39.} Der Sinn ist hier derselbe, wie wenn die ersten (untersten) Kabiren Söhne des Hephästos heißen. Nämlich sie alle zusammen sind nur Sydyk, der eine vollendete Herrscher lebt nur in ihnen, sie sind nur gleichsam die einzelnen Glieder des Einen; die den Vater verwirklichenden und sichtbar machenden Kräfte, die insofern auch in der Offenbarung oder Sichtbarkeit ihm vorangehen. Denn irren würde sich, wer aus diesem Verhältnis etwas für die Vorstellung der Emanation schließen wollte; es gilt hier, was ein in \emph{Bentl. Ep. crit. ad Mill. subj. Hist. chr. Joh. Mal. p. 81.} angeführter χρησμὸς in andrer Beziehung sagt: ὁ παλαιὸς νέος καὶ ὁ νέος ἀρχαῖος, ὁ πατὴρ γόνος καὶ ὁ γόνος πατήρ. Sind also die Kabiren Söhne Sydyks und war desselben Sydyks (Sedeks) Priester jener König von Salem, so wäre vielleicht erlaubt, zu sagen, dieser Malki-Sedek war der erste bekannte Kabir (so hießen ja auch die Priester und Geweihte), dem das System eröffnet war, das im Lauf der Zeiten vollendeter Klarheit bis in die Sieben-ja in die Achtzahl aufgeschlossen werden sollte. Doch nur zweifelnd dürfen diese ältesten Verbindungen angedeutet werden. Zu sichtlich, um vom besonnensten Forscher ganz von der Hand gewiesen zu werden, sind jene Anzeigen doch auch wieder zu schwach, um eine eigentliche Behauptung auf sie zu gründen. Eine größere, in weiterem Umfang und von andern Seiten hergeführte Untersuchung könnte jedoch ihre Kraft verstärken.} Also ein von untergeordneten Persönlichkeiten oder Naturgottheiten zu einer höchsten sie alle beherrschenden Persönlichkeit, zu einem überweltlichen Gott, aufsteigendes System war die kabirische Lehre.\footnote{Von einem solchen System sagten dann auch wohl im Altertum schon diejenigen, die es nicht bis zum Ende fortdachten oder verstanden, es sei nur Naturphilosophie. So Cicero \emph{de nat. D. 1. 42. Prætereo Samothraciam eaque}\\\hspace*{5mm}--- --- \emph{Quae Lemni}\\\hspace*{5mm}\emph{Nocturno aditu occulta coluntur}\\\hspace*{5mm}\emph{Sylvestribus sepibus densa}:\\\hspace*{5mm}\emph{Quibus explicatis ad rationemque revocatis rerum magis natura cognoscitur quam Deorum. Sainte-Croix p. 356.} "`\emph{Clément d'Alexandrie avoue, que l'Epoptie étoit une espèce de physiologie},"' dazu \emph{Strom. 4. p. 164.} Aber diese Stelle sagt etwas ganz Anderes, nämlich: "`die dem Kanon der Wahrheit (der christlichen Lehre) gemäße Naturphilosophie (Physiologie), eine Überlieferung höherer Erkenntnis, eher aber eine Epoptie zu nennen, fängt von der kosmogonischen Art der Untersuchung an und steigt von da zu derjenigen auf, die göttliche Dinge betrifft."'} Noch weit entfernt aber ist diese Darstellung von jener andern Behauptung, die zuerst Warburton ausgeschmückt, nach ihm auch deutsche Gelehrte annehmlich gefunden, welcher zufolge das eigentliche Geheimnis aller Mysterien des Altertums die Lehre von der Einheit Gottes war, und zwar in jenem verneinenden alle Vielheit ausschließenden Sinne, den die jetzige Zeit mit diesem Begriffe verbindet. Undenkbar wäre schon an sich ein solcher Widerspruch zwischen dem öffentlichen Götterdienst und der Geheimlehre. Er konnte, wie Sainte-Croix bemerkt, nicht kurze Zeit, geschweige an zweitausend Jahre dauren, ohne die Altäre umzustoßen, ja ohne die Ruhe der bürgerlichen Gesellschaft zu erschüttern. Mit der einen Hand erschaffen und mit der andern vernichten, öffentlich täuschen und insgeheim aufklären, den Götterdienst durch Gesetze befestigen, die Frevel dagegen mit Ernst bestrafen, heimlich den Unglauben nähren und aufmuntern, welche Gesetzgebung!\footnote{\emph{Sainte-Croix p. 355.}} Ein Gedanke, der einer in so manchen Verhältnissen an Betrug gewöhnten Zeit vielleicht zusagen konnte, den aber das grade, gesunde, kräftige Altertum wie mit Einer Stimme verwirft.\footnote{Die Freigebigkeit mit den Erklärungen durch Betrug, Priestergaukelei u. s. w. ist gewiss bezeichnend für die letzte Zeit. Der Lüge werden Kräfte zugetraut, die man kaum der Wahrheit zuschreibt. So blödsinnig auch war das Altertum nicht, wenn es gleich nicht mit vermeinter Schlauheit überall Täuschung witterte. Wenn nicht im Heidentum etwas sehr Ernstliches und mehr, als man denkt, Wirkliches lag, wie konnte der Monotheismus so lange Zeit brauchen, seiner Meister zu werden? Erweiterte Erfahrung, die von Zeit zu Zeit manches begreifen lehrt, was unbegreiflich schien, erteilte schon Warnungen genug. Die neuste betrifft die tönende Memnonssäule. Mancherlei Tatsachen, z. B. das periodische Aufhören und Wiederkommen des Tons, auch dass offenbar mehrere solcher tönenden Säulen waren, Umstände, die kürzlich Jacobs in der ang. Abh. mit scharfsichtiger Gewandtheit zusammengestellt, hinderten nicht, Priesteranstalt dabei zu vermuten. Nun kommen die gewiss unverdächtigen Franzosen, und siehe noch jetzt tönen bei'm Aufgang der Sonne die Granit-Blöcke des thebäischen Thals.} Es ist vielmehr alle Wahrscheinlichkeit, dass in den Geheimnissen ebendasselbe, was in dem öffentlichen Dienst, aber nur nach seinen verborgenen Beziehungen dargestellt wurde, und dass jene von diesem sich nicht mehr unterschieden, als etwa die esoterischen oder akroamatischen Vorträge der Philosophen von den exoterischen. Vollends aber jener, nicht altnicht neutestamentlich, nur etwa mohammedanisch zu nennende Monotheismus, dessen Begriff doch eigentlich immer jenen Behauptungen zu Grund gelegt wird, widerstrebt dem ganzen Altertum und der schöneren Menschlichkeit, die sich ganz in dem Ausspruche des Heraklit spiegelt, dem auch Plato Beifall gegeben: Das Eine weise Wesen will nicht das alleinige genannt sein, den Namen Zeus will es!\footnote{Οἶδα ἐγὼ καὶ Πλάτωνα προσμαρτυροῦντα Ἠρακλείτῷ, γράφοντι • Ἔν το σοφὸν, μοῦνον λέγεσθαι οὐκ ἐθέλει, καὶ ἐθέλει Ζηνὸς ὄνομα. \emph{Clem. Al. Strom. 50. 5. p. 718.} Vgl. Voß zu Virgils Landbau S. 808. Mohammedanisch darf der Monotheismus wohl heißen, der nur Einer Persönlichkeit, oder einer ganz einfachen Kraft den Namen Gott zugedacht. Dass er nicht neutestamentlich, bedarf keines Beweises; dass auch nicht alttestamentlich, darüber s. Weltalter 1ster Teil.} Eine andere Folge möchte man versucht sein, jener obwohl flüchtigen Vergleichung zu geben zwischen samothrakischen und alttestamentlichen Vorstellungen, die weiter fortgesetzt noch auf tiefere Übereinstimmungen leiten würde. Man könnte darin eine neue Bestätigung sehen wollen der älteren von Gerhard Vossius, Bochart und anderen ehrenwerten Forschern gefassten Ansicht. Nach derselben ist die gesamte Götterlehre des Heidentums nur Verunstaltung der alttestamentlichen Geschichte und der an das Volk Gottes ergangenen Offenbarung.\footnote{Vgl. Kreuzer Vorr. zu 4. S. 4. Gefallen ist wohl dieses System weniger durch sich selbst als durch die abgeschmackten Anwendungen, eines \emph{Huetius} z. B.} Diese also wird für ein Äußerstes und Letztes genommen, über das keine geschichtliche Erklärung hinausgehen kann. Wie aber, wenn diese Annahme selbst nur willkührlich wäre? Wenn sich schon in griechischer Götterlehre (von indischer und anderer morgenländischer nicht zu reden) Trümmer einer Erkenntnis, ja eines wissenschaftlichen Systems zeigten, das weit über den Umkreis hinausginge, den die älteste durch schriftliche Denkmäler bekannte Offenbarung gezogen hat?\footnote{Ich sage: eines wissenschaftlichen Systems, nicht eines bloß instinktmäßigen Erkennens, etwa in Visionen oder im Hellsehen oder auf andere ähnliche Arten, die man sich heutzutage ausdenkt, da einige geradezu der Wissenschaft entsagen, andere wo möglich ein Wissen ohne Wissenschaft aufbringen möchten. Da übrigens das Dasein eines solchen Ursystems, das, älter als alle schriftliche Denkmäler, die gemeinschaftliche Quelle aller religiösen Lehren und Vorstellungen ist, im Text nicht eigentlich behauptet, sondern nur als eine Möglichkeit hingestellt wird, so wird es wohl verstattet sein, dieser Anführung wegen auf künftige, nicht einen Teil betreffende, sondern es selbst (das Ursystem) in seiner Ganzheit herzustellen suchende Forschungen zu verweisen, nach deren Mitteilung dann gegen die Annahme sich erklären mag, wer sie nicht als die wahrscheinlichste erkennen zu müssen glaubt.} Wenn überhaupt diese nicht sowohl einen neuen Strom von Erkenntnis eröffnet hätte, als den durch eine frühere schon eröffneten nur in ein engeres, aber eben darum sicherer fortleitendes Beet eingeschlossen? Wenn sie, nach einmal eingetretener Verderbnis und unaufhaltsamer Entartung in Vielgötterei, mit weisester Einschränkung, von jenem Ursystem nur einen Teil, aber doch diejenigen Züge erhalten hätte, die wieder ins große und umfassende Ganze leiten können? Diesem jedoch sei wie ihm wolle, so beweisen jene Vergleichungen wenigstens, dass der griechische Götterglaube auf höhere Quellen als auf ägyptische und indische Vorstellungen, zurückzuführen ist. Ja wenn die Frage entstünde, welche von den verschiedenen Götterlehren, ob die ägyptische und indische, ob die griechische näher der Urquelle geschöpft sei; der unbefangene Forscher würde kaum anstehen, für die letzte zu entscheiden. In der griechischen Fabel, jener Göttergeschichte, wie sie vorzüglich Homer den Griechen gedichtet, ist es eine unschuldige, fast kindische Phantasie, die, nur gleichsam versuchsweise, spielend und mit dem Vorbehalt es wiederherzustellen, das Band auflöst, wodurch die vielen Götter Ein Gott sind; im ägyptischen und indischen System ist ein ernstlicher Missverstand, ja ein dämonisches nicht zu verkennen, ein nur mit Absicht wirkender Geist des Irrtums, den der Missverstand ins Ungeheure, ja ins Gräuelhafte auswirkt. Hatte jenes Naturvolk der Pelasger, aus dem alle griechische Kraft und Herrlichkeit aufgegangen scheint,\footnote{Ohngefähr, wie alle Kraft und Herrlichkeit des neueren Europas, aus den germanischen Völkern, mit denen die Pelasger überhaupt manche Züge gemein haben; ihre Wanderungen, und die Urteile, die von späteren Geschichtsforschern über beide ergangen (s. unter andern \emph{Larcher Chronol., Herodote T. 7. p. 277.}), sind nicht die stärksten derselben.} die Grundbegriffe schon getrübt, nicht in natürlicher Unschuld und Frische erhalten, nimmer, so hoch wir den lebendigen Sinn der Griechen anschlagen mögen, konnten diese Vorstellungen in so lautere Schönheit sich entfalten, nimmer so treu, so arglos, so unbefangen mitten im Spiel, jene tieferen Verbindungen bewahren, deren geheimer Zauber uns auch dann noch trifft, wenn wir die Göttergestalten in ihrer völligen dichterischen und künstlerischen Unabhängigkeit vor uns walten lassen. Wiederhergestellt wurde jenes im Spiel der Dichtung gelöste Band im Ernst der Geheimlehren. Geschichtlich unzweifelhaft ist, dass auch diese den Griechen vom Ausland oder von den Barbaren gekommen. Aber warum eben aus Ägypten? Weil Herodot von den dodonäischen Priesterinnen gehört, aus Ägypten haben die Pelasger zuerst die Namen der Gottheiten erfahren?\footnote{\emph{50. 2. c. 52. extr.}} Aber derselbe Herodot gibt kurz zuvor eben diese Herleitung der griechischen Götternamen aus Ägypten nur für seine Meinung,\footnote{Δοκέω δ᾽ ὧν μάλιστα ἀπ᾿ Ἀιγύπτου ἀπῖχθαι. 2. 50.} die umso weniger Entscheidendes haben kann, da ihr so wesentliche Mittel der Beurteilung fehlten und Urkunden, die vor uns aufgeschlagen liegen.\footnote{Dass die Namen der meisten Götter nach Griechenland aus Ägypten gekommen sei'n, kann ohnehin nicht buchstäblich genommen werden. Vielleicht wenn Herodots Kenntnisse weiter sich ausdehnten, weit entfernt die griechischen Götternamen aus Ägypten abzuleiten, zweifelte er, ob die ägyptischen selbst ägyptischen Ursprungs seien. Von Osiris war schon die Rede. Wer sich noch mehr überzeugen will, sehe die ebenso ungewissen als flachen Erklärungen an, die von ägyptischen Götternamen aus der koptischen Sprache seit Kirchers Zeiten, von Jablonsky, \emph{Georgii} (\emph{Alphab. Tibet.}), \emph{Zoëga} und andern, gegeben worden sind. Wie unnütz also muss es erscheinen, ägyptische Etymologien noch weiter, auch auf griechische Namen, auszudehnen! Hievon nur Ein Beispiel an dem orphischen Herikapäos. Ehemals wollte man in ihm durch kabbalistische Rechnung den \emph{Schem hamphorasch} (Jehovah-Namen) finden; das nennt \emph{Bentley ep. ad Mill. p. 4.} mit Recht \emph{aniles Cabalistarum fabulas}. Aber nun kam das ägyptische Vorurteil. Münter allein in der ang. Abh. S. 34. Anm. gibt zwei Erklärungen. Noch mehrere kann man bei Kreuzer 3. 388. angeführt finden. Bentley, der den Namen doch nicht loswerden kann und p. 90. zum zweitenmal auf ihn zurückkommt, begnügt sich, zu bemerken, die Sylbe κεπ (nach der Lesart Ἡρικεπαῖος) könne nimmermehr weder griechisch noch lateinisch sein. Darum habe er wohlgetan, den \emph{ineptis plerumque et cassis Etymologiis} (nämlich aus griechischer Sprache) nicht nachzugehen; schwören wolle er, dass selbst Orpheus keine anzugeben wüsste. Ohne sich zu vermessen, könnte man dagegen schwören, eine (freilich nicht griechische) Etymologie anzugeben, die der unvergleichliche Bentley selbst, wenn er von den Toten wiederkäme, als die wahre erkennen würde. Doch erwarte man darum nichts außerordentliches, sondern nur etwas ganz leichtes und einfaches. Das Wort Ἡρικαπαῖος ist nicht mehr noch weniger als das hebräische \foreignlanguage{hebrew}{\<'ErEk: 'ap*ayim>} (\emph{Erec-Apaim}), dass \emph{Exod. 34, 6.} und anderwärts als Name oder Prädicat des wahren Gottes vorkommt, oder, damit es noch ähnlicher sei, nach der chaldäischen Form (s. \emph{Buxt. Lex. p. 216.}) \foreignlanguage{hebrew}{\<'ryk 'pyn>} (\emph{Erik-Apain}), welches den Langmütigen, Mitleidigen, der weiten Herzens ist, bedeutet. Und das ist er ja, der Herikapäos, der mit Dionysos (s. Anm. 65.) so viel Ähnlichkeit hat (αὐτὸς δὲ ὁ Διόνυσος, sagt \emph{Proclus cit. ad Orph fragm. ed. Gesn. = Herm. p. 466.}) καὶ Φάνης καὶ Ἡρικαπαῖος συνεχῶς ὀνομάζεται der Lebensgeber (ζωοδοτηρ, \emph{Malal. hist. chron. p. 91.}), der weitherzige Gott, im Gegensatz mit dem engherzigen, der das Leben vielmehr verschließt, hindert. Es ist den Hellenisten zu verzeihen, wenn sie, immer die griechischen Herleitungen im Auge, den Etymologien nicht hold sind; auch Ruhnkenius hat sich nicht Einmal stark darüber erklärt. Doch sollte man nicht alle und aus jeder Sprache verschmähen, denn z. B. weder in kritischer noch in historischer Beziehung kann es unwichtig sein, zu wissen, dass der orphische Herikapäos hebräisch oder alttestamentlich ist. Doch dieser Name ist ja blos orphisch und beweist also nichts für ägyptische! Nun höre man \emph{Plut. de Is. et Os. p. 359.} Τὸ δὲ ἕτερον ὄνομα τοῦ θεοῦ (τοῦ Ὀσίριδος) τὸν ΟΜΦΙΝ εὐεργέτην ὁ Ἑρμαῖός Φησιν δηλοῦν ἑρμηνευόμενον. Omphis also ein zweiter Name des Osiris? Hier weiß sogar Jablonski keinen Rat. Die Stelle muss verdorben sein, gewiss hat Plutarch Ῥόμφιν geschrieben, denn nur für ein solches Wort lässt sich aus dem Koptischen die Bedeutung des Wohlthuenden herbeischaffen, \emph{Vocc. æg. in Opusc. ed. de Water 1. p. 184.} Wenn man aber weiß, dass dasselbe Wort, aus dem oben der Herikapäus erklärt worden, auch (oder vielmehr ursprünglich) \foreignlanguage{hebrew}{\<'rk 'kpyn>} (\emph{erik-anphin}) geschrieben wurde, so wird man durch die sehr natürliche Abkürzung nicht nur den Namen, sondern auch die Bedentung: der Wohltätige, erklärt finden. Wir könnten nun noch weiter gehen; denn jener \emph{Erik-apin} verlangt ja auch seinen Gegensatz, einen engherzigen Gott; es scheint nicht schwer zu sagen, in welchem Namen dieser zu finden ist; aber dies mag einstweilen hinreichen, nur aufmerksam zu machen. Der Zweifel, der in Ansehung der ägyptischen Götternamen geäußert worden, dürfte mit der Zeit wohl auch in Ansehung der indischen laut werden; versteht sich der bedeutendsten. Dass ein Volk die Namen der Götter, die es nicht selbst erfunden, nicht zu verändern gewagt, ist bei weitem mehr, als das Gegenteil wahrscheinlich. Auch an die Namen war ein Zauber geknüpft, und was der allgemeine Aberglaube von Beschwörungsformeln hält, dass sie nur in der Sprache wirken, in welcher sie überliefert worden, galt wohl auch von Götter-Namen. So behielt Samothraki mit dem alten Dienst nicht nur die alten Namen, sondern auch in heiligen Gebräuchen gewisser Ausdrücke einer eigenen alten Sprache (παλαιᾶς ἰδίας διαλέκτεου) bis auf Diodors von Sicilien Zeiten, der diese Wörter zwar von den Autochthonen der Insel herleiten will (\emph{50. 5. p. 357.}), die aber alle wahrscheinlich von der Art des Worts κοίης waren (Anmerk. 81.). So behielt Eleusis die fremdlautende Entlassungsformul, so die Sabazien ihr Hyes Attes! näherliegender Vergleichungen nicht zu gedenken! Doch wozu auch nur dieses, da das Beispiel der beweglichen Griechen allein entscheidend ist, die selbst im freien dichterischen Gebrauch die Namen beibehielten, von denen Herodot (weit bestimmter davon redend, als von der ägyptischen Herkunft) sich durch seine Untersuchungen überzeugt zu haben versichert, dass sie mit wenigen Ausnahmen (die auch nicht einmal alle Ausnahmen sind) den Griechen von den Barbaren gekommen.} Welche ganz andere Welt ging dem Vater der Geschichte auf, wenn er die althebräischen Denkmäler kannte, ihm, dessen Aufmerksamkeit nicht entgangen war, dass die ersten bacchischen Orgien Griechenlands von jenen Phönikiern herkommen, die sich mit dem Tyrier Kadmus in Bäotien niedergelassen.\footnote{\emph{Herodot. 50. 2. 49. extr.}} Über die Geheimnisse Samothrakis äußert er sich entschieden, die Insel habe sie von den Pelasgern empfangen, welche zuerst dort, ehe mit den Athenern zusammen gewohnt.\footnote{\emph{50. 2. c. 51.}} Der einzige, doch nur scheinbare Grund, der einige Forscher bewegen konnte, die erste Quelle des Kabiren-Dienstes in Ägypten zu suchen,\footnote{Münter \emph{l. c. p. 30.} Kreuzer 2. 285. ff. Jacobs über die Memn. Anm. 63.} ist eine besondere Erzählung des ionischen Geschichtsschreibers. Zu Memphis sei Kambyses in das Heiligtum des Hephästos gegangen und habe des Bildes nicht wenig gespottet. Denn es sei, wie die Phönikischen Patäken, die Nachbildung eines Pygmäen. Auch in das Heiligtum der Kabiren, in das niemanden als dem Priester zu gehen verstattet gewesen, sei der Frevler gedrungen und habe unter vielem Gelächter ihre Bilder verbrannt, denn auch sie seien den Bildern des Hephästos ähnlich.\footnote{\emph{50. 3. c. 37.}} Die Vergleichung der Hephästos- und der Kabiren-Bilder mit den phönikischen Schutzgöttern würde jedoch die umgekehrte Ableitung der Kabiren Ägyptens von den Zwerggöttern Phöniziens ebenso gut verstatten, da dieses Land nach ebenso unverwerflichen Zeugnissen zu den ältesten Sitzen der Kabiren gehört.\footnote{\emph{Euseb. pr. ev. p. 38.} "`Saturn gab dem Poseidon und den Kabiren die Stadt Beryth zum Sitz."' Es ist dies die einzige mir bekannte Stelle, wo Poseidon und die Kabiren zusammen, aber unterschieden, genannt werden. Nämlich Poseidon ist der Kabiren oder vielmehr sie sind des Poseidons Gegensatz. Er ist das blindlings Auseinanderrollende, Spaltende, Zertrennende; sie das Zusammenhaltende. Die Kabiren halten den Poseidon bewältigt, sie selbst überwindet wieder ein andrer, der insofern κύριος τῆς ὑγρᾶς Φύσεως (Anm. 66.), aber in einem ganz andern Sinn ist, als der zerstörende Poseidon. Spuren genug dieses Gegensatzes gibt der 21. Gesang der Ilias. Grade die Verbindung des Poseidons und der Kabiren in jener Stelle ist Beweis der Urkundlichkeit der Angabe. Man erinnere sich: was Herodot eben auch über Poseidon sagt. Nach dieser Ansicht möchte auch was §. 5. \emph{in.} des Textes erwähnt ist, noch eine tiefere Deutung zulassen. Insofern schiene mir die Bochartische Erklärung von Patäken noch immer vorzuziehen, wenn man als den Grundbegriff von \foreignlanguage{hebrew}{\<b.t.h>} (wohl nicht unrichtig) \emph{firmus fuit, firmiter innixus est} annähme. Patäken wären alsdann die festmachenden, die sichern Grund gebenden; Gegensatz von \emph{instabilis tellus, innabilis unda}.} Über die erste Herkunft des Kabiren-Dienstes möchte also aus dieser Geschichte nichts zu schließen sein. Umso merkwürdiger ist an sich die Erzählung, in Pygmäen-Gestalt habe man die Kabiren zu Memphis gesehen. Wie reimt sich diese Gestalt, wir wollen nicht sagen mit jener Vorstellung, nach welcher Hephästos der höchste Gott sowohl des kabirischen als des ägyptischen Systems ist und alle anderen Götter nur Ausflüsse von ihm; wie reimt sich eine solche Abbildung auch nur mit dem Namen der großen Götter, welcher den Kabiren so allgemein beigelegt wird? Einer der älteren Untersucher wollte die Schwierigkeit durch Auslegung hinwegräumen,\footnote{\emph{Gutberleth Diss. de Mysteriis. Deorum Cabirorum, insert. ej. Opusc. Franeck. 1704. et Poleni Suppl. ad Thes. antt. Gr. et R. T. 2. p. 824.} Dieser übersetzt πυγμαίου ἀνδρὸς μίμησιν durch \emph{fortis et robusti viri imaginem}, ein Sprachgebrauch, für den er nichts anzuführen weiß, als \emph{Ez. 27, 2.} nach der griechischen Übersetzung des Aquila, wo πυγμαῖσι, meint er, dem Zusammenhang nach nur starke Männer bedeuten könne.} untunlich schon darum, weil unzweifelhafte Spuren sind, dass dieselben Götter auch außer Ägypten zwergartig gebildet wurden.\footnote{\emph{Kreuzer. Dionysus p. 133. ss.}} In bildlichen Vorstellungen, wie bei dem Dichter, trägt der Greis Anchises die vaterländischen Penaten in der Hand aus Ilium,\footnote{\emph{Tu, genitor, cape sacra manu patriosque Penates.} \emph{Aen.} 2. 717.} ein Beweis wenigstens der Kleinheit dieser Bilder, die Gottheiten darstellten, welche mit kabirischen zunächst verwandt waren. Man könnte versucht sein zu sagen: die ersten Kabiren wenigstens sei'n alle dienende Gottheiten oder Camille, darum sei'n sie als Knaben gebildet worden. Aber Knaben sind keine Zwerge. Angemessener ist folgendes, zumal es auf einer Vorstellung beruht, die erweislich vorhanden war.\footnote{S. die 72ste Anm.} Als Götter und als die ältesten der Wesen wurden sie notwendig in ehrwürdiger Gestalt und als Alte gedacht; als Camille aber jugendlich und wie Knaben. Die noch rohe aber aufrichtige Idoloplastik wußte diese streitenden Begriffe nur in der Gestalt von Zwergen zu vereinigen. Vorauszusetzen dabei ist allerdings, was sich indes auch sonst rechtfertigen lässt, dass nur die ersten Kabiren in solcher Gestalt abgebildet wurden, denn nur als Söhne des Hephästos, nur sofern selbst Hephäste, waren die Kabiren zu Memphis in Pygmäengestalt zu sehen. Sonst glauben wir, einen, der menschlichen Einbildungskraft auch sonst gewöhnlichen und besonders wieder an altdeutsche und nordische Vorstellungen erinnernden, Zug darin zu finden, diesen nämlich, große aber mehr zauberische als natürliche Kräfte mit der Zwergengestalt vereinigt zu denken. Hat doch schon einer alten nichts weniger als gradezu verwerflichen Ableitung zufolge unser deutsches Wort Zwerg das griechische Theurgos zur Wurzel und demnach von Hausaus die Bedeutung eines theurgischen zauberkräftigen Wesens. Auch an unsre Bergmännlein dürfen wir erinnern, von denen noch unser treuherziger Landsmann Georg Agricola zu erzählen weiß; denn auch sie sind ja so zu reden Söhne des Hephästos, die mit Metallen Verkehr haben und sogar Waffen aus ihnen verfertigen.\footnote{\emph{Wachter. Gloss. Germ. 2. pag. 1989. Zwerg} (\emph{Anglosax. dwerg, dweorh, Franc. duverch}), \emph{Dæmon silvestris montes et saxa inhabitans, vocem compellantibus reddens, et nescio quae arma fabricans, secundum Mythologiam Islandorum, cui nomen Edda. Verel. in Jnd. duergur et in plural. duergar, semidæmones, rupicolæ, arte fabrili mirabiles.} \emph{Gudmundo Andreae in explicatione Voluspæ Stroph. 7. dwergi sic dicuntur à} θεοῦ ἔργον; warum nicht gradezu von θεουργός? Über die Bergmännlein, Wichtlein u. s. w. lässt sich der wackere G. Agricola (s. über ihn v. Goethes Farbenlehre 2. S. 237.) in der Abh. \emph{De Animantibus subterraneis} (\emph{De re metall. libri 12. p. 491.}) deutsch übersetzt also vernehmen. "`Von andern werden sie (die die Griechen \emph{Cobalos} nennen) Bergmännlein genannt nach ihrem gewöhnlichen Leibesmaß, denn sie erschienen wie Zwerge dreier Spannen hoch und zwar wie alte Männlein (\emph{seneciores}), gekleidet wie die Bergleute, in einem gekappten Hemd und mit einem um die Lenden herabhangenden Schurzleder (wie Kabiren auf Münzen; Kabiren-Hammer und Schlägel fehlt in andern Beschreibungen auch nicht). Diese pflegen den Erzgräbern keinen Schaden zu tun, sondern schweifen herum in den Schachten und Gängen und scheinen alle mögliche Arbeiten vorzunehmen, da sie doch nichts tun. Bisweilen werfen sie die Arbeiter mit Steinchen, verletzen sie jedoch nie, wenn sie nicht gereizt und in ihrer Gaukel-Arbeit gestört werden. Weshalb die Bergleute durch sie von der Arbeit nicht abgeschreckt, sondern als durch ein gutes Zeichen aufgemuntert werden, desto eifriger und stärker drauf zu setzen und stärker zu arbeiten."' Auch aus Theophr. Paracelsus wäre viel von den \emph{Pygmæis} anzuführen, dass er doch wohl nicht bloß aus seinem Gehirn, sondern aus gemeiner Volkssage genommen. Ob er ihnen gleich manches Böse nachsagt, rühmt er sie doch auch wieder als solche, "`die oft unsre Warner, Wächter und Beschützer sind in großen Nöten, helfen oft einem außer Gefängnis und dergleichen Hülfe mehr."' Den böhmischen Gebirgsbewohnern kommen sie bis in die Häuser als wahre \emph{lares familiares} oder \emph{lemures}, dass man sie unter der Erde kann hämmern und schmieden hören, deshalb heißen sie dort und in angränzenden deutschen Ländern auch Hausschmiede. S. \emph{Balbin. Misc. hist. Boh. 50. 1. p. 45.} In dieselbe Classe gehören die ebenfalls von Agricola p. 492. erwähnte \emph{dæmones, qui quotidie partem laboris perficiunt, curant jumenta, et quos, quia generi humano sunt aut saltem esse videntur amici, Germani Gutelos appellant} (Χρηστοὶ heißen auch die Kabiren, \emph{Macrob. 3. 4.} vielleicht auch jene alttest. Penaten die Theraphim, nach dem arab. $\symbolAAQ$. Aus einem alten Wörterbuch führt \emph{Scherz. Gloss. Germ. med. æv. 2. pag. 2011.} an: "`\emph{wichtelein, wichte, schretlein penates}."' Die sanftmütigen heißen vorzugsweise Kobeln, Kobolde, ein Wort das schon Agricola und nach ihm Wachter u. a. vom griech. Κόβαλος herleiten. Nun sagt \emph{Is. Vossius ad Hesych. voc.} Καβάρνοι \emph{not. 12.} "`Καβάρνοι, Κάβειροι, Κόβαροι, Κόβαλοι (auch wohl das bei Hesych. gleich darauffolgende κόβειρος) \emph{ejusdem omnia videatur naturæ}."' Bei den unzähligen Beispielen der Verwechselung von \emph{R} und \emph{L} ist nicht zu zweifeln, dass κόβαλοι für κόβαροι gesetzt wird, und dass dies mit κάβεὶρος einerlei \emph{Etymon} habe, ist eben so wenig zweifelhaft. Hiedurch ist also die in den Vorstellungen nachgewiesne Verbindung auch in den Namen aufgezeigt. Oft genug während dieser Untersuchung hat sich der ebenso nahe als tief eingreifende Bezug zwischen den Kabiren und den Laren und Manen dargeboten (vgl. \emph{Arnob. adv. Gent. 3. p. 124. ed. Lugd. Bat.}), aber wir mussten uns einschränken.} Da indes mit der Gestalt von Pygmäen grade der Begriff übernatürlicher Stärke verbunden ist, so könnte es nicht befremden, wenn etwa die, welche als Zwergen, in einer andern Wendung als Riesen gedacht wurden,\footnote{Die Spur einer solchen Vorstellung könnte man in dem Namen \emph{Anaces} (Ἄνακες) suchen wollen: denn wie dem auch sei, die einzig wahrscheinliche Erklärung dieses erst späten in \emph{Anactes} verwandelten Worts (s. \emph{Cic. de n. D. 3. 21.}) liegt in dem Enakim der Vorzeit, \emph{Deut. 1, 28.} In der nordischen Fabel und Dichttunst finden sich meist, wo Riesen, auch Zwerge. Wer denkt nicht an das "`viel starke Gezwerg,"' dass zugleich mit dem Nibelungen-Recken der Schätze und Burgen hütet und dem nebst großer Stärke Zauberkraft inwohnt?} nicht auffallen, wenn unter den, wie es scheint, noch kleiner gedachten, Idäischen Daktylen\footnote{Πυγμαῖος wird von πυγμὴ, einer Faust hoch, erklärt. Δάκτυλοι sind Finger.} auch Hercules genannt wird, und wenn jenes unförmliche Bild der ältesten Kabiren sich in die herrlichen Gestalten der Dioskuren verklärt.\footnote{Söhne Sydyks (Anm. 84.) und Dios-Kuren ist einerlei Name. Aber derselbe Name ist ja noch urkundlicher in jenen \foreignlanguage{hebrew}{\<bky h'lhym>} \emph{Gen. 6.} vorhanden (dass darunter Söhne des höchsten Gottes gemeint sind, zeigt das \foreignlanguage{hebrew}{\<h>} \emph{emphat.} vor \foreignlanguage{hebrew}{\<'lhym>}). Von diesen erzählt das älteste Geschichtswerk: "`Und die Söhne Gottes sahen die Töchter des Menschen, dass sie schön waren, und nahmen sich zu Weibern, die ihnen gefielen,"' worauf in demselben Zusammenhang folgt: "`In jenen Tagen waren Nephilim (Riesen) auf der Erde, zumal nachdem die Söhne Gottes sich mit den Menschentöchtern verbanden und sich Kinder zeugten. Diess sind die Gewaltigen, die Männer des Namens (die Berühmten) von Urzeiten der Welt her."' Es ist doch etwas ganz wunderbares um diese Stelle, mag sie nun für mythisches Bruchstück nach der beliebten Weise oder für Geschichte genommen werden. Will man nicht den ungereimten jüdischen Fabeln Glauben beimessen, so kann man \foreignlanguage{hebrew}{\<bky h'lhym>} nur von Verehrern des wahren Gottes erklären, die gleichsam als abgesondert von den übrigen Menschen und als ein eignes Geschlecht vorgestellt werden. Es waren also so zu reden die Eingeweihten der ersten und ältesten Mysterien; von Anfang an war etwas abgeschlossen, nur einem Teil des Menschengeschlechts vertraut, das sich erst allmählig wie von einem Mittelpunkt aus verbreiten sollte. Ist es nicht auffallend, dass aller höhere und bessere Glaube gleich anfänglich in Griechenland und sonst unter der Form von Geheimlehren auftritt? Augenblickliche und örtliche Ursachen lassen sich doch nicht immer und überall denken, sondern das Geheimnis, die Abgeschlossenheit schien, gleich ursprünglich und vom Anfang her, zugleich mit der Sache selbst gegeben. Söhne des höchsten Gottes wurden jene Inhaber der ältesten Geheimlehre, wie die in ihrem Ursprung offenbar menschliche Zwillinge Dios-Kuren wurden und zuletzt selbst unter die Kabiren übergingen. Von diesen höheren Naturen stammen die ersten menschlichen Heroen, die Nephilim (Niflungen?) die gewaltig waren, solange sie lebten und noch in der Unterwelt (Niffelheim der altnordischen Mythologie?) groß und berühmt sind, s. \emph{Es. 14, 9.} Jeder mag suchen diese wunderbaren Anzeigen so gut er kann weiter zu verknüpfen, aber sehr natürlich ist doch, sich nach einer Erklärung der so allgemeinen Mysterien-Form schon in den ältesten Zeiten umzusehen. Was war auch die strenge Absonderung des jüdischen Volks anders, als eine den Mysterien ähnliche Anstalt, nur dass sie nicht zwischen Menschen desselben Volks, sondern zwischen einem Volk und allen übrigen eine Scheidewand zog? Erst das Christentum sollte alle Schranken aufheben.} Auf den Begriff zauberischer, theurgischer Kräfte führt also auch die Angabe ihrer Gestalt zurück. Welche Bedeutung durch ihren gemeinschaftlichen Namen ausgedrückt sei, möchte zuletzt Untersuchung verdienen. Hierüber jedoch ist unter allen Forschern fast Eine Meinung. Den Begriff der mächtigen, der starken Götter drücke der Kabiren-Name aus, nach der Bedeutung eines gleichlautenden hebräischen Worts. Alle andere Bedenklichkeiten gegen diese Erklärung\footnote{Nämlich aus dem hebr. \foreignlanguage{hebrew}{\<kab*iyr>}, mächtig stark. Der für diese Erklärung stimmender ist eine große Anzahl, \emph{Scaliger ad Varr.} und \emph{ad chron. Euseb. Gerh. Voss. de Idolol. p. 173. Bochart. G. S Grotius in Schol. ad Matth 4, 24 Selden de Diis Syris Synt. 2. p. 287. 361. Marsham canon. chron. p 35. Gutberleth l. c.} und alle Neuern. Die Hauptzweifel gegen diese Erklärung sind, dass die unbedingte Bedeutung von mächtig, wenigstens ohne das Arabische zu Hülfe zu nehmen, nicht erweislich ist; überall scheint es nur den Begriff des durch Überfluss Mächtigen und Starken anzudeuten, s. \emph{Job. 31, 25. 8, 2.} Entscheidender ist die vermißte \emph{proprietas verbi}, indem es von göttlicher Stärke und Größe niemals gebraucht wird. Man könnte das verwandte \foreignlanguage{hebrew}{\<gbr>} zu Hülfe nehmen, wovon \foreignlanguage{hebrew}{\<gb*yor>} in der Zusammensetzung mit \foreignlanguage{hebrew}{\<'l>} gebraucht wird; \emph{Gebhurah} ist eine von den Eigenschaften Gottes \emph{1 Par. 29, 2.} und eine der zehen Sephiroth; \emph{2 Sam. 22, 2.} übersetzt \emph{Targ. Jon.} die Worte (Jehovah fährt) auf dem Cherub durch \foreignlanguage{hebrew}{\<bgbwrwtyh>}, entweder nach dem Sinn oder zufolge des syrischen $\symbolAAR$, dass ein Wörterbuch durch \emph{fortis, validus} erklärt. Allein das Wort, das eigentlich dem \emph{Kabir} entsprechen sollte, \emph{Gebbir}, hat keine Beziehung auf göttliche Kraft. \emph{Spencer. de leg. vet. Hebr. ritualibus 2. p. 848.} erinnert bei den Cherubim an die Kabiren; das Gemeinschaftliche scheint ihm die Stärke. Es ließen sich aber wohl nähere Beziehungen auffinden. Wie sie auch immer gestaltet sein mochten, jene rätselhaften Wesen; beschrieben werden sie als Gestalten, über denen der höchste Gott ruht, s. \emph{1 Sam. 4. 4.}, also als untergeordnete Wesen, als Camille, wenn man sie menschlich nehmen wollte, denen jedoch ebenfalls mächtige Kräfte inwohnen. Auf den ungefähren Gleichlaut wird sich niemand berufen wollen; der erinnerte eher an die nächsten Verwandten der Kabiren, die Korybanten; ein Name, der nicht leicht anderen, als morgenländischen Ursprungs sein kann.} überwindet beinah' dass Eine, dass eben diese Götter immer und überall, zusammen und auch einzeln, die Großen, die Mächtigen genannt werden.\footnote{\emph{In Augurum libris Divi potes sunt, in Samothraki} θεοὶ δυνατοί. \emph{Varro de l. l. 50. 4.} Vgl. \emph{Cassius Hemina} bei \emph{Macr. Sat. 3. 4.}; eine Menge Inschriften, wovon einige bei Gutberleth.} Dennoch was bürgt dafür, dass eben dieser Begriff durch das Wort Kabiren ausgedrückt worden? Es bleibt immer der nicht grundlose Zweifel, dass eigentlich nur die höheren Götter des kabirischen Systems die Großen genannt worden.\footnote{Auffallend ist immer die Wortverbindung \emph{Aen. 3, 12.}\\\hspace*{15mm}--- --- --- \emph{feror exul in altum}\\\hspace*{10mm}\emph{Cum sociis, natoque, Penatibus et magnis Dîs.}\\\hspace*{5mm}Das \emph{et declarative} genommen ist matt. Es bleibt nichts übrig, als \emph{Penates} für die den großen Göttern vorangehende (insofern von ihnen unzertrennliche, aber doch verschiedene) zu nehmen, womit auch die allein wahrscheinliche Etymologie (s. Anm. 72.) übereinstimmt.} Aber auch zu allgemein, nicht bezeichnend genug für den eigentümlichen Begriff, lautet der Name, um sich durch die erste Ähnlichkeit hinreißen zu lassen. Von selbst dringt uns die Untersuchung, die Eigentümlichkeiten der Kabiren nochmals in Einem Bilde zusammen zu fassen. Die ersten Kabiren also waren magische oder bestimmter zu reden theurgische, die höheren Götter zur Wirkung bringende Kräfte oder Naturen.\footnote{Der Hauptbeweis dieser Behauptung liegt in den Namen und der Aufeinanderfolge. Dass sie aber allgemein für zauberkräftige Naturen angesehen worden, dafür nur einige Nachweisungen! Unmittelbare Abkömmlinge der Kabiren, Korybanten oder Samothrakier (dies nimmt er alles für gleichbedeutend) sind nach Sanchuniathon, \emph{Euseb. p. 36.} die die Kenntnis der Kräuter, Heilung giftiger Bisse und die Beschwörungen zuerst erfunden. Strabo \emph{50. 10. p. 466.} sagt, nach einigen sei'n die Korybanten, die Kabiren, die Idäischen Dactylen und die Telchinen einerlei, nach andern Verwandte und nur durch geringe Unterschiede voneinander getrennt. Von den Idäischen Dactylen aber sagt \emph{Schol. Apoll. Paris. 50. 1. v. 1131.} γόητες δὲ ἦσαν καὶ Φαρμακεῖς; und auch hier war Zauber und Gegenzauber. Nämlich die linken, wie Pherecydes lehrte, waren unter ihnen die γόητες, die den Zauber knüpfenden, die rechten aber die den Zauber lösenden. Einige lehrten, die rechten (Finger, Dactylen) sei'n männlich, die linken weiblich. Von denselben sagt der ephemerisirende \emph{Diod. Sic. 5. p. 392.} da sie Zauberer gewesen, haben sie sich der Beschwörungen, Einweihungen und Geheimlehren beflissen und auf Samothraki verweilend die Einwohner durch dies alles in nicht geringes Erstaunen gesetzt, fast gleichlautend mit manchen Erzählungen von Othin und seinen Gesellen. --- Über die Zauberkräfte der Telchinen. S. \emph{Diod. 5. c. 55. Strab. 14. p. 653. extr. Hesych. h. v.} u. a.} Doch nicht einzeln, nur in ihrer unauflöslichen Folge und Verkettung üben sie den Zauber aus, durch den das Überweltliche in die Wirklichkeit gezogen wird. Nun stehen auch die durch sie zur Offenbarung gebrachten Götter mit ihnen wieder in einer magischen Verknüpfung. Die ganze Kabiren-Reihe bildet also eine vom Tiefsten bis ins Höchste reichende Zauberkette. Kein Glied dieser Kette darf unwirksam sein oder austreten, soll nicht der Zauber verschwinden. Wie nicht die Erscheinung Eines Dioskuren, wie nur das Zeichen der zwei vereinten Flammen den Seefahrern heilbedeutend ist, so sind die Kabiren nur zusammen die großen heilbringenden Götter und werden nicht einzeln, sondern nur gemeinschaftlich verehrt.\footnote{Schon wegen dieses erwiesenen Grundbegriffes (der ihrer Natur nach untrennlichen) war von der vermeinten Entdeckung verschiedener Epochen in der Geschichte des samothrakischen Dienstes kein Gebrauch zu machen. \emph{Sainte-Croix p. 28. ss.} behauptet, es sei'n in S. nur zwei Gottheiten verehrt worden (Himmel und Erde); darauf sei'n ägyptische und phönikische Vorstellungen hinzugekommen; hierauf habe man angefangen, die alten samothrakischen Gottheiten mit den griechischen zu verwechseln, aus der einen sei Ceres geworden, aus der andern Proserpina, aus der dritten Pluton, aus der vierten, erst von Ägypten hergekommenen, Merun: eine dritte Epoche will er aus der Stelle des Plinius (Anm. 46.) herausbringen. Das Gelindeste, was sich darüber sagen lässt, ist, dass es lauter willkührliche und unerwiesne Vorstellungen sind. Er versichert zwar: "`\emph{Diodore nous dit en termes claires, que le culte de Samothraki fut restauré, mais que les raisons de cet événement n'étoient connues, que des seuls adeptes},"' und dazu \emph{Diod. 50. 5. c. 49.} Wahrscheinlich ist die Stelle c. 48. gemeint (Δία --- παραδεῖξαν αὐτῷ (τῷ Ἰασίονι) τὴν τῶν μυστηρίων τελετὴν, πάλαι μὲν οὖσαν ἐν τῇ νήσω, τότε δέ πως παραδοθεῖσαν, ὧν (\emph{sc.} μυστηρίων) οὐ θέμις ἀκοῦσαι πλὴν τῶν μεμυημένων, wo die lat. Übersetzung hat, \emph{ritus antea quidem in insula receptos, sed tum traditione venovatos}. Eben so wenig zu benutzen war der Heyne'sche Fund, \emph{Exc. 9. ad Aen. 2.} "`\emph{quæ, ut simpliciora, ita probabiliora habeo, hæc sunt, magnas intercessisse mutationes harum religionum; ab initio fuisse Cœlum et Terram; postea accessisse duos alios: nunc quatuor ista numina, domesticis nominibus insignita} (der Schol. des Ap. nennt sie noch mit ihrem ältesten Namen), \emph{interpretatione varia ad diversos Deos Græcorum referri cœpisse: igitur in illis memorari videas Cererem et Proserpinam, Haden et Mercurium, ab aliis Bacchum et Jovem; etiam Vulcanum in iis quæsitum, item Cybelen; nec improbabile fit, adscita fuisse hæc ipsa sacra seriori ætate};"' also im Grunde eben das, was \emph{Sainte-Croix}. Dieses atomistische Verfahren, etwas, das als ein Ganzes nicht begreiflich scheint, durch Zusammenstückelung zu erklären, sollte sich wenigstens immer auf gründliche Beweise stützen. Aber wenn Athenion z. B., bei \emph{Schol. Ap. l. c.}, von nur zwei Kabiren spricht, geschieht es auf eine Art, die zeigt, dass er vom Eigentlichen nicht redet oder es nicht kannte, die zwei Kabiren sind ihm Jasion und Dardanus. Und doch führt ihn \emph{Sainte-Croix} als Gewährsmann dafür an, dass nur zwei Gottheiten gewesen. Sind Jasion und Dardanus vielleicht auch \emph{Cœlum} und \emph{Terra}? Höchst lehrreich dagegen sind Varros Äußerungen, aus denen, weil leichthin gelesenen oder wenigstens missverstandenen, eigentlich zuerst jene Meinung aufgebracht worden, in S. habe man anfänglich nur zwei Gottheiten verehrt. So lautet die Hauptstelle im Zusammenhang 50. 4. p. 17. \emph{Immortalia et mortalia expediam, ita ut prius, quod ad Deos pertinet, dicam.} (Aus dieser \emph{præfatio} erhellt, dass die zunächst folgende Worte eigene Philosophie des Varro sind.) \emph{Principes Dei Cœlum et Terra. Hi Dei iidem, qui in Aegypto Serapis et Isis et [...]ste Harpocrates digito (qui) significat} (hier wird den zweien gleich eine dritte beigesellt; ob diese Worte der Vers eines Dichters, etwa des Lucilius, sind oder eigene des Varro, ist gleichgültig --- \emph{iidem}), \emph{qui sunt Taautes et Astarte apud Phœnicas, ut îdem principes in Latio, Saturnus et Ops. Terra enim et Cœlum, ut Samothracum initia docent, sunt Dei magni et hi, quos (modo) dixi multeis nominibus. Nam neque, quas Ambracia} (so hat die römische Handschrift; \emph{vulg. Samothracia}, offenbar gegen den Zusammenhang, der anch Scalinger's \emph{Imbrasia} zurückweist) \emph{ante portas statuit duas virileis species aheneas, Dei magni, neque, ut volgus putat, hi Samothrakis Dii, qui Castor et Pollux: sed hi (scil. Samothrakis Dii sunt) mas et femina} (also verschieden von jenen in Ambracia zu sehenden zwei männlichen Gestalten), \emph{et hi (s. iidem), quos augurum libri scriptos habent sic: Divi potes, et sunt (scil. hi, potes dicti) pro illeis qui in Samothraki} θεοὶ δυνατοί. \emph{Hæc duo, cœlum et terra, quod Anima et Corpus, Humidum et Frigidum.}"' Die recht gelesene Stelle sagt offenbar nur dieses: Himmel und Erde sind die ersten, alles anfangenden, Gottheiten; diese werden in Ägypten durch Serapis und Isis, in Phönikiern durch Astarte und Taaut, in Samothrakien ebenfalls durch eine männliche und weibliche Gottheit repräsentiert. Varro sagt also nicht einmal, die Zweizahl sei die älteste Form der Lehre (Kreuzer 2. 291. Anm.), sondern nur die \emph{principes Dei} seien zwei; gesetzt, es war wirklich seine Absicht, alle Gottheiten auf eine ihnen zu Grunde liegende Zweiheit philosophisch zurückzuführen, so ist doch zwischen zwei ersten oder allen zu Grunde liegenden Gottheiten und zwischen zwei zuerst allein angenommenen ein deutlicher Unterschied. Dass die gegebene Erklärung von \emph{principibus Deis} (anfangenden Gottheiten) nicht aus der Luft gegriffen ist, erhellt ausfolgenden auch für sich merkwürdigen Stellen und die manchem auch neuerlich kund gewordenen Missverstand heilsame Arzenei sein könnten. "`\emph{Saturnus unus est de principibus Deis}"' (\emph{op. August. de Civ. Dei 50. 7. c. 9.}). "`\emph{Saturnus pater a Jove filio est superatus}"' (\emph{ib. c. 19.}), "`\emph{Juppiter Deus est habens potestatem causarum, quibus aliquid fit in mundo. Ei praeponitur Janus, quoniam penes Janum sunt prima, penes Jovem summa. Merito ergo Rex omnium Juppiter habetur. Prima enim Vincuntur a summis, quia licet prima praecedunt tempore, summa superant dignitate.}"' \emph{ib. c. 19.} Zu vergleichen ist damit der Ausdruck Jovis \emph{consiliarii et principes} in den unter Anm. 115. angef. Worten. Diese letzten Stellen des römischen Gelehrten möchten wohl auch Licht geben, inwiefern einige, wie \emph{Schol. Apoll. l. c.} erwähnt, sagen konnten, es gebe, vorzugsweise (so muss wohl πρότερον übersetzt werden, wenn man es nicht auf Φασί bezieht), nur zwei Kabiren, Zeus, den älteren, und Dionysos, den jüngeren (von beiden); die einzige Stelle, die man für die ursprüngliche Zweizahl, doch nicht von \emph{Cœlum} und \emph{Terra}, anführen könnte. Die dies sagten (es waren nur Einige) hatten aus dem tiefsten geschöpft. Hier kann nämlich nicht der erste Dionysos (s. Anm. 80.), sondern nur der Höchste gemeint sein, dem selbst Zeus vorangeht. So ließ sich wohl sagen: es gebe eigentlich nur zwei Kabiren; denn diese zwei waren die höchsten, auf die alles hinausging, die gleichsam allein übrigblieben, \emph{a quibus}, mit Varro zu reden, \emph{reliqui omnes superabantur}. Aus den zuletzt angeführten Stellen des Varro folgt wohl auch, dass wenigstens in dem Buch (\emph{de Diis select.}), aus dem sie genommen sind, \emph{mit principibus} oder \emph{primis Diis} zugleich höhern, jene, wie er sagt, überwindende, anerkannt waren, es folgt also, dass er die Reihe fortgesetzt dachte und nicht einmal in seiner Philosophie bei der Zweizahl stehen blieb. Hinwiederum konnte ihn diese Fortsetzung nicht hindern, doch wieder alles (die ganze Reihe) auf den Grundgegensatz von Männlichem und Weiblichem dualistisch zurückzuleiten. Denn alle Philophie wird auf eine solche allem zu Grunde liegende Zweiheit geführt, ohne darum zu behaupten, dass nur zwei Wesen sei'n. Auch nennt er anderwärts (\emph{Augustin. l. c. c. 27.}) \emph{Cœlum et Terram} nur die \emph{duo principia Deorum}, und Augustinus sagt von ihm nur (\emph{c. eod. in.}) "`\emph{Vir doctissimus et acutissimus Varro hos omnes Deos in cœlum et terram redigere ac referre conatur}."' Er macht ihm dort (\emph{p. 62. in. ed. Paris}) den, jedoch wahrscheinlich auf Missverstand beruhenden Vorwurf: \emph{istum in libro selectorum Deorum rationem illam trium} (nicht \emph{duorum}) \emph{Deorum, quibus quasi cuncta complexus est, perdidisse.} Diess bezieht sich auf folgende Ansicht, die wir ebenfalls durch Augustinus (ebendas.) als Varronisch kennen, und die sich mit dem Gedanken, alles auf eine Zweiheit zurückzuleiten, gar wohl verträgt. "`\emph{Ducitur (Varro) quadam ratione verosimili, cœlum esse quod faciat, terram quæ patiatur, et ideo illi masculinam vim tribuit, huic femininam. --- --- Hic etiam Samothracum nobilia mysteria in superiore libro sic interpretatur, eaque se, quæ nec suis nota sunt, scribendo expositurum eisque missurum quasi religiosissime pollicetur. Dicit enim, se ibi multis indiciis collegisse in simulacris, aliud significare cœlum, aliud terram, aliud exempla rerum, quas Plato appellat Ideas.}"' Hier behauptet Varro entschieden die Dreinicht die Zweizahl in den samothrakischen Vorstellungen. Es leidet auch keinen Zweifel, dass aus den angegebenen drei Grundbegriffen die ganze Reihe zu entwickeln oder umgekehrt, die ganze Folge der Zahlen auf die Dreiheit zurückzubringen ist (\emph{redigenda}, wie es oben ausgedrückt wurde). Nämlich alle kabirische Wesen sind nur fortdaurende Steigerungen, so dass dieselbe Zahl oder Persönlichkeit in verschiedenen Potenzen wiederkehrt; alle Zahlen demnach auf gewisse, und zwar unstreitig drei, Grundzahlen zurückkommen. So ist jene Dreiheit das (wie er sich ausdrückt) \emph{de quo}, dass \emph{à quo} und dass \emph{secundum quod aliquid fiat}, offenbar eben die Reihe, welche die drei ersten samothrakischen Gottheiten wirklich bilden. Demeter-Persephone (in der wahren Zählung gelten beide nur für Eins) ist das \emph{de quo}, Dionysus das \emph{à quo}, Kadmilos dass \emph{secundum quod aliquid fit}, und der unbefangene Forscher wird gerne gestehen, dass die Varronische Auslegung für das Ganze unserer Ansicht keine geringe Bestätigung ist. Jene allen zu Grunde liegende Dreizahl kann indes fortschreitend sich wiederholen, und so wie die Drei Persephone, Dionysos, Kadmilos sein können, so können sie auch Juno, Jupiter, Minerva sein, wie sie Varro gleich nach der eben angeführten Stelle erklärt. Varros Äußerungen verdienten wohl diese ausführliche Erörterung. Was ein solcher Mann, der Gelehrteste nicht nur seines Volks, sondern vieler Zeiten und der an Ort und Stelle, bei noch bestehendem samothrakischem Dienst, alles aufs genaueste erforscht hatte, über Sinn und Bedeutung desselben urteilte, das ist wohl auch jetzt noch entscheidend. Zugleich wird dieses hinreichen zum Beweis, dass die Angabe: es sei'n von Anfang nur zwei Kabiren gewesen und die andern erst in der Folge zufällig dazu gekommen, alles geschichtlichen Grundes ermangelt, dass also nicht einmal nötig ist, deshalb an das übrigens sehr Wahrscheinliche zu erinnern, dass nicht allen alle Zahlen (oder Grade) mitgeteilt und vielen vielleicht wirklich nur zwei Persönlichkeiten bekannt wurden. Aber der das System zuerst besaß, musste es ganz besitzen, durch Zusammenstückelung konnte es nimmer entstehen. Die Kabirenreihe bildet eine unauflösliche Folge, sie ist ein seiner Natur nach unteilbares System.} Also den Begriff der unauflöslich (wie Dioskuren) Vereinigten und zwar der magisch Vereinigten musste der Name bedeuten, wenn er vollkommen ihre gemeinschaftliche Natur ausdrücken sollte. Hätte man nun für den gegebenen Begriff den Namen zu erfinden, so wäre kfootnoteein mehr bezeichnender auszudenken als der der Kabiren, sobald man denselben von einem andern Wort der nämlichen Sprache ableitet, welches zugleich den Begriff einer unauflöslichen Vereinigung und den des Zaubers in sich schließt.\footnote{Diese Wurzel ist \foreignlanguage{hebrew}{\<.hbr>}, \emph{consociavit, conjunxit se}. Davon \foreignlanguage{hebrew}{\<.hAberiym>} \emph{socii}, und zwar mit dem bestimmten Begriff, dass Mehrere wie Ein Mann sind. So die entscheidende Stelle \emph{Jud. 20, 2.} \foreignlanguage{hebrew}{\<.hbrym k'y/s '.hd>}. Dieses ist die gewöhnliche Form. Aber auch \foreignlanguage{hebrew}{\<.hbyrym>} kommt vor, in einer gleich anzuführenden Stelle. Den beiden Formen scheint die doppelte griechische Schreibung, κάβειροι (\emph{Caberi}; in gedruckten Büchern sieht man wohl auch καβήρους) und κάβιροι zu entsprechen. Es wäre davon genug, aber folgende Stelle eines jüdischen Buchs beweist noch auf ganz besondre Weise die Eigentlichkeit des Ausdrucks; vom Metatron, den wir schon als den kabbalistischen Kadmilos kennen gelernt, heißt es (Eisenm. entd. Jud. T. 2. S. 401.) \foreignlanguage{hebrew}{\<m.t.trwN /shw' gbwh m.hbyryw tq /skh hk.sr>}, "`der Knabe Metatron, der 500 Jahr (auch dieser Ausdruck nicht unbedeutend) höher ist, als seine \emph{Chabkirim}, d. h. als seine Genossen, Gesellen."' Kaum enthält man sich an die sächsische Abschwörungsformul zu denken, in \emph{Eccard. monum. catech. p. 78.} "`\emph{end eo forsacho --- --- Woden end Saxu-Ote, en de allen unholdum, the hira genotas sint},"' allen Unholden, die ihre Genossen sind. Ich weiß nicht, welche Meinung gerade in Lehrbüchern oder anderen nach herrschenden Ansichten eingerichteten Schriften über die jüdische Philosophie oder Kabbala angenommen ist; aber so viel getraue ich aus noch ziemlich oberflächlicher Bekanntschaft mit derselben zu beweisen, dass sie Trümmer und Überbleibsel enthält, sehr entstellte wenn man will, aber doch Überbleibsel jenes Ursystems, das der Schlüssel aller religiösen Systeme ist, und dass die Juden nicht ganz unwahr reden, wenn sie die Kabbala für Überlieferung einer Lehre ausgeben, die außer der in den schriftlichen Urkunden vorhandenen, geoffenbarten (eben darum offenbaren), als umfassenderes aber geheimes, nicht allgemein mitgeteiltes noch mitteilbares, System vorhanden war. Sehr erwünscht muss daher dem Kenner die eben von Wien kommende Ankündigung eines hebräisch-rabbinischen Werks erscheinen, das alle Lehrmeinungen des \emph{Ben Jochoi}, Verfassers des ebenso berühmten als wichtigen Werkes Sohar aus den Quellen zu sammeln verspricht. Möchte ein jüdischer oder andrer Gelehrter Unterstützung genug finden, den ganzen Sohar herauszugeben und auch andere Quellen zu öffnen! Es ist fast traurig zu sehen, wie man auch in diesen Forschungen von den wahren Quellen so ganz sich abgewendet hat. In Ägyptens selbst dunkeln und unenträtselbaren Hieroglyphen hat man den Schlüssel alter Religionen suchen wollen, jetzt ist von nichts als Indiens Sprache und Weisheit die Rede; aber die hebräische Sprache und Schriften, zuvorderst des A. T., in welcher die Wurzeln der Lehre und selbst der Sprache aller alten religiösen Systeme, bis ins Einzelnste deutlich erkennbar sind, liegen unerforscht. Sehr zu wünschen ist, dass diese ehrwürdigsten Denkmäler bald aus den Händen bloßer Theologen in die der reinen Geschichtsforscher übergehen, da sie hoffen dürften, dieselbe unbefangene Würdigung zu erfahren, und als Quellen doch wenigstens eben so viel zu gelten, als die Homerischen Gedichte oder Herodots Erzählungen. Es soll damit nicht gesagt sein, dass der rein geschichtliche Forscher und der Theolog nicht in Einer Person vereiniget sein könne. In diesem muss doch zuletzt alles zusammentreffen. Aber vorerst wird es schwerhalten, obwohl man anfängt, allgemein einzusehen, dass Dogmen mit Gewalt hinein und sie mit Gewalt heraus erklären, in der wahren Schätzung keinen Unterschied macht, nur dass das Letzte, wie alles blos negative Verfahren, viel eher zur Unduldsamkeit und zur Seichtigkeit führt, als das erste, welches wenigstens ein positiv verbindendes, Zusammenhang suchendes Bestreben ist. Der Name Kabir, oder eigentlich \emph{Chabir} ist ein alttestamentliches Wort, das zugleich unteilbare Verbindung und magische Vertettung ausdrückt. Das bekannte \foreignlanguage{hebrew}{\<.hober .hAbEr>} \emph{Deut. 18, 2.} heißt, wörtlich und barbarisch übersetzt, \emph{nectens nexum, consocians consociationem}, aber der Sinn ist, \emph{magicam exercens artem, incantator}; daher auch \foreignlanguage{hebrew}{\<.ha:bAriym>} schlechtweg \emph{incantationes}, \emph{Es. 47, 9. 12.} Diese mit dem, Wort \foreignlanguage{hebrew}{\<.hbr hbr>} verbundene Bedeutung lässt sich nicht denken ohne eine Vorstellung, ähnlich oder gleich der, die dem System der Kabiren zu Grunde liegt. Bei den vielen Bemühungen, die Bedeutung des Kabiren-Namens zu enträtseln, musste wohl auch einer oder der andere auf jene Wurzel fallen. Der Erste, soviel ich weiß, \emph{Anton. Astorius in Diss. de Diis Cabiris, Venet. 1703. insert. Poleni Suppl. etc. T. 2. p. 873. ss.} Er begnügt sich aber §. 11. mit der Bedeutung von Zauberern, die \foreignlanguage{hebrew}{\<.hbrym>} heißen, \emph{quia se jungunt dæmonibus}, und beweist dann mit Anwendung der Anm. 3. erwähnten Stellen, doch nicht ohne manche Verworrenheit, die Kabiren seien (weder die \emph{Penates Dii}, noch Dämonen, noch Naturgottheiten im Varronischen Sinn, sondern) bloße menschliche Zauberer gewesen, welche den Menschen zuerst den Götter- und Dämonendienst, die Mittel sie zu gewinnen und zu zwingen gelehrt haben, darauf aber selbst für Götter gehalten worden sein. Dem Adr. Reland, \emph{Diss. de Diis Cab. in ej. Dissertt. misc. 1. pag. 191. ss.}, war bei Durchlesung der Schrift von Astori eingefallen, dass man \emph{Cabiri} nach jenem hebräischen Wort, wohl auch durch \emph{juncti, socii}, übersetzen könne, wobei sich ihm gleich die Dioskuren, Δίδυμοι, \emph{Gemini, socialia sidera} glücklich anboten. Aber bald störte dies Glück, dass außer den Dioskuren auch Ceres, Proserpine, Pluto, Mercur Kabiren sind; es wird ein Versuch gemacht, auch in diesen etwas zu finden, wodurch sie \emph{socii} sind, es findet sich, dass sie alle θεοἱ χθόνιοι sind und mit den Toten beschäftiget, aber auch Zeus und nach \emph{Dionys. Hal. Arch. 1. 3.} auch Apollo sind Kabiren, hier hört also die Erklärung auf, von der zugestanden wird, sie tauge nur, wenn man sich mit einem für die Dioskuren und jene \emph{Deos inferos} zugleich passenden \emph{Etymon} begnügen wolle, für alle aber passe \foreignlanguage{hebrew}{\<gbyrym>}, denn das drücke \emph{Deos potes} aus. Sehr natürlich wäre die Erwartung, Spuren des Kabiren-Namens auch in andern morgenländischen Sprachen zu finden. Zunächst bietet sich an, was \emph{Buxt. Lex. chald. talm. p. 704.} erwähnt. \emph{In Talmud sæpe vocantur Sacerdotes Persarum} \foreignlanguage{hebrew}{\<.hbrym>}, \emph{vel Persae in genere ita vocantur, ut in Jevammoth fol. 63, 2. Ad Ps. 14. "`dicit stultus in corde suo"' R. Jochanan dixit} \foreignlanguage{hebrew}{\<'lh .hbrym>}. \emph{Isti sunt Chaverim. Gloss.} \foreignlanguage{hebrew}{\<.hbrym>}, \emph{i. e. Persæ impii, qui non agnoscunt gloriam Israëlitarum}. --- --- \emph{Dixit R. Immi ad R. Levi: Ostende mihi Persas; dixit, similes sunt exercitibus} (?\foreignlanguage{hebrew}{\<.sb'wt>}) \emph{domus Israëlis, ostende mihi} \foreignlanguage{hebrew}{\<hbryn>}, \emph{dixit, Similes sunt Angelis vastatoribus etc. Baal Arach scribit: Persæ vocabant Sacrificulos. Sacerdotes} \foreignlanguage{hebrew}{\<.hbrym>} \emph{et fuerunt isti} \foreignlanguage{hebrew}{\<.hbrym>} \emph{pessimi, graviterque affligentes Israëlem.}"' Eben dahin gehört, was \emph{Hyde H. r. v. P. p. 365.} anführt, \emph{Origenes contra Celsum meminit} Περσῶν ἢ Καβείρων. Wie man nun in jenen \emph{Chabherin}, welche die Perser selbst und ihre Priester sein sollen, nicht umhinkann, die jetzt noch so genannten Ghebern zu erkennen, so muss auch wohl dieser Name zuletzt mit dem Kabiren-Namen einer und derselbe sein. Ganz eigen ist die von Kreuzer 2. 287. aus \emph{Fouchersar la rel. des Perses, Ac. des Inscr. T. 29}. und Anh. zum Zend-Avesta 1. 217. angeführte Meinung: Kabiren seien nicht als Gabirim, starke Männer; Menschen oder Götter, die mit Schmieden und Bearbeiten der Metalle sich abgeben. Nach der Parsischen Feuer-Religion waren die Schmiede unrein, weil sie das Feuer entweihten. Daher der verächtliche Begriff, der mit dem Wort Ghebr verbunden war und bis auf den heutigen Tag im ganzen Morgenlande fortdauert. Also Ghebern sind eigentlich Schmiede, die das Feuer verunreinigen, und mit diesem Namen werden eben die belegt, welche der alten Parsenlehre treu das Feuer heilighalten und anbeten, und die ihnen diesen verächtlichen Namen beilegen, sind die mohammedanischen Perser, die also wohl das Feuer für heilig achten! Oder werden die Anbeter des reinen Feuers, blos um sie zu ärgern, Schmiede genannt, welche ausgesuchte Bosheit! Erträglicher drückt sich Hyde aus: \emph{Alterum a Mahommedanis (et ante eos ab aliis) hinc populi (veterum Persarum reliquiis) impingitur Epitheton Ghebr S. Guebr, i. e. in genere Infideles, in specie Ignicolæ}. Er scheint doch anzunehmen, dass die altgläubigen Parsen nicht von den Muhammedanern erst den Namen erhalten. Doch weil er heutzutag eine nachteilige Bedeutung hat, glaubt er den Namen von jeher verächtlich gebraucht. Da aber das Wort $\symbolAAS$ nach muhammedanisch-persischen Wörterbüchern nichts anders als eben \emph{ignicola} bedeutet, so kann es wohl im Munde der Muhammedaner verächtlich, den Parsis selbst nur ehrenwert sein. Unstreitig ist es ihr uralter Name, der in der persischen Sprache so wenig als $\symbolAAT$ ein \emph{Etymon} hat, also auf einen auswärtigen Ursprung hindeutet, der nirgends als in \foreignlanguage{hebrew}{\<.hbr>} zu finden ist. Eine Religion des Feuers und der Feuer-Magie ist auch der Kabirismus, der jedoch in Persien verschiedne Umwandelungen erlitt, bis er durch die letzte von Zoroaster unternommene sich in das Licht- und Feuer-Gesetz Ormusd's verlor. Der sonst unter den Türken auch für Juden und Christen gebräuchliche Name \emph{Gaur} hat mit dem der Ghebern nichts gemein, er ist verdorben aus \emph{Kafur, Plural.} von $\symbolAAU$, \emph{infidelis}, s. \emph{Langlès} zu \emph{Chardin Voy. en P. 8. p. 356. not.} Unter den Handschriften des \emph{Musei Borgiani} befindet sich die Übersetzung eines indischen Buchs, \emph{Mulpanci} genannt, welches der Übersetzer \emph{P. Marco a Tumba} durch \emph{libro della radice o libro del fondamento} erklärt; in diesem soll der besondre Lehrbegriff der cabiristischen Sekte Indiens enthalten sein. S. \emph{P. Paulini a St. Barth. Mus. Borg. Vel. Cod. Mss. p. 158.} Eine deutsche Übersetzung des ersten Gesangs, vom Herrn Bischof Münter mitgeteilt, findet sich im 3. Band der Fundgruben des Orients. Es ist aber nicht viel mehr daraus abzunehmen, als was schon \emph{Georgii, Alph. Tibet. p. 98.}, sagt, "`\emph{Brahmanes Cabiritarum Verbo vim creandi tribuunt}"' und \emph{P. Marco} bei \emph{Paul p. 61.} \emph{Li Cabiristi non credono altro Dio, che la virtù generativa, che dicono essere in ogni cosa del mondo}. Von Lehrsätzen, die eine besondere Verwandtschaft mit dem Kabirensystem beurkundeten, keine Spur! Da nun auch vom \emph{P. Paulino p. 159.} der indische Name durch \emph{Cabì vel Cavì} ausgedrückt wird so kann nur ein indischer Gelehrter wissen, inwiefern daraus Cabiriten oder eine Cabiristische Secte Indiens gemacht werden kann. Das sanskredamische \emph{Cavì} oder \emph{Cabì} bedeutet nach \emph{P. Paul.} so viel als \emph{homo doctus, poëta}, und dann ferner, \emph{theologus, sapiens}. Damit würde die Talmudische Bedeutung von \foreignlanguage{hebrew}{\<.hbr>} übereinstimmen, \emph{Buxt. p. 703. 704.} "`\emph{Doctoribus Hebræorum priscis dicebatur} \foreignlanguage{hebrew}{\<.hAbEr>} \emph{Magister s. Rabbi recens creatus} --- \emph{sed generaliter etiam idem quod} \foreignlanguage{hebrew}{\<.hkm tlmyr>}, \emph{Doctor s. sapiens. Hinc in libro Cosri sapiens ille, qui Regi Persarum interroganti respondet, vocatur} \foreignlanguage{hebrew}{\<.hbr>}; auch das arab. $\symbolAAV$ \emph{scivit, cognovit}, wovon $\symbolAAW$, \emph{sciens gnarus}, selbst in der Bedeutung von \emph{omniscius, qui et præterita et futura novit}, der alle Dinge im Zusammenhang kennt, Beiwort von Gott. Indisch Gelehrte mögen Aufschluss geben, inwiefern das indische \emph{Cabì} oder \emph{Cavì} wirklich einen Bezug zu diesem Worte hat.} Wird nun schon durch das genaue Zusammentreffen von Wort und Sache diese Erklärung die wahrscheinlichere, so erhebt sie zur Gewissheit eine unerwartet aber umso bestätigender hinzutretende Ähnlichkeit. Ein Götter-Rat, also ein zusammengehöriges Ganzes von Göttern, findet sich bei den alten Etruskern, unbekannt waren die einzelnen Namen,\footnote{Hieraus erhellt wenigstens, wie ungewiss es ist, dass diese \emph{senatores Deorum}, wie sie bei \emph{Mart Cap. de nupt. Phil. 50. 1. p. 16.} genannt werden, die zwölf Götter der bekannten Verse des Ennius (\emph{Apul. de deo Socr. p. 225.}) sein.} aber sie alle zusammen hießen \emph{Consentes} und \emph{Complices},\footnote{\emph{Arnob. adv. Gent. 3. p. 123. Varro, qui sunt introrsus atque in intimis penetralibus cœli, Deos esse censet, quos loquimur} (nämlich \emph{Penates}, denn von diesen ist im ganzen Zusammenhang die Rede), \emph{nec eorum numerum nec nomina sciri. Hos Consentes et Complices Etrusci ajunt et nominant, quod unâ oriantur et occidant unâ; sex mares et totidem feminas, nominibus ignotis et memorationis parcissimæ: sed eos summi Jovis consiliarios et principes existimari.} Die Bedeutung von \emph{complices} ist klar, wie die von \emph{consentes}, man mag nun dies auf \emph{consentiri} beziehen oder von \emph{consum} ableiten, wie \emph{præsentis} von \emph{præsum}. Die Lesart \emph{memorationis parcissimæ} hat eine Handschrift für sich. S. \emph{Coll. Var. lectt. subj. ed. cit.} Das aus der römischen Ausg. und der ihr zu Grunde gelegten Hds. aufgenommene \emph{miserationis} ist hier sinnlos. Sonst findet sich das Wort \emph{Consentes} nur in Inschriften, z. B. \emph{Jovi O. M. ceterisque Dîs consentibus}, auch \emph{Mercurio Consenti} bei \emph{Maffei p. 238.} Im Werk \emph{de re rust. 1. 1.} unterscheidet Varro, "`\emph{Deos consentes urbanos, quorum imagines (Romæ) ad forum auratæ stant}"' von den \emph{consentibus rusticis}, deren aber fünf männliche, sieben weibliche sind; offenbar jedoch nur Nachbildungen jener ursprünglichen. Dass die \emph{Dii Consentes} nicht mit den \emph{selectis} einerlei waren, erhellt aus \emph{Aug. de civ. D. 4. 23. Quis enim ferat, quod neque inter Deos consentes, quos dicunt in consilium Jovis adhiberi, nec inter Deos, quos Selectos vocant Felicitas constituta sit.} Auch für den Begriff: \emph{Senatores, consiliarii sit.} \foreignlanguage{hebrew}{\<.hbrym>} der eigentliche Ausdruck. Auf hebräisch-samaritanischen Münzen heißen die Glieder des dem Hohenpriester beigegebenen hohen Rats (des \emph{Synedrii}) seine \foreignlanguage{hebrew}{\<.hbrym>}. S. u. a. \emph{Tychsen in Commentt. Gott. 11. 155.}} welches nur Erklärung, ja wörtliche Übersetzung des Kabiren-Namens ist, wofern ihm die von uns angenommene Bedeutung zugeschrieben wird. Es waren ihrer sechs männliche und sechs weibliche Wesen, außer diesen aber, dem sie gemeinschaftlich untergeordnet waren, Jupiter. Denkt man an die Geschlechts-Doppelheit aller alten Gottheiten, nicht eben dass in einerlei Wesen widernatürlich beide Geschlechter vereiniget waren, sondern dass jede Persönlichkeit oder so zu sagen jede Stufe in der Götterfolge durch eine männliche und weibliche Gottheit zugleich bezeichnet war,\footnote{Vgl. oben Anm. 31.} so entdeckt sich auch hier wieder jene kabirische, sich in Jupiter als Einheit auflösende Siebenzahl.\footnote{Hiebei muss nämlich Jupiter in doppelter Beziehung gedacht werden, einmal, sofern er selbst einer der sieben, dann sofern Zeus, wie die Orphiker sagen, der Anfang, Zeus das Mittel und Zeus das Ende ist.} Verschiedene Götter, sind sie doch zusammen wie Einer.\footnote{Also im Grunde, wenn man die gehörige Unterscheidung hinzudenkt, wie Elohim in der Mehrzahl mit dem Zeitwort in der Einzahl verbunden wird.} Dorthin, nach dem Tuseerland hatten, wie geschichtlich bekannt ist, Pelasgische Pflanzer ihre Götter gebracht, an Laviniums Küste schiffte Aeneas die trojanischen Penaten aus. Und von eben diesen tuskischen Göttern versichert Varro, \emph{Complices} sei'n sie genannt worden, weil sie nur miteinander leben und miteinander sterben können. Unmöglich wäre, zu diesem Ausdruck etwas hinzuzusetzen oder den wahren Gedanken jener vereinigten Götter trefflicher zu bezeichnen. Und so wird hinwiederum die erforschte Bedeutung des Namens urkundlich für den inneren Sinn des Kabirensystems, ein Zeugnis unserer zuerst aus der Folge dieser Götter entwickelten Erklärung. Darstellung des unauflöslichen, in einer Folge von Steigerungen vom Tiefsten ins Höchste fortschreitenden Lebens, Darstellung der allgemeinen Magie und der im ganzen Weltall immer daurenden Theurgie, durch welche das Unsichtbare ja Überwirkliche unablässig zur Offenbarung und Wirklichkeit gebracht wird, das war ihrem tiefsten Sinn nach die heilig geachtete Lehre der Kabiren. In diesen Ausdrücken freilich wurde sie dort, in Samothraki, schwerlich vorgetragen; ohnehin hatte die Einweihung in die Geheimnisse mehr die Absicht, sich für Leben und Tod den höheren Göttern zu verbinden, als Aufschluss über das Weltall zu erhalten. Als theurgische Mittel dieser Verbindung wurden die unteren Götter betrachtet und als solche auch verehrt. Nicht abwärts, in die sichtbare Welt herein, aufwärts erstreckte sich dieser Zauber. Der Eingeweihte wurde durch die empfangenen Weihen selbst ein Glied jener magischen Kette, er selber ein Kabir, aufgenommen in den unzerreißbaren Zusammenhang und wie die alte Geschichte sich ausdrückt dem Heer der oberen Götter zugesellt.\footnote{Ἡς στρατιῆς εἷς ἐιμι, λαχὼν θεὸν ἡγεμονῆα, bei Münter S. 8. Darum hießen die samothrakischen Geheimnisse vorzugsweise ἄῤῥηκτα, die unzerreißbaren. \emph{Orph. Argon. 464. 465.}\\\hspace*{15mm}ζαθεὴν Σαμοθρᾐκην\\\hspace*{5mm}Ἔνθα καὶ ὅρκια Φρικτὰ θεῶν, ἄῤῥηκτα βροτοῖσιν,\\\hspace*{5mm}wo Bruuck \emph{ad Apoll. 1. 917.} und nach ihm Hermann erstens sehr unnötig ὅρκια in das gewöhnliche ὄργια, zweitens ἄῤῥηκτα in ἄῤῥητα verwandelt und so grade die eigentümliche Farbe in diesen Stellen verwischt haben. Schon dass sie ungewöhnlicher, also die schwerere Lesart ist, sollte die hergebrachte schützen; welcher Abschreiber wäre wohl unwissend genug, dass ihm nicht für ὅρκια, ὄργια, für ἄῤῥηκτα, ἄῤῥητα einfiele, welcher gelehrt genug, um sich das Gegenteil einfallen zu lassen? Ὅρκια \emph{subint.} ἱερὰ heißen die kabirischen Weihen ihrer Eigentümlichkeit wegen; sie waren Aufnahme in den Bund der Götter (man denke nur immer an den alt- und neutestamentlichen Begriff dieses Worts, \emph{Berith} hieß ihr ältester Sitz in Phönikiern). Auf einen Bund deuteten auch die Binden (ταινίαι), das Abzeichen des Eingeweihten, \emph{Schol. Ap. l. c.} Wie hier, ist auch \emph{Ap. 1. 917.} ἄῤῥηκτα das eigentümliche Wort.} In dem Sinn mochten die Kabiren oder ihre Diener wohl Erfinder von Zaubergesängen heißen, wie Sokrates sagt, das Kind in uns müsse immerfort beschworen und wie mit Zaubergesängen geheilt werden, bis es von der Todesfurcht frei sei.\footnote{\emph{Plat. in Phæd. p. 77. 177. Bip.}} Von der einen Seite ganz auf die Gesinnung und das Leben sich hinrichtend mochte die eigentliche Lehre von den anderen ganz versinnlichet werden, wahrscheinlich indem man den Chor der Götter durch den Reigen der Gestirne darstellte.\footnote{In zwei Schaaren sind aber gesondert die Seelen der Toten,\\\hspace*{5mm}Eine die unstet irret umher auf der Erde, die Andre,\\\hspace*{5mm}Welche den Reigen beginnt mit den leuchtenden Himmelsgestirnen,\\\hspace*{5mm}Diesem Heere bin ich gesellt, denn der Gott war mein Führer.\\\hspace*{15mm}Samothr. Inschr. nach Münter.} Und welch' herrlicheres Sinnbild des Grundgedankens ließ sich erfinden, als die unauflöslich verbundene Bewegung dieser Himmelslichter, in deren Chor kein Glied fehlen kann ohne Zusammensturz des Ganzen, von denen aufs eigentlichste zu sagen ist, dass sie nur zusammen geboren werden und nur zugleich mit einander sterben können! Vieles (doch wissen wir es nicht) mochte im Lauf der Zeiten verhüllt, Manches (ein Schicksal vieler höheren und besseren Lehren) getrübt und der Bedeutung beraubt werden. Aber welche Verhüllungen sie gelitten, welche Richtungen genommen haben mag, unzerstörlich blieb der Grundgedanke, unverkennbar das Ganze der ursprünglichen Lehre, ein aus ferner Urzeit geretteter Glaube, der reinste und der Wahrheit ähnlichste des ganzen Heidentums. Nicht ganz unwürdig aber schien der Feier dieses Tages der Versuch, einen Glauben hoher Vergangenheit zu enträtseln. Denn Erforschung des Vergangenen erfüllt deu größten Teil aller wissenschaftlichen Arbeit. Ob es die ältesten Züge, Kriegstaten und Verfassungen der Völker sind, die erforscht werden, ob das Bild untergegangner Schöpfungen der reichen Natur aus fast unkenntlichen Abdrücken wiederhergestellt wird, ob die Spuren des Wegs aufgesucht werden, dem die Erde in ihrer Entwickelung gefolgt ist; immer gehen diese Nachforschungen auf Zeiten der Vergangenheit. Von allem Forschungswürdigen bleibt aber das Würdigste, was einst Menschen innerlich vereint, worin Tausende und zum Teil die Besten ihrer Zeit die höchste Weihe des Lebens erkannt. In den späteren Zeiten römischer Kaiserherrschaft wurde der einst heilige Name der Kabiren durch Schmeichelei entweiht; auf Münzen erscheinen nicht bloß die Brustbilder des frommen Antoninus oder des Marcus Aurelius, auch der Kopf eines Domitian mit der Umschrift kabirischer Gottheiten.\footnote{\emph{Eckhel Doctr. Num. Vet. 3. p. 375. ss.}} Uns wäre die schönste Anwendung des Namens in dem Augenblick verstattet, da er zugleich an jenes kabirische Bündnis erinnert, durch welches erst die Macht gebrochen, endlich die letzten Zuckungen erstickt worden eines wahrhaft typhonischen Reichs, das nur zu enden gedroht in allgemeiner Entsittlichung. Aber jeden fremderen Ausdruck, jede künstlichere Wendung stößt das einfache Gefühl von sich, in dem wir uns des allgeliebten Königes freuen und in welches wie in die heißen Gelübde für Sein lang dauerndes Wohl wir mit Seinem ganzen Volk einstimmen.
\clearpage
\section{Anmerkungen.}
\paragraph{}
Jeder, der nicht Fremdling ist in Forschungen dieser Art, wird, auch unversichert, von selbst glauben, leichter und angenehmer hätte der Verfasser den Stoff der folgenden Anmerkungen in den Text selbst verarbeitet, als nun von ihm ausgeschieden. Aber etymologische Forschungen, und solche, bei denen es auf Vergleichung von Stellen und Worten alter Schriftsteller ankommt, eignen sich nicht für einen öffentlichen Vortrag, zumal vor gemischten Zuhörern. Den aus der Sonderung entstandenen Nachteilen musste also der Verfasser sich unterwerfen. Zuerst dem, dass manche Behauptung gradezu aufgestellt worden, die allmählich eingeleitet und aus den einzelnen hieher verwiesenen Untersuchungen Schritt für Schritt entwickelt leichter Eingang finden konnte. Sodann dass die Anmerkungen den Text überschwellen, ja mitunter wohl sich ganz unabhängig von ihm zu machen scheinen. In Bezug auf solche Fälle bemerke ich daher, dass manches angeführt werden musste, dass nicht der einzelnen Erklärung, sondern dem ganzen System von Erklärung zur Stütze dient, dass hier zuerst angewendet worden. Einiges musste insofern über den Text hinaus zu gehen scheinen, das doch wirklich nötig ist, ihn zu begründen. Wer daher über das Ganze der Ansicht urteilen will, wird umso weniger umhinkönnen, den Anmerkungen ein eignes Studium zu widmen. Wenn auf die sprachlichen Erörterungen fast zu ängstlicher Fleiß verwendet scheinen sollte, so ist dem Verf. angenehmer, deshalb getadelt als wegen des Gegenteils gelobt zu werden; denn solche Untersuchungen, wenn nicht mit Strenge und oft peinlicher Sorgfalt getrieben, sind gar Nichts.
\clearpage
\section{Nachschrift.}
\paragraph{}
Die voranstehende Abhandlung gehört ihrer ursprünglichen Bestimmung nach zu einer Reihe von Werken, die sich auf die Weltalter als gemeinschaftlichen Mittelpunkt beziehen. Dass sie durch eine äußere Veranlassung früher erscheint, konnte jene Bestimmung nicht ändern, und als Beilage jenes Werks wird sie darum in dem weiteren Umkreis erscheinen, in den sie sich durch den Buchhandel verbreitet. Dieser Zusatz hebt ihre Selbstständigkeit nicht auf, da man ihr hoffentlich zugestehen wird, auch ganz für sich und ohne alle Beziehung existieren zu können. Nicht an sich, nur der Intention des Verfassers nach Beilage eines andern Werks, ist sie zugleich Anfang und Übergang zu mehreren andern, deren Absicht ist, das eigentliche Ursystem der Menschheit, nach wissenschaftlicher Entwickelung, wo möglich auf geschichtlichem Weg, aus langer Verdunkelung an's Licht zu bringen. Denn untrennlich von Geschichte ist die dis zu einem gewissen Punkt gelangte Wissenschaft und fast notwendig der Übergang der einen in die andre. Nicht zufällig geht der allgemeineren Untersuchung die besondere des samothrakischen Systems voran; es war Absicht, dieses zum Grunde zu legen; denn wie gemacht zum Schlüssel aller übrigen ist durch hohes Alter wie durch Klarheit und Einfachheit ihrer Umrisse die Kabirenlehre. So viel also über den weiteren Zusammenhang dieser Abhandlung, die übrigens ganz für sich genommen werden muss, und auch ganz für sich, nach ihrem besonderen Inhalt, geprüft sein.
\end{document}
